\documentclass{beamer}
% definitions for slides for the semantics course - CSC501
% Lutz Hamel, (c) 2006

\usetheme{Warsaw}
\usepackage{bussproofs}
\usepackage{amsmath}
\usepackage{amssymb}
\usepackage{latexsym}
\usepackage{xypic}
\usepackage{alltt}
\usepackage{rotating}
\usepackage{graphicx}

\newcommand{\ul}[1]{\underline{#1}}
\newcommand{\orbar}{\,|\,}
\newcommand{\co}{\,\colon\;}
\newcommand{\syntaxset}[1]{\ensuremath{\mbox{\bf #1}}}
\newcommand{\nonterm}[1]{\ensuremath{\mbox{#1}}}
\newcommand{\term}[1]{\ensuremath{\mbox{\bf #1}}}
\newcommand{\ifstmt}[3]{\ensuremath{{\bf if}\; {#1}\;{\bf then}\;{#2}\;{\bf else}\;{#3}\;\term{end}}}
\newcommand{\whilestmt}[2]{\ensuremath{{\bf while}\; {#1}\;{\bf do}\;{#2}\; \term{end}}}
\newcommand{\funcstmt}[3]{\ensuremath{{\bf fun}\; {#1}\; {\bf is}\; {#2} \; {\bf return}\; {#3}}}
\newcommand{\localstmt}[1]{\ensuremath{{\bf local}\; {#1}}}
\newcommand{\pairmap}[3]{\ensuremath{( {#1}, {#2} ) \mapsto {#3}}}
\newcommand{\single}[1]{\ensuremath{\langle {#1}\rangle}}
\newcommand{\pair}[2]{\ensuremath{({#1}, {#2} )}}
\newcommand{\triple}[3]{\ensuremath{\langle {#1}, {#2}, {#3} \rangle}}
\newcommand{\cond}[3]{\ensuremath{({#1}?\,{#2} : {#3})}}
\newcommand{\condbar}[3]{\ensuremath{({#1}\rightarrow{#2} \mid {#3})}}
\newcommand{\sem}[2]{\ensuremath{{#1}-\!\!-\!\!\!\!\gg{#2}}}
\newcommand{\liftfunc}[1]{\ensuremath{\lfloor{#1}\rfloor}}
\newcommand{\bcode}{\begin{alltt}\scriptsize}
\newcommand{\ecode}{\end{alltt}}
\newcommand{\comment}[1]{}

\newcommand{\fframe}[1]{
\begin{center}
\fbox{
\begin{minipage}{3.6in}
{#1}
\end{minipage}
}
\end{center}
}







\begin{document}

\begin{frame} [fragile] {Simple I/O}

Here we extend our language with  a simple I/O mechanism that allows
us to initialize variables and examine variables in the executing program:
\begin{quote}
\begin{alltt}
C ::=  put(A)  |  get(x)
\end{alltt}
\end{quote}
The informal semantics is that 'put' allows us to write an expression to the terminal and 
'get' allows us to initialize a declared variable with an integer value read from the terminal.
\end{frame}

\begin{frame} [fragile] {Simple I/O}

The formal semantics is as follows:

\vspace{.1in}

{\scriptsize
\begin{alltt}
(put(A),State) -->> State :-           %io%  writing
    (A,State) -->> ValA,
    write(A),
    write(' is '),
    writeln(ValA),!.

(get(X),State) -->> OState :-          %io% reading
    lookup(X,State,_),
    write('Enter integer value for '),
    write(X),
    write(': '),
    read(Val),
    int(Val),
    bindval(X,Val,State,OState),!.
\end{alltt}
}
\end{frame}

\begin{frame} [fragile] {Simple I/O}
Now we can write programs such as these:
\scriptsize
\begin{alltt}
?- ['sem-block.pl'].
%   xis.pl compiled 0.01 sec, 7,792 bytes
%  preamble.pl compiled 0.01 sec, 8,956 bytes
%  xis.pl compiled 0.00 sec, 148 bytes
% sem-block.pl compiled 0.01 sec, 18,284 bytes
true.

?- ((var(x) seq get(x) seq put(x)),s) -->> V. 
Enter integer value for x: 3.
x is 3
V = push([bind(3, x)], s).

?- ((var(x) seq get(x) seq put(add(x,1))),s) -->> V.
Enter integer value for x: 5.
add(x,1) is 6
V = push([bind(5, x)], s).

?- 
\end{alltt}
\end{frame}

\begin{frame}[fragile]{Block Structured Languages}

\scriptsize
\begin{itemize}
\item
In most languages a block introduces a new scope allowing for {\em local variable declarations}.
In C blocks are introduced with the curly braces,
{\tiny
\begin{verbatim}
{
   int x;
}
\end{verbatim}
}
\item
We can access the values
of variables in non-local scope.  Consider the following code,
{\tiny
\begin{verbatim}
{
   int x = 2;
   {
      int y = 3;
      x = y + x;  /* accessing the surrounding scope via 'x' */
   }
}
\end{verbatim}
}

\item
In most languages blocks can be nested $\Rightarrow$ {\em variable shadowing}.
{\tiny
\begin{verbatim}
{
   int x = 1;
   {
      int x = 2;  /* the original 'x' is no longer visible in this scope */
   }
}
\end{verbatim}
}
 
\end{itemize}


\end{frame}



\begin{frame}{Block Structured Languages}
Recall that in our simple language we have variable declarations and now we introduce
blocks:
\begin{quote}
\begin{alltt}
C ::=  var(x)  |  block(C)
\end{alltt}
\end{quote}
Think of 'block' as 'begin C end' where C could be any command including sequential composition.
\end{frame}

\begin{frame}{Block Structured Languages}
Going back to our observations on block structured languages
\begin{itemize}
\item
Local variable declarations.  When we leave a scope with local variables those 
variables should become undeclared:
\begin{alltt}
\scriptsize
  var(x) seq block( var(y) seq assign(y,1) ) seq assign(y,x)
\end{alltt}
\item
Non-local side effects. When assigning a value to a variable declared in a surrounding scope
we need to update the value of that variable, the value printed out for x should be 2:  
\begin{alltt}
\scriptsize
  var(x) seq assign(x,1) seq block( assign(x,2) ) seq put(x)
\end{alltt}

\item
Variable shadowing.  Redeclaring a variable in a nested scope with the same name as
a variable in the outer scope makes the variable in the outer scope unavailable,
the value printed out for x should be 1:
\begin{alltt}
\scriptsize
  var(x) seq assign(x,1) seq block( var(x) seq assign(x,2) ) seq put(x)
\end{alltt}
\end{itemize}

\end{frame}

\begin{frame}[fragile]{Block Structured Languages}

\small
Formal Semantics:
\begin{alltt}\scriptsize
(var(X),State) -->> OState :-          % decl,
    {\color{red}declarevar}(X,State,OState),!.

(assign(X,A),State) -->> OState :-     % assignment
    lookup(X,State,_),                 % only allowed to assign to variables
    (A,State) -->> ValA,               % that have been declared
    {\color{red}bindval}(X,ValA,State,OState),!.

(block(C),State) -->> OState :-        %block%     block statement
    {\color{red}pushenv}(State,LocalState),
    (C,LocalState) -->> S,
    {\color{red}popenv}(S,OState),!.
\end{alltt}

\vspace{.1in}

{\bf Note:} Each block now pushes its own binding environment on an environment stack.

\vspace{.1in}

{\bf Note:} The new semantic procedures 'declarevar' and 'bindval' are necessary because
declaring and binding is done differently with nested scopes.
\end{frame}

\begin{frame}[fragile]{Block Structured Languages}
Semantic procedures 'pushenv' and 'popenv':
\begin{alltt}
\scriptsize
%%%%%%%%%%%%%%%%%%%%%%%%%%%%%%%%%%%%%%%%%%%%%%%%%%%%%%%%%%%%%%%%%%%%%
% the predicate 'pushenv(+State,-FinalState)' pushes
% a new binding term list on the stack
:- dynamic pushenv/2.

pushenv(S,env([],S)) :- !.

%%%%%%%%%%%%%%%%%%%%%%%%%%%%%%%%%%%%%%%%%%%%%%%%%%%%%%%%%%%%%%%%%%%%%
% the predicate 'popenv(+State,-FinalState)' pops
% a  binding term list off the stack
:- dynamic popenv/2.

popenv(env(_,S),S) :- !.
\end{alltt}
\end{frame}

\begin{frame}[fragile]{Block Structured Languages}
Looking up variable bindings in a stack of binding environments:
\begin{alltt}\scriptsize
% the predicate 'lookup(+Variable,+State,-Value)' looks up
% the variable in the state and returns its bound value.
:- dynamic lookup/3.                % modifiable predicate

lookup(_,s0,_) :- fail.

lookup(X,env([],S),Val) :-
    lookup(X,S,Val),!.

lookup(X,env([bind(Val,X)|_],_),Val).

lookup(X,env([_|Rest],S),Val) :- 
    lookup(X,env(Rest,S),Val),!.
\end{alltt}
\end{frame}


\begin{frame}[fragile]{Block Structured Languages}
Semantic procedure 'declarevar':
\begin{alltt}\scriptsize
% the predicate 'declarevar(+Variable,+State,-FinalState)' declares
% a variable by inserting a new binding term into the current
% environment.
:- dynamic declarevar/3.                   % modifiable predicate

declarevar(X,S,env([bind(0,X)],S)) :-
    atom(S),!.

declarevar(X,env(L,S),env([bind(0,X)|L],S)) :- !.
\end{alltt}
\end{frame}

\begin{frame}[fragile]{Block Structured Languages}
Semantic procedure 'bindval':
\begin{alltt}\scriptsize
% the predicate 'bindval(+Variable,+Value,+State,-FinalState)' updates
% a binding term in the state.  this update is done "in place"
% in order to support global variables.  the predicate has to
% search both the binding list and the stack of binding
% lists.
:- dynamic bindval/4.                   % modifiable predicate

bindval(_,_,s0,_) :- 
    fail.

bindval(X,Val,env([],S),env([],NewS)) :-
    bindval(X,Val,S,NewS),!.

bindval(X,Val,env([bind(_,X)|L],S),env([bind(Val,X)|L],S)),!.

bindval(X,Val,env([bind(V,Y)|L],S),env([bind(V,Y)|NewL],NewS)) :-
    bindval(X,Val,env(L,S),env(NewL,NewS)),!.
\end{alltt}
\end{frame}

\begin{frame}[fragile]{Block Structured Languages}
\begin{alltt}\tiny
?- ['sem-block.pl'].
%   xis.pl compiled 0.00 sec, 7,792 bytes
%  preamble.pl compiled 0.00 sec, 8,956 bytes
%  xis.pl compiled 0.00 sec, 148 bytes
% sem-block.pl compiled 0.00 sec, 18,192 bytes
true.

?- ((var(x) seq block( var(y) seq assign(y,1) ) seq assign(y,x)),s) -->> V. 
false.

?- ((var(x) seq assign(x,1) seq block( assign(x,2) ) seq put(x)),s) -->> V.
x is 2
V = env([bind(2, x)], s).

?- (( var(x) seq assign(x,1) seq block( var(x) seq assign(x,2) ) seq put(x)),s) -->> V. 
x is 1
V = env([bind(1, x)], s).

?- 
\end{alltt}
\end{frame}
\end{document}

