\documentclass{beamer}
% definitions for slides for the semantics course - CSC501
% Lutz Hamel, (c) 2006

\usetheme{Warsaw}
\usepackage{bussproofs}
\usepackage{amsmath}
\usepackage{amssymb}
\usepackage{latexsym}
\usepackage{xypic}
\usepackage{alltt}
\usepackage{rotating}
\usepackage{graphicx}

\newcommand{\ul}[1]{\underline{#1}}
\newcommand{\orbar}{\,|\,}
\newcommand{\co}{\,\colon\;}
\newcommand{\syntaxset}[1]{\ensuremath{\mbox{\bf #1}}}
\newcommand{\nonterm}[1]{\ensuremath{\mbox{#1}}}
\newcommand{\term}[1]{\ensuremath{\mbox{\bf #1}}}
\newcommand{\ifstmt}[3]{\ensuremath{{\bf if}\; {#1}\;{\bf then}\;{#2}\;{\bf else}\;{#3}\;\term{end}}}
\newcommand{\whilestmt}[2]{\ensuremath{{\bf while}\; {#1}\;{\bf do}\;{#2}\; \term{end}}}
\newcommand{\funcstmt}[3]{\ensuremath{{\bf fun}\; {#1}\; {\bf is}\; {#2} \; {\bf return}\; {#3}}}
\newcommand{\localstmt}[1]{\ensuremath{{\bf local}\; {#1}}}
\newcommand{\pairmap}[3]{\ensuremath{( {#1}, {#2} ) \mapsto {#3}}}
\newcommand{\single}[1]{\ensuremath{\langle {#1}\rangle}}
\newcommand{\pair}[2]{\ensuremath{({#1}, {#2} )}}
\newcommand{\triple}[3]{\ensuremath{\langle {#1}, {#2}, {#3} \rangle}}
\newcommand{\cond}[3]{\ensuremath{({#1}?\,{#2} : {#3})}}
\newcommand{\condbar}[3]{\ensuremath{({#1}\rightarrow{#2} \mid {#3})}}
\newcommand{\sem}[2]{\ensuremath{{#1}-\!\!-\!\!\!\!\gg{#2}}}
\newcommand{\liftfunc}[1]{\ensuremath{\lfloor{#1}\rfloor}}
\newcommand{\bcode}{\begin{alltt}\scriptsize}
\newcommand{\ecode}{\end{alltt}}
\newcommand{\comment}[1]{}

\newcommand{\fframe}[1]{
\begin{center}
\fbox{
\begin{minipage}{3.6in}
{#1}
\end{minipage}
}
\end{center}
}







\begin{document}

\begin{frame}[fragile]{Grammars}

\begin{itemize}
\item
Grammars play a crucial role in programming languages because they precisely capture the
syntactic nature of programming languages.

\item
We start our discussion of grammars by looking at the nature of sequences of symbols,
where sequences of symbols form the foundation of any language, both natural and
artificial.

\item
We will call sequences of symbols {\em strings}.

\end{itemize}


\end{frame}

\begin{frame}[fragile]{Strings}

\small

{\bf Definition:} [Strings over an Alphabet]\footnote{\tiny Based on material from the book ``An Introduction to the Theory of Computation,'' Eitan Gurari, Ohio State University,Computer Science Press, 1989.}
\begin{itemize}
\item
An {\em alphabet} is a finite, nonempty set -- we refer to the elements of an alphabet as {\em symbols}.
\item
A finite sequence of symbols from a given alphabet is called a {\em string over the alphabet}.
\item
A string that consists of a sequence $a_1, a_2,\ldots, a_n$ of symbols is denoted by the juxtaposition $a_1a_2\ldots a_n$.
\item
The length of some string $s$  is denoted by $|s|$ and assumed to equal the number of symbols in the string.
\item
Strings that have zero symbols, called {\em empty strings}, are denoted by $\epsilon$ with $|\epsilon|=0$.
\end{itemize}
\end{frame}

\begin{frame}[fragile]{Strings}

\small

{\bf Example:}    Let $\Gamma_1 = \{a, \ldots, z\}$ and $\Gamma_2 = \{0,\dots, 9\}$ is alphabets. $abb$ is a string over $\Gamma_1$, and $123$ is a string over $\Gamma_2$. $ba12$ is neither a string over $\Gamma_1$ nor a string over $\Gamma_2$ but it is a string over $\Gamma_1\cup\Gamma_2$.
The string $314 \ldots$ is not a string over $\Gamma_2$, because it is not a finite sequence.

\vspace{.1in}

Some other observations:
\begin{itemize}
\item The empty string  $\epsilon$ is a string over any alphabet.
\item The empty set $\emptyset$ is not an alphabet because it contains no element.
\item The set of natural numbers is not an alphabet, because it is not finite.
\end{itemize}
\end{frame}

\begin{frame}[fragile]{Strings}

{\bf Definition:} [Kleene Closure] Given some alphabet $\Gamma$ then the set of all possible strings over
$\Gamma$ including the empty string $\epsilon$ is denoted by $\Gamma^*$ and is called the {\em Kleene
Closure of $\Gamma$}.  (Similar to the power set construction with the exception that the constructed set is always infinite.)

\vspace{.1in}
{\bf Example:} Let $\Gamma = \{ a \}$, then $\Gamma^* = \{\epsilon, a, aa, aaa, aaaa,\dots\}$.

\vspace{.1in}
{\bf Example:} Let $\Gamma = \{a,b\}$, then
\[
\Gamma^* = \{\epsilon,a,b,aa,bb,ab,ba,aaa,aab,\ldots\}.
\]
\end{frame}

\begin{frame}[fragile]{Strings}

{\bf Definition:} [String Concatenation] Given some alphabet $\Gamma$ with $s_1\in\Gamma^*$ and $s_2\in\Gamma^*$,
then the {\em concatenation of the strings} written as $s_1s_2$ also belongs to $\Gamma^*$, that is
the string $s_1s_2\in\Gamma^*$.
\end{frame}

\begin{frame}[fragile]{Grammars}

\small

{\bf Definition:} [Context-Free Grammar] A {\em context-free grammar} is a triple $(\Gamma,\rightarrow,\gamma)$ such that,
\begin{itemize}
\item
$\Gamma = T \cup N$ with $T\cap N = \emptyset$, where $T$ is a set of symbols called the {\em terminals} and $N$ is a set of symbols
called the {\em non-terminals},\footnote{The fact that $T$ and $N$ are non-overlapping means
that there will never be confusion between terminals and non-terminals.}
\item
$\rightarrow\subseteq N\times\Gamma^*$ is a set of rules of
the form $u \rightarrow v$ with $u\in N$ and $v\in\Gamma^*$,
\item
$\gamma$ is called the {\em start symbol} and $\gamma\in N$.
\end{itemize}
\end{frame}

\begin{frame}[fragile]{Grammars}

\small
{\bf Example:} Grammar for arithmetic expressions.  We define the grammar $(\Gamma, \rightarrow, \gamma)$ as follows:
\begin{itemize}
\item $\Gamma = T \cup N$ with $T = \{ a, b, c, +, *, (, )\}$ and $N=\{ E \}$,
\item $\rightarrow$ is is defined as,
{\scriptsize
\[
\begin{array}{rcl}
E & \rightarrow & E + E \\
E & \rightarrow & E * E \\
E & \rightarrow & ( E )\\
E & \rightarrow & a\\
E & \rightarrow & b\\
E & \rightarrow & c
\end{array}
\]
}
\item $\gamma = E$ (clearly this satisfies $\gamma\in N$).
\end{itemize}
\end{frame}





\begin{frame}[fragile]{Rewriting Relation}

\small
In order for a grammar $(\Gamma,\rightarrow,\gamma)$ to be useful we allow rules
to be applied to {\em substrings} of given strings;
let $s = xuy$,$t=xvy$ with $x,y,v \in \Gamma^*$, $u\in N$, and a rule $u\rightarrow v$,
then we say that {\em $s$ rewrites to $t$} and we write,
\[
s \Rightarrow t.
\]
More formally,

\vspace{.1in}
{\bf Definition:} [one-step rewriting relation] Let $(\Gamma,\rightarrow,\gamma)$ be a be context-free grammar,
then the {\em one-step rewriting relation} $\Rightarrow\subseteq \Gamma^*\times\Gamma^*$ is  the set with $(s,t) \in \Rightarrow$ for strings $s,t \in\Gamma^*$
 if and only if there exist $x, y, v \in\Gamma^*$ and $u\in N$ with $s = xuy, t = xvy$, and $u\rightarrow v$.

\vspace{.1in}
In plain English: any two strings $s,t$ belong to the relation $\Rightarrow$ if and only if they can be related by
a rewrite rule.
\end{frame}



\begin{frame}[fragile]{Rewriting Relation}
With grammars, derivations always start with the start symbol $\gamma \in \Gamma^*$. Consider,
\[
E\Rightarrow E * E \Rightarrow ( E ) * E \Rightarrow ( E + E ) * E \Rightarrow (a + E) * E \Rightarrow (a + b) * E
\Rightarrow (a + b) * c.
\]
Here, $(a+b)*c$ is a {\em normal form} often also called a {\em terminal-} or {\em derived-string}.
Normal forms are characterized by the fact that no additional rewriting can be applied to them.

We often write,
\[
E \Rightarrow^* (a + b) * c
\]
stating that the normal form is derivable from the start symbol in zero or more steps.

\end{frame}




\begin{frame}[fragile]{Grammars}

{\bf Exercise:} Identify the rule that was applied at each rewrite step in the above derivation.

{\bf Exercise:} Derive the string $((a))$.

{\bf Exercise:} Derive the string $a + b * c$.

\end{frame}

\begin{frame}[fragile]{Grammars}

\small
We are now in the position to define exactly what we mean by a {\em language}.

\vspace{.2in}
{\bf Definition:}[Language] Let $G = (\Gamma,\rightarrow,\gamma)$ be a
context-free grammar, then we define the {\em language of
grammar $G$} as the set of all terminal strings that can be derived from the start symbol $\gamma$ by rewriting
using the rules in $\rightarrow$.  Formally\footnote{Observe that $T^*$ is the set of all terminal strings.},
\[
L(G) = \{ q \mid \gamma \Rightarrow^* q \wedge q\in T^*\}.
\]

\vspace{.3in}
{\bf Example:} Let $J = (\Gamma,\rightarrow,\gamma)$ be the grammar of Java, then $L(J)$ is the set of all possible Java
programs.  The Java language is defined as the set of all possible Java programs.
\end{frame}


\begin{frame}[fragile]{Grammars}

\small
{\bf Observations:}
\begin{itemize}
\item
With the concept of a language we can now ask interesting questions.  For example,
given a grammar $G$ and some sentence $p\in T^*$, does $p$ belong to $L(G)$?
\item
If we let $J$ be the grammar of Java, then asking whether some string $p\in T^*$ is in $L(J)$
is equivalent to asking whether $p$ is a {\em syntactically correct program}.
\end{itemize}

\end{frame}



\begin{frame}[fragile]{Grammars}

\scriptsize
{\bf Example:} A simple imperative language.  We define grammar $G=(\Gamma,\rightarrow,\gamma)$ as follows:
\begin{itemize}
\item $\Gamma = T \cup N$ where
{\tiny
\[
T =\{\term{0},\ldots,\term{9},\term{a},\ldots,\term{z},\term{true},\term{false},\term{skip},
\term{if},\term{then},\term{else},\term{while},\term{do},\term{end}+,-,*,=,\le,!,\&\&,||,:=,;,(,)\}
\]
}
and
{\tiny
\[
N = \{A,B,C,D,L,V\}.
\]
}
\item The rule set $\rightarrow$ is defined by,
{\tiny
\[
\begin{array}{rcl}
\nonterm{A} &\rightarrow& D \orbar V \orbar \nonterm{A} +\nonterm{A} \orbar \nonterm{A} - \nonterm{A} \orbar
	\nonterm{A} * \nonterm{A}\orbar ( \nonterm{A} )\\

\nonterm{B} &\rightarrow& \mbox{\bf true} \orbar \mbox{\bf false} \orbar \nonterm{A} = \nonterm{A} \orbar
	\nonterm{A} \le \nonterm{A} \orbar ! \nonterm{B} \orbar \nonterm{B} \&\& \nonterm{B} \orbar
	\nonterm{B} || \nonterm{B} \orbar (\nonterm{B})\\

\nonterm{C} &\rightarrow& \mbox{\bf skip} \orbar \nonterm{V} := \nonterm{A} \orbar\nonterm{C}\; ; \nonterm{C} \orbar
	\mbox{\bf if} \; \nonterm{B}\; \mbox{\bf then}\;\nonterm{C}\; \mbox{\bf else}\; \nonterm{C} \; \term{end}\orbar
	\mbox{\bf while}\; \nonterm{B}\; \mbox{\bf  do}\; \nonterm{C}\; \term{end}\\

\nonterm{D} &\rightarrow& \nonterm{L} \orbar - \nonterm{L}\\

\nonterm{L} &\rightarrow& \term{0}\, \nonterm{L} \orbar \ldots \orbar  \term{9} \,\nonterm{L} \orbar \term{0} \orbar \ldots\orbar\term{9}\\

\nonterm{V} &\rightarrow& \term{a}\,\nonterm{V} \orbar \ldots\orbar \term{z}\,\nonterm{V} \orbar \term{a} \orbar \ldots \term{z}
\end{array}
\]
}
\item $\gamma =\nonterm{C}$.
\end{itemize}
\end{frame}

\begin{frame}{Grammars}
Here are some elements in $L(G)$,
\[
\begin{array}{l}
x := 1 ; y := x\\
v := 1; \ifstmt{v \le 0}{v := (-1)*v}{\term{skip}}\\
n := 5 ; f := 1 ; \whilestmt{2 \le n}{f := n*f ; n := n - 1}
\end{array}
\]
{\bf Exercise:} Prove that they belong to $L(G)$.

\end{frame}

\begin{frame}{Grammars}

Reading: Denotational Semantics/Schmidt -- pages 5 thru 8.
Assignment \#1 -- see BrightSpace

\end{frame}



\end{document}
