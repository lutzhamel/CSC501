\documentclass{beamer}
% definitions for slides for the semantics course - CSC501
% Lutz Hamel, (c) 2006

\usetheme{Warsaw}
\usepackage{bussproofs}
\usepackage{amsmath}
\usepackage{amssymb}
\usepackage{latexsym}
\usepackage{xypic}
\usepackage{alltt}
\usepackage{rotating}
\usepackage{graphicx}

\newcommand{\ul}[1]{\underline{#1}}
\newcommand{\orbar}{\,|\,}
\newcommand{\co}{\,\colon\;}
\newcommand{\syntaxset}[1]{\ensuremath{\mbox{\bf #1}}}
\newcommand{\nonterm}[1]{\ensuremath{\mbox{#1}}}
\newcommand{\term}[1]{\ensuremath{\mbox{\bf #1}}}
\newcommand{\ifstmt}[3]{\ensuremath{{\bf if}\; {#1}\;{\bf then}\;{#2}\;{\bf else}\;{#3}\;\term{end}}}
\newcommand{\whilestmt}[2]{\ensuremath{{\bf while}\; {#1}\;{\bf do}\;{#2}\; \term{end}}}
\newcommand{\funcstmt}[3]{\ensuremath{{\bf fun}\; {#1}\; {\bf is}\; {#2} \; {\bf return}\; {#3}}}
\newcommand{\localstmt}[1]{\ensuremath{{\bf local}\; {#1}}}
\newcommand{\pairmap}[3]{\ensuremath{( {#1}, {#2} ) \mapsto {#3}}}
\newcommand{\single}[1]{\ensuremath{\langle {#1}\rangle}}
\newcommand{\pair}[2]{\ensuremath{({#1}, {#2} )}}
\newcommand{\triple}[3]{\ensuremath{\langle {#1}, {#2}, {#3} \rangle}}
\newcommand{\cond}[3]{\ensuremath{({#1}?\,{#2} : {#3})}}
\newcommand{\condbar}[3]{\ensuremath{({#1}\rightarrow{#2} \mid {#3})}}
\newcommand{\sem}[2]{\ensuremath{{#1}-\!\!-\!\!\!\!\gg{#2}}}
\newcommand{\liftfunc}[1]{\ensuremath{\lfloor{#1}\rfloor}}
\newcommand{\bcode}{\begin{alltt}\scriptsize}
\newcommand{\ecode}{\end{alltt}}
\newcommand{\comment}[1]{}

\newcommand{\fframe}[1]{
\begin{center}
\fbox{
\begin{minipage}{3.6in}
{#1}
\end{minipage}
}
\end{center}
}







\begin{document}

\begin{frame}[fragile]{Loop Invariants}

\scriptsize

Let us  prove the following program correct for all possible 
values of $n$ and $n \ge 0$:

\begin{verbatim}
assign(i,0) seq
assign(p,0) seq
while not(eq(i,n)) do 
    assign(i,add(i,1)) seq
    assign(p,add(p,i)) 
\end{verbatim}

\vspace{.1in}
With the pre- and postconditions:
\begin{eqnarray*}
{\rm pre}(s) &\equiv & {\rm lookup}(n,s,vn) \wedge vn \ge 0\\
{\rm post}(s) &\equiv& {\rm lookup}(p,s,vp) \wedge vp = \sum_{j=0}^{vn} j\\
\end{eqnarray*}
It becomes clear that the postcondition is an appropriate model for the program when
we try various values for $v$, i.e., $vn =3$.  It is easy to see that in this case
$vp = 6$.
\end{frame}


\begin{frame}[fragile]{Loop Invariants}

\scriptsize

Let us use loop invariants to prove the following program correct:

\vspace{.1in}
{
{\color{red}\{$pre$\}}\\
{\tt assign(i,0) seq}\\
{\tt assign(p,0) seq}\\
{\color{red}\{$inv$\}  }\\                                  
{\tt while not(eq(i,n)) do }\\
{\tt\verb"   "} {\color{red}\{$inv$ $\wedge$  $B$\}  }\\

{\tt\verb"   "}{\tt assign(i,add(i,1)) seq }  \\
{\tt\verb"   "}{\tt assign(p,add(p,i))}  \\
{\tt\verb"   "}{\color{red}\{$inv$\} } \\                            
{\color{red}\{$inv$ $\wedge$ $\neg B$\}}\\
{\color{red}\{$post$\} } 
}

\vspace{.1in}

where $B$ is computed as $\sem{({\rm le}(i,n),s)}{B}$.
\begin{eqnarray*}
{\rm pre}(s) &\equiv & {\rm lookup}(n,s,vn)\wedge vn \ge 0\\
{\rm post}(s) &\equiv& {\rm lookup}(p,s,vp) \wedge vp = \sum_{j=0}^{vn} j\\
{\rm inv}(s) &\equiv& {\rm lookup}(p,s,vp) \wedge \sem{(i,s)}{vi} \wedge vp = \sum_{j=0}^{vi} j
\end{eqnarray*}

{\bf NOTE:} the loop invariants can often be derived from the post condition.
\end{frame}

\begin{frame}[fragile]{Loop Invariants}

\small
Our three proof obligations are:


\vspace{.1in}

$\sem{({\rm init},S)}{Q} \wedge [{\rm pre}(S)\Rightarrow {\rm inv}(Q)]$

\vspace{.05in}

$\sem{({\rm body},S)}{Q} \wedge \sem{({\rm guard},S)}{B} \wedge [({\rm inv}(s) \wedge B)\Rightarrow {\rm inv}(Q)]$


\vspace{.05in}

$\sem{({\rm guard},T)}{B} \wedge [(inv(T) \wedge \neg\,B) \Rightarrow post(T)]$

\end{frame}



\begin{frame}[fragile]{Loop Invariants}

{\scriptsize
\begin{verbatim}
% sum.pl
:-['sem.pl'].

:- >>> 'define the parts of our program'.
init(assign(i,0) seq assign(p,0)).
guard(not(eq(i,n))).
body(assign(i,add(i,1)) seq assign(p,add(p,i))).

:- >>> 'define our sum operation'.
:- dynamic sum/2.

sum(0,0).

sum(X,Y) :-
    T1 is X-1,
    sum(T1,T2),
    Y is X+T2.

:- >>> 'first proof obligation'.
:- >>> 'assume precondition'.
:- asserta(lookup(n,s,vn)).                                                                        
:- >>> 'proof the invariant'.                                                                      
:- init(I),(I,s) -->> Q,lookup(p,Q,VP),lookup(i,Q,VI), sum(VI,VP).
:- retract(lookup(n,s,vn)).                                                                        
\end{verbatim}
}
\end{frame}

\begin{frame}[fragile]{Loop Invariants}

{\scriptsize
\begin{verbatim}
:- >>> 'second  proof obligation'.                                                                 
:- >>> 'assume invariant on s'.                                                                    
:- asserta(lookup(p,s,vp)).
:- asserta(lookup(i,s,vi)).                                                                        
:- asserta(sum(vi,vp)).                                                                            
% implies                                                                                          
:- asserta(sum(vi+1,vp+(vi+1))).                                                                   
:- >>> 'assume guard on s'.                                                                        
:- asserta((not(eq(i,n)),s) -->> true).
:- >>> 'proof the invariant on Q'.
:- body(Bd),(Bd,s) -->> Q,lookup(p,Q,VP),lookup(i,Q,VI), sum(VI,VP).
:- retract(lookup(p,s,vp)).                                                                        
:- retract(lookup(i,s,vi)).
:- retract(sum(vi,vp)).
:- retract(sum(vi+1,vp+(vi+1))).
:- retract((not(eq(i,n)),s) -->> true).
\end{verbatim}
}
\end{frame}

\begin{frame}[fragile]{Loop Invariants}

{\scriptsize
\begin{verbatim}
:- >>> 'third  proof obligation'.
:- >>> 'assume the invariant on s'.
:- asserta(lookup(p,s,vp)).                                                                        
:- asserta(lookup(i,s,vi)).
:- asserta(sum(vi,vp)).
:- >>> 'assume NOT guard on s'.
:- asserta((not(eq(i,n)),s) -->> not(true)).
% implies
:- asserta((eq(i,n),s) -->> true).
% implies
:- asserta(sum(vn,vp)).
:- >>> 'prove postcondition on s'.
:- lookup(p,s,VP),sum(vn,VP).
:- retract(lookup(p,s,vp)).                                                                        
:- retract(lookup(i,s,vi)).
:- retract(sum(vi,vp)).
:- retract((not(eq(i,n)),s) -->> not(true)).
:- retract((eq(i,n),s) -->> true).
:- retract(sum(vn,vp)).
\end{verbatim}
}
\end{frame}


\begin{frame}[fragile]{Factorial}

\small
Let us consider the factorial program:

\vspace{.1in}

\begin{verbatim}
assign(i,1) seq
assign(z,1) seq
while not(eq(i,n)) do
     assign(i,add(i,1)) seq
     assign(z,mult(z,i))
\end{verbatim}

\vspace{.1in}

With pre- and postconditions:
\begin{eqnarray*}
{\rm pre}(S) &\equiv& {\rm lookup}(n,S,vn) \wedge vn \ge 1\\
{\rm post}(Q) &\equiv& {\rm lookup}(z,Q,vn!)\\
{\rm inv}(T) &\equiv& {\rm lookup}(z,T,vz)\wedge{\rm lookup}(i,T,vi)\wedge vz=vi!\\
\end{eqnarray*}

\end{frame}

\begin{frame}[fragile]{Factorial}

\small
Our three proof obligations are:


\vspace{.1in}

$\sem{({\rm init},S)}{Q} \wedge [{\rm pre}(S)\Rightarrow {\rm inv}(Q)]$

\vspace{.05in}

$\sem{({\rm body},S)}{Q} \wedge \sem{({\rm guard},S)}{B} \wedge [({\rm inv}(s) \wedge B)\Rightarrow {\rm inv}(Q)]$


\vspace{.05in}

$\sem{({\rm guard},T)}{B} \wedge [(inv(T) \wedge \neg\,B) \Rightarrow post(T)]$

\end{frame}



\begin{frame}[fragile]{Factorial}

\scriptsize
\begin{verbatim}
% fact.pl
:-['sem.pl'].

:- >>> 'define the parts of our program'.
init(assign(i,1) seq assign(z,1)).
guard(not(eq(i,n))).
body(assign(i,add(i,1)) seq assign(z,mult(z,i))).

:- >>> 'define our fact operation'.
:- dynamic fact/2.

fact(1,1).

fact(X,Y) :-
    T1 is X-1,
    fact(T1,T2),
    Y is X*T2.

:- >>> 'first proof obligation'.
:- >>> 'assume precondition'.
:- asserta(lookup(n,s,vn)).                                                                        
:- >>> 'prove the invariant'.                                                                      
:- init(I),(I,s) -->> Q,lookup(z,Q,VZ),lookup(i,Q,VI), fact(VI,VZ).
:- retract(lookup(n,s,vn)).                                                                        
\end{verbatim}
\end{frame}

\begin{frame}[fragile]{Factorial}

\scriptsize
\begin{verbatim}
:- >>> 'second  proof obligation'.                                                                 
:- >>> 'assume invariant on s'.                                                                    
:- asserta(lookup(z,s,vz)).
:- asserta(lookup(i,s,vi)).                                                                        
:- asserta(fact(vi,vz)).                                                                           
% implies                                                                                          
:- asserta(fact(vi+1,vz*(vi+1))).                                                                  
:- >>> 'assume guard on s'.                                                                        
:- asserta((not(eq(i,n)),s) -->> true).
:- >>> 'prove the invariant on Q'.
:- body(Bd),(Bd,s) -->> Q,lookup(z,Q,VZ),lookup(i,Q,VI), fact(VI,VZ).
:- retract(lookup(z,s,vz)).                                                                        
:- retract(lookup(i,s,vi)).
:- retract(fact(vi,vz)).
:- retract(fact(vi+1,vz*(vi+1))).
:- retract((not(eq(i,n)),s) -->> true).
\end{verbatim}
\end{frame}

\begin{frame}[fragile]{Factorial}

\scriptsize
\begin{verbatim}
:- >>> 'third  proof obligation'.
:- >>> 'assume the invariant on s'.
:- asserta(lookup(z,s,vz)).                                                                        
:- asserta(lookup(i,s,vi)).
:- asserta(fact(vi,vz)).
:- >>> 'assume NOT guard on s'.
:- asserta((not(eq(i,n)),s) -->> not(true)).
% implies
:- asserta((eq(i,n),s) -->> true).
% implies
:- asserta(fact(vn,vz)).
:- >>> 'prove postcondition on s'.
:- lookup(z,s,VZ),fact(vn,VZ).
:- retract(lookup(z,s,vz)).                                                                        
:- retract(lookup(i,s,vi)).
:- retract(fact(vi,vz)).
:- retract((not(eq(i,n)),s) -->> not(true)).
:- retract((eq(i,n),s) -->> true).
:- retract(fact(vn,vz)).
\end{verbatim}
\end{frame}


\end{document}
%%%%%%%%%%%%%%%%%%%%%%%%%%% end of template1.tex %%%%%%%%%%%%%%%%%%%%%%%%%%%%%%%%

