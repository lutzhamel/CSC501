%template1.tex
%The following LaTeX source file represents the simplest kind of slide presentation; no overlays, no included graphics. Substitute your favorite style for ``pascal''. To create the PDF file template1.pdf, (1) be sure to use the prosper class, then (2) execute the command latex template1.tex, and (3) the command dvipdf template1.dvi.

%%%%%%%%%%%%%%%%%%%%%%%%%%%%%%% template1.tex %%%%%%%%%%%%%%%%%%%%%%%%%%%%%%%%%%%
\documentclass{beamer}
% definitions for slides for the semantics course - CSC501
% Lutz Hamel, (c) 2006

\usetheme{Warsaw}
\usepackage{bussproofs}
\usepackage{amsmath}
\usepackage{amssymb}
\usepackage{latexsym}
\usepackage{xypic}
\usepackage{alltt}
\usepackage{rotating}
\usepackage{graphicx}
\usepackage{url}

\newcommand{\ul}[1]{\underline{#1}}
\newcommand{\orbar}{\,|\,}
\newcommand{\co}{\,\colon\;}
\newcommand{\syntaxset}[1]{\ensuremath{\mbox{\bf #1}}}
\newcommand{\nonterm}[1]{\ensuremath{\mbox{#1}}}
\newcommand{\term}[1]{\ensuremath{\mbox{\bf #1}}}
\newcommand{\ifstmt}[3]{\ensuremath{{\bf if}\; {#1}\;{\bf then}\;{#2}\;{\bf else}\;{#3}\;\term{end}}}
\newcommand{\whilestmt}[2]{\ensuremath{{\bf while}\; {#1}\;{\bf do}\;{#2}\; \term{end}}}
\newcommand{\funcstmt}[3]{\ensuremath{{\bf fun}\; {#1}\; {\bf is}\; {#2} \; {\bf return}\; {#3}}}
\newcommand{\localstmt}[1]{\ensuremath{{\bf local}\; {#1}}}
\newcommand{\pairmap}[3]{\ensuremath{( {#1}, {#2} ) \mapsto {#3}}}
\newcommand{\single}[1]{\ensuremath{\langle {#1}\rangle}}
\newcommand{\pair}[2]{\ensuremath{({#1}, {#2} )}}
\newcommand{\triple}[3]{\ensuremath{\langle {#1}, {#2}, {#3} \rangle}}
\newcommand{\cond}[3]{\ensuremath{({#1}?\,{#2} : {#3})}}
\newcommand{\condbar}[3]{\ensuremath{({#1}\rightarrow{#2} \mid {#3})}}
\newcommand{\sem}[2]{\ensuremath{{#1}-\!\!-\!\!\!\!\gg{#2}}}
\newcommand{\liftfunc}[1]{\ensuremath{\lfloor{#1}\rfloor}}
\newcommand{\bcode}{\begin{alltt}\scriptsize}
\newcommand{\ecode}{\end{alltt}}
\newcommand{\comment}[1]{}

\newcommand{\fframe}[1]{
\begin{center}
\fbox{
\begin{minipage}{3.6in}
{#1}
\end{minipage}
}
\end{center}
}

\begin{document}

\begin{frame}{Executable Specifications}
\small
Given the similarity between natural deduction,


\begin{quote}
\begin{minipage}{2in}
\begin{description}

\item[Inference Step]\hspace{.1in}\\
\begin{prooftree}
\AxiomC{premise$_1$}
\AxiomC{$\cdots$}
\AxiomC{premise$_n$}
\RightLabel{\hspace{.1in}(condition)}
\TrinaryInfC{conclusion}
 \end{prooftree}

\item[Assumption]\hspace{1in}\\  
\begin{prooftree}
\AxiomC{}
\UnaryInfC{conclusion}
\end{prooftree}

\end{description}
\end{minipage}
\end{quote}

and Horn clause logic, 

\begin{quote}
\begin{minipage}{2in}
\begin{description}

\item[Inference Step] $P_0 \;\mbox{:-}\; P_1,\ldots,P_n.$

\item[Fact] $P_0 \;\mbox{:-}\; {\bf true}.$

\end{description}
\end{minipage}
\end{quote}

it is natural to assume that we can implement our operational semantics in Prolog.
\end{frame}

\begin{frame}{New Syntax}
\scriptsize
In order to do this we have to rewrite our syntax specification in terms of Prolog terms.

\vspace{.1in}

Arithmetic expressions:
\[
A ::= n \orbar x \orbar {\bf add}(A, A) \orbar {\bf sub}(A, A) \orbar {\bf mult}(A, A)
\]
where $n$ is any valid Prolog integer value and $x$ any valid Prolog object name.

\noindent
Boolean expressions:
\[
B ::= \mbox{\bf true} \orbar \mbox{\bf false} \orbar {\bf eq}(A,A) \orbar
	{\bf le}(A, A) \orbar {\bf not}(B) \orbar {\bf and}(B, B) \orbar {\bf or}(B, B)
\]
Commands:
\[
C ::={\bf skip} \orbar {\bf assign}(x,A) \orbar {\bf seq}(C,C) \orbar
	{\bf if}(B, C,  C) \orbar
	{\bf whiledo}(B, C)
\]
{\bf Note:} the $\bf seq$ operator is made infix for convenience - just remember that it is left-associative - under certain
circumstances you will need to help the Prolog parser out with parentheses around the appropriate terms.
\end{frame}

\begin{frame}{Programs}
\tiny
{\bf Example 1:}
\[
\begin{array}{l}
v := 1; \ifstmt{v \le 0}{v := (-1)*v}{{\bf skip}}\\
\\
{\bf assign}(v, 1)\; {\bf seq}\; {\bf if}({\bf le}(v , 0), {\bf assign}(v, {\bf mult}(-1,v)),{\bf skip})
\end{array}
\]
{\bf Example 2:}
\[
\begin{array}{l}
n := 5 ; y := 1 ; \whilestmt{2 \le n}{(y := n*y ; n := n - 1)}\\
\\
{\bf assign}(n,5) \; {\bf seq}\; {\bf assign}(y,1) \; {\bf seq}\; {\bf whiledo}({\bf le}(2,n),{\bf assign}(y,{\bf mult}(n,y)) \; {\bf seq}\; {\bf assign}(n,{\bf sub}(n,1)))
\end{array}
\]


\end{frame}

\begin{frame}{State}
\small
Prolog does not allow us to pass functions around (it is a first-order language), therefore, we cannot
use the definition of state from our natural deduction operational semantics.

\vspace{.1in}

However, consider the following,
\[
\sigma[m/x][n/y][k/z],
\]
This could be interpreted as a list of variable bindings applied to the state $\sigma$ if
we interpret the $[\ldots ]$ as list constructors and juxtaposition as a list append operation,
\[
{\color{red}\sigma}[m/x,n/y,k/z].
\]
In Prolog we can model this as the term structure
\[
{\bf state}([{\bf bind}(k,z),{\bf bind}(n,y),{\bf bind}(m,x)],{\color{red}{\bf s}})
\]
where $k,n,m\in \mathbb{I}$ and $x,y,z\in\syntaxset{Loc}$ and $s$ represents an arbitrary state.
\end{frame}

\begin{frame}{State}
\small
If the state is the initial state, where
\[
\sigma_0[m/x,n/y,k/z]
\]
then
\[
{\bf state}([{\bf bind}(k,z),{\bf bind}(n,y),{\bf bind}(m,x)],{\bf s}0)
\]
where ${\bf s}0$ is a reserved symbol for the initial state in our representation.

\vspace{.1in}

Given our representation we have an interesting equivalence, given a state $s$, then
\[
s \equiv {\bf state}([\; ],s)
\]
We will make use of this equivalence when encoding our semantic predicates.
\end{frame}

\begin{frame}[fragile]{State}
\small
Since we turned the state representation from a function into a list, we now have to adjust
the variable lookup mechanism.  We do this via the predicate {\em lookup},

{\scriptsize
\begin{verbatim}
% the predicate 'lookup(+Variable,+State,-Value)' looks up
% the variable in the state and returns its bound value.
:- dynamic lookup/3.                % modifiable predicate

lookup(_,s0,0).

lookup(X,state([],S),Val) :-
    lookup(X,S,Val).

lookup(X,state([bind(Val,X)|_],_),Val).

lookup(X,state([_|Rest],S),Val) :-
    lookup(X,state(Rest,S),Val).
\end{verbatim}
}

\end{frame}

\begin{frame}{Arithmetic Expression Summary}

\small
Recall our natural deduction definition for arithmetic expressions:

\vspace{.1in}

%\begin{prooftree}
\AxiomC{}
\RightLabel{\hspace{.1in}for $n\in \syntaxset{D}$ and $\sigma \in \Sigma$}
\UnaryInfC{$( n, \sigma ) \mapsto \mbox{eval}(n)$}
\DisplayProof\\
\vspace{.2in}
%\end{prooftree}
%\begin{prooftree}
\AxiomC{}
\RightLabel{\hspace{.1in}for $x\in \syntaxset{Loc}$  and $\sigma \in \Sigma$}
\UnaryInfC{$( x, \sigma ) \mapsto \sigma(x)$}
\DisplayProof\\
\vspace{.2in}
%\end{prooftree}
%\begin{prooftree}
\AxiomC{$( a_0,\sigma) \mapsto k_0$}
\AxiomC{$( a_1,\sigma) \mapsto k_1$}
\RightLabel{\hspace{.1in}\small where $k= k_0 + k_1$}
\BinaryInfC{$( a_0 + a_1, \sigma) \mapsto k$}
\DisplayProof\\
\vspace{.2in}
% \end{prooftree}
%\begin{prooftree}
\AxiomC{$( a_0,\sigma) \mapsto k_0$} 
 \AxiomC{$( a_1,\sigma) \mapsto k_1$}
 \RightLabel{\hspace{.1in}\small where $k = k_0 - k_1$}
 \BinaryInfC{$( a_0 - a_1, \sigma) \mapsto k$}
\DisplayProof\\
\vspace{.2in}
% \end{prooftree}
%\begin{prooftree}
\AxiomC{$( a_0,\sigma) \mapsto k_0$} 
\AxiomC{$( a_1,\sigma) \mapsto k_1$}
\RightLabel{\hspace{.1in}\small where $k = k_0 \times k_1$}
\BinaryInfC{$( a_0 * a_1, \sigma) \mapsto k$}
\DisplayProof\\
\vspace{.2in}
%\end{prooftree}
{\bf NOTE:} $k$, $k_0$, $k_1  \in \mathbb{I}$, $a_0$, $a_1 \in \syntaxset{Aexp}$, and 
$\sigma \in \Sigma$.
\end{frame}


\begin{frame}[fragile]{Prolog Aexp Semantics}
\tiny
\begin{verbatim}
%%%%%%%%%%%%%%%%%%%%%%%%%%%%%%%%%%%%%%%%%%%%%%%%%%%%%%%%%%%%%%%%%%%%%
% semantics of arithmetic expressions

(C,_) -->> C :-                    % constants
    int(C),!.

(X,State) -->> Val :-              % variables
    atom(X),
    lookup(X,State,Val),!.

(add(A,B),State) -->> Val :-       % addition
    (A,State) -->> ValA,
    (B,State) -->> ValB,
    Val xis ValA + ValB,!.

(sub(A,B),State) -->> Val :-       % subtraction
    (A,State) -->> ValA,
    (B,State) -->> ValB,
    Val xis ValA - ValB,!.

(mult(A,B),State) -->> Val :-     % multiplication
    (A,State) -->> ValA,
    (B,State) -->> ValB,
    Val xis ValA * ValB,!.
\end{verbatim}

\vspace{.1in}

{\bf Note:} The cut predicate is necessary to make sure that these
rules are interpreted as {\em state transitions}, i.e., once a state
transition has occurred in an abstract machine it cannot be undone.
\end{frame}


\begin{frame}[fragile]{The {\tt -->>} Predicate}

\tiny
\begin{verbatim}
%%%%%%%%%%%%%%%%%%%%%%%%%%%%%%%%%%%%%%%%%%%%%%%%%%%%%%%%%%%%%%%%%%%%%
% the predicate '(+Syntax,+State) -->> -SemanticValue' computes
% the semantic value for each syntactic structure

:- op(700,xfx,-->>).
:- dynamic (-->>)/2.                % modifiable predicate
:- multifile (-->>)/2.
\end{verbatim}

\end{frame}

\begin{frame}[fragile]{The `xis' Predicate}

\tiny
\begin{verbatim}
bash-3.2$ prolog -f prolog-semantics/preamble.pl
%  xis.pl compiled 0.00 sec, 6,960 bytes
% /Users/lutz/Documents/csc501/prolog-semantics/preamble.pl compiled 0.00 sec, 9,532 bytes
Welcome to SWI-Prolog (Multi-threaded, 32 bits, Version 5.10.1)
Copyright (c) 1990-2010 University of Amsterdam, VU Amsterdam
SWI-Prolog comes with ABSOLUTELY NO WARRANTY. This is free software,
and you are welcome to redistribute it under certain conditions.
Please visit http://www.swi-prolog.org for details.

For help, use ?- help(Topic). or ?- apropos(Word).

?- X is 1 + 2.
X = 3.

?- X xis 1 + 2.
X = 3.

?- X is y + 2.
ERROR: is/2: Arithmetic: `y/0' is not a function

?- X xis y + 2.
X = y+2.

?- 
\end{verbatim}
\scriptsize
{\bf Note:} both the {\tt -->>} and the xis predicate are defined in the preamble.pl file.  Suggestion: put the preamble.pl file
in a known place such as the super-directory of all your projects, then you can load it as 'prolog -f ../preamble.pl'.
\end{frame}



\begin{frame}[fragile]{Evaluation of Arithmetic Expressions}
\small
Let $ae \equiv (2 * 3) + 5$, prove that the semantic value of this expression in some state $s$ is equal  to $11$ using the Prolog semantics (assume that the semantics is given in the file 'sem.pl').

\begin{verbatim}
?- ['sem.pl'].
%  preamble.pl compiled 0.00 sec, 900 bytes
%  xis.pl compiled 0.00 sec, 6,788 bytes
% sem.pl compiled 0.00 sec, 14,164 bytes
true.

?- (add(mult(2,3),5),s) -->> V, V = 11.
V = 11.

?- 
\end{verbatim}
\end{frame}

\begin{frame}[fragile]{Evaluation of Arithmetic Expressions}
\small
Now, let $ae \equiv x + 1$, where $x \in \syntaxset{Loc}$, prove that the semantic value of this expression in some
state $s$ is equal to $vx +1$ where {\tt lookup(x,s,vx)}.

\begin{verbatim}
?- ['sem.pl'].
%  preamble.pl compiled 0.00 sec, 900 bytes
%  xis.pl compiled 0.00 sec, 6,788 bytes
% sem.pl compiled 0.00 sec, 14,164 bytes
true.

?- asserta(lookup(x,s,vx)).
true.

?- (add(x,1),s) -->> vx+1. 
true.

?- 
\end{verbatim}

{\bf Note:} the predicate 'asserta' is preferable because it inserts the clause at the {\em top} of the rule database.
\end{frame}

\begin{frame}[fragile]{Evaluation of Arithmetic Expressions}
\small
Now, let $ae \equiv x + 1$, where $x \in \syntaxset{Loc}$, prove that the semantic value of this expression in the initial state
$s0$ equal to $1$.

\begin{verbatim}
?- ['sem.pl'].
%  preamble.pl compiled 0.00 sec, 900 bytes
%  xis.pl compiled 0.00 sec, 6,788 bytes
% sem.pl compiled 0.00 sec, 14,164 bytes
true.

?- (add(x,1),s0) -->> 1.
true.

?- 
\end{verbatim}
\end{frame}

\begin{frame}[fragile]{Evaluation of Arithmetic Expressions}
\scriptsize
Consider the following proof:
Let $ae \equiv x + 3*5$, where $x \in \syntaxset{Loc}$, prove that the semantic value of this expression in some
state $s$ is equal to $vx + 15$ where {\tt lookup(x,s,vx)}.


\begin{verbatim}
?- ['sem.pl'].
%  preamble.pl compiled 0.00 sec, 900 bytes
%  xis.pl compiled 0.00 sec, 6,788 bytes
% sem.pl compiled 0.00 sec, 14,164 bytes
true.

?- asserta(lookup(x,s,vx)).
true.

?- (add(x,mult(3,5)),s) -->> vx+15.
true.

?- 
\end{verbatim}
\end{frame}

\begin{frame}[fragile]{Theorem Proving with Prolog}
\scriptsize
Using Prolog as a theorem prover:
\begin{enumerate}

\item
{\em The Prolog Meta-Language} -- we can {\bf consult} programs, {\bf assert} assumptions,
{\bf retract} assumptions, and {\bf query} a program in order to prove a theorem.

\item
{\em Universally Quantified Variables in Queries} -- consider the proof,
{\scriptsize
\begin{alltt}
?- (mult(3,5),s) -->> V, V = 15.
\end{alltt}
}
 can be interpreted in standard FOL
as,
\[
{\color{red}\forall s} \exists V[(\sem{{\bf mult}(3,5),s)} {V} \wedge {\bf =}(V,15)].
\]
$\Rightarrow$ We use symbolic {\bf constants} in queries to express universally quantified variables.

\item
{\em Proof Scores} -- we can write a proof as a Prolog meta-language program.

\item
{\em Soundness} -- under certain circumstances the default resolution rule can be unsound, to avoid
this insert the following code into your proof scores:
\begin{verbatim}
:- set_prolog_flag(occurs_check,true).
\end{verbatim}
If you use the 'preamble.pl' file this is done automatically for you.
\end{enumerate}
\end{frame}

\begin{frame}[fragile]{Theorem Proving with Prolog}

\small
The second point on the previous slide deserves some additional attention.  Consider for the moment that
we would like to prove
\[
\forall s [\sem{({\bf mult}(3,5),s)}{15}].
\]
If we write this blindly as a query using standard Prolog variables, then
{\scriptsize
\begin{alltt}
?- (mult(3,5),{\color{red}S}) -->> 15.
\end{alltt}
}
Now interpreting this query according to Prolog, then
\[
{\color{red}\exists s} [\sem{({\bf mult}(3,5),s)}{15}].
\]
That means, this query does {\em not} prove our intended proof goal.
\end{frame}

\begin{frame}[fragile]{Theorem Proving with Prolog}

\scriptsize

From FOL quantification theory we have the following axiom (Universal Generalization).  Let $p$ be a predicate and let
$y\in U$ be some arbitrarily chosen element of some universe $U$, then
\[
\frac{p(y)}{\therefore \forall x[p(x)]}
\]
with $x\in U$.  In plain English,
\begin{quote}
If I can show that a predicate holds for an arbitrarily chosen element of some universe, then I can infer 
that this predicate holds for all elements of that universe.
\end{quote}
With this we can rewrite our query as
{\scriptsize
\begin{alltt}
?- (mult(3,5),{\color{red}y}) --> 15.
\end{alltt}
}
Here the lowercase $y$ is an element of some universe, in this case the States, and therefore, if Prolog can prove this
goal, we can conclude that
\[
\forall x [\sem{({\bf mult}(3,5),x)}{15}].
\]
with $x\in \mbox{States}$.
\end{frame}

\begin{frame}[fragile]{Theorem Proving with Prolog}

\scriptsize

We have to be careful with universal generalization; the statement  ``{\em some arbitrarily chosen element of some 
universe}'' has a
specific meaning:  
\begin{quote}
The element is not allowed to reveal its structure or internal state.
\end{quote}
Consequently, the predicate we want to generalize {\em is not allowed to investigate the structure or state of the
element}.



\end{frame}

\begin{frame}[fragile]{Theorem Proving with Prolog}

\scriptsize
{\bf Example:} Let $\Sigma$ be the set of all states and let $\sigma'\in\Sigma$ be some arbitrarily chosen element
of that set.  Let $p(\sigma') \equiv \sigma'(x) = 20$. Now, applying universal generalization we have,
\[
\frac{p(\sigma')}{\therefore \forall \sigma[p(\sigma)]}
\]
with $\sigma\in\Sigma$.  This is clearly a fallacious argument, there will be many states in which $\sigma(x) \ne 20$.
This argument failed because the predicate $p$ investigated the internal structure of $\sigma'$.

\vspace{.3in}
{\bf Example:} Let $\Sigma$ be the set of all states and let $\sigma'\in\Sigma$ be some arbitrarily chosen element
of that set.  Let $p(\sigma') \equiv \sigma'[20/x](x) = 20$. Now, applying universal generalization we have,
\[
\frac{p(\sigma')}{\therefore \forall \sigma[p(\sigma)]}
\]
with $\sigma\in\Sigma$.  This argument clearly holds because we did not look at the internal structure of the
element.

\end{frame}

\begin{frame}[fragile]{Theorem Proving with Prolog}

\scriptsize

{\bf Example:} Let $\mathbb{N}$ be the set of all natural numbers and let $7\in\mathbb{N}$ be some arbitrarily chosen
element of that set.  Let $p(7)\equiv 7 \le 100$.  Now, applying universal generalization we have, 
\[
\frac{p(7)}{\therefore \forall k[p(k)]}
\]
with $k\in\mathbb{N}$.  Again, this argument fails because we allowed the predicate to investigate the structure of
the element (the value $7$).

\vspace{.3in}
{\bf Example:} Let $\mathbb{N}$ be the set of all natural numbers and let $n\in\mathbb{N}$ be some arbitrarily chosen
element of that set.  Let $p(n)\equiv n \le n+100$.  Now, applying universal generalization we have, 
\[
\frac{p(n)}{\therefore \forall k[p(k)]}
\]
with $k\in\mathbb{N}$.  This argument succeeds because we did not look at the structure (specific value) of the element.

\end{frame}



\begin{frame}[fragile]{Theorem Proving with Prolog}

\scriptsize
\begin{center}
    \includegraphics[height=30mm]{images/swans}
\end{center}

\vspace{.1in}
The {\em black swan problem} is a classic problem from machine learning theory set forward by mathematician and logician Karl Popper at the beginning of the past century: looking at all the swans in Europe and
North America one would conclude that {\em all swans are white} (set D).  However, if you look at the worldwide population of swans (set X) you will also find black swans (in Australia) and therefore the conclusion based on set D is not valid.
To avoid these invalid inferences, in machine learning theory this often framed as a probabilistic statement:
\begin{quote}
Based on set D it is {\bf most likely} that all swans are white.
\end{quote}

\end{frame}


\begin{frame}[fragile]{Theorem Proving with Prolog}

\scriptsize
Coming back to our universal generalization problem.

\vspace{.1in}
{\bf Example:} Let $X$ be the set of all swans worldwide, let $x'\in X$ be some arbitrarily selected element
of that set.  Let,
\[
p(x') \equiv x' \mbox{ is a white swan.}
\] 
 Now, applying universal generalization we have, 
\[
\frac{p(x')}{\therefore \forall x[p(x)]}
\]
with $x\in X$.  Again, this argument fails because we allowed the predicate to investigate the structure of
the element  $x'$ directly by looking at its color.
\end{frame}


\begin{frame}[fragile]{\large Proof Scores}
\tiny
\begin{verbatim}
% load preamble
:- ['preamble.pl'].

% proof1.pl
% Proof score:
%
:- >>> 'Show that'.
:- >>> '(forall x)(forall s)(forall vx)(exists V)[(add(x,mult(3,5)),s)-->>V ^ =(V,vx+15)]'.
:- >>> ' assuming lookup(x,s,vx)'.

% load semantics
:- ['sem.pl'].

% state our assumption
:- asserta(lookup(x,s,vx)).                                                                 
                                                                                            
% run the proof                                                                             
:- (add(x,mult(3,5)),s)-->>V, V = vx + 15.
\end{verbatim}
\end{frame}

\begin{frame}{Expression Equivalence}
\small
In our Prolog framework semantic equivalence between arithmetic expression
can be formulated as follows:
\[
a_0 \sim a_1 \mbox{ iff } \forall s,\exists V_0,V_1\left[ \sem{(a_0,s)}{V_0}\wedge \sem{(a_1,s)}{V_1} \wedge \mbox{=}(V_0,V_1) \right ],
\]
for $a_0, a_1 \in \syntaxset{Aexp}$.

\end{frame}

\begin{frame}[fragile]{Expression Equivalence}
\small
\begin{verbatim}
% load preamble
:- ['preamble.pl'].

% proof-equiv.pl
:- >>> ' prove that mult(2,3) ~ add(3,3)'.
%
% show that
% (forall s)(exists V0,V1)
%         [(mult(2,3),s)-->>V0 ^ (add(3,3),s)-->>V1 ^ =(V0,V1)]

% load semantics
:- ['sem.pl'].

% proof
:- (mult(2,3),s)-->>V0 , (add(3,3),s)-->>V1 , V0 = V1.
\end{verbatim}
\end{frame}


\begin{frame}[fragile]{Expression Equivalence}
\scriptsize
\begin{verbatim}
?- ['proof-equiv.pl'].
%  preamble.pl compiled 0.00 sec, 924 bytes
>>>  prove that mult(2,3) ~ add(3,3)
%   preamble.pl compiled 0.00 sec, 128 bytes
%   xis.pl compiled 0.00 sec, 6,788 bytes
%  sem.pl compiled 0.00 sec, 12,548 bytes
% proof-equiv.pl compiled 0.01 sec, 14,876 bytes
true.

?- 
\end{verbatim}

\end{frame}


\begin{frame}[fragile]{Expression Equivalence}
\scriptsize
\begin{verbatim}
% proof-comm.pl
:- ['preamble.pl'].

:- >>> 'prove that add(a0,a1) ~ add(a1,a0)'.
%
% show that
% (forall s,a0,a1)(exists V0,V1)
%         [sem(add(a0,a1),s,V0)^sem(add(a1,a0),s,V1)^=(V0,V1)]                              
% assuming                                                                                  
% (a0,s) -->> va0.                                                                          
% (a1,s) -->> va1.                                                                          
                                                                                            
% load semantics                                                                            
:- ['sem.pl'].                                                                              
                                                                                            
% assumptions on semantic values of expressions                                             
:- asserta((a0,s)-->>va0).                                                                  
:- asserta((a1,s)-->>va1).                                                                  
                                                                                            
% assumption on integer addition commutativity                                              
:- asserta(comm(A + B, B + A)).                                                             
                                                                                            
% proof                                                                                     
:- (add(a0,a1),s)-->>V0, (add(a1,a0),s)-->>V1,comm(V0,VC0),VC0=V1.
\end{verbatim}
\end{frame}



\end{document}
%%%%%%%%%%%%%%%%%%%%%%%%%%% end of template1.tex %%%%%%%%%%%%%%%%%%%%%%%%%%%%%%%%

