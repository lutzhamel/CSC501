\documentclass{beamer}
% definitions for slides for the semantics course - CSC501
% Lutz Hamel, (c) 2006

\usetheme{Warsaw}
\usepackage{bussproofs}
\usepackage{amsmath}
\usepackage{amssymb}
\usepackage{latexsym}
\usepackage{xypic}
\usepackage{alltt}
\usepackage{rotating}
\usepackage{graphicx}
\usepackage{url}

\newcommand{\ul}[1]{\underline{#1}}
\newcommand{\orbar}{\,|\,}
\newcommand{\co}{\,\colon\;}
\newcommand{\syntaxset}[1]{\ensuremath{\mbox{\bf #1}}}
\newcommand{\nonterm}[1]{\ensuremath{\mbox{#1}}}
\newcommand{\term}[1]{\ensuremath{\mbox{\bf #1}}}
\newcommand{\ifstmt}[3]{\ensuremath{{\bf if}\; {#1}\;{\bf then}\;{#2}\;{\bf else}\;{#3}\;\term{end}}}
\newcommand{\whilestmt}[2]{\ensuremath{{\bf while}\; {#1}\;{\bf do}\;{#2}\; \term{end}}}
\newcommand{\funcstmt}[3]{\ensuremath{{\bf fun}\; {#1}\; {\bf is}\; {#2} \; {\bf return}\; {#3}}}
\newcommand{\localstmt}[1]{\ensuremath{{\bf local}\; {#1}}}
\newcommand{\pairmap}[3]{\ensuremath{( {#1}, {#2} ) \mapsto {#3}}}
\newcommand{\single}[1]{\ensuremath{\langle {#1}\rangle}}
\newcommand{\pair}[2]{\ensuremath{({#1}, {#2} )}}
\newcommand{\triple}[3]{\ensuremath{\langle {#1}, {#2}, {#3} \rangle}}
\newcommand{\cond}[3]{\ensuremath{({#1}?\,{#2} : {#3})}}
\newcommand{\condbar}[3]{\ensuremath{({#1}\rightarrow{#2} \mid {#3})}}
\newcommand{\sem}[2]{\ensuremath{{#1}-\!\!-\!\!\!\!\gg{#2}}}
\newcommand{\liftfunc}[1]{\ensuremath{\lfloor{#1}\rfloor}}
\newcommand{\bcode}{\begin{alltt}\scriptsize}
\newcommand{\ecode}{\end{alltt}}
\newcommand{\comment}[1]{}

\newcommand{\fframe}[1]{
\begin{center}
\fbox{
\begin{minipage}{3.6in}
{#1}
\end{minipage}
}
\end{center}
}

\begin{document}

\begin{frame}[fragile]{Translational Semantics}

\small 
What if we define the semantics of some language in terms of another language by translating our source
language syntax into the syntax of some target language?

\vspace{.1in}

In this case we obtain the following diagram,

\vspace{.1in}
\[
\xymatrix{
{\rm Model}_{\rm Source} & \approx &{\rm Model}_{\rm Target}\\
{\rm Syntax}_{\rm Source}\ar[u]^{\rm interpretation}\ar[rr]_{\rm translation}& &{\rm Syntax}_{\rm Target}\ar[u]_{\rm interpretation}
}
\]
\end{frame}

\begin{frame}[fragile]{Translational Semantics}

\small 

Two Observations:
\begin{enumerate}
\item This means we have now two ways to interpret a sentence in the source language: 
\begin{enumerate}
\item we can interpret the
sentence in the source language model 
\item we can first translate the sentence and then interpret it in the 
model of the target language
\end{enumerate}
\item We say that the translation is {\em correct} if we can establish a correspondence between the source and
target models for each source language syntax structure.
\end{enumerate}
\end{frame}

\begin{frame}[fragile]{Translational Semantics}
\tiny
\begin{minipage}[t]{1.9in}
Source Language:
\begin{verbatim}
  A ::= n
     |  x
     |  add(A,A)
     |  sub(A,A)
     |  mult(A,A)

  B ::= true
     |  false
     |  eq(A,A)
     |  le(A,A)
     |  not(B)
     |  and(B,B)
     |  or(B,B)

  C ::= skip
     |  assign(x,A)
     |  seq(C,C)
     |  if(B,C,C)
     |  whiledo(B,C)
\end{verbatim}
\end{minipage}
\begin{minipage}[t]{1.9in}
Target Language:
\begin{verbatim}
 prog ::= [ cmseq ]   |  [ ]

 cmseq ::=  cm  |  cm , cmseq

 cm ::= push(V)
     |  add
     |  sub
     |  mult
     |  and
     |  or
     |  neg
     |  eq
     |  le
     |  pop(x)
     |  label(L)
     |  jmp(L)
     |  jmpt(L)
     |  jmpf(L)
     |  stop

 V ::=  x  |  n  | true | false

 L ::=  <alpha string>
\end{verbatim}
\end{minipage}
\end{frame}

\begin{frame}[fragile]{Translational Semantics}

\small
We can define a semantics for our source language by defining equivalent sentences
in the target language for each syntactic construct in the source language.
Consider the arithmetic expressions:
\tiny
\begin{alltt}
translate(N,[push(N)]) :- int(N),!.

translate(X,[push(X)]) :- atom(X),!.

translate(add(A,B),Target) :-
    translate(A,TargetA),
    translate(B,TargetB),
    TargetOP = [add],
    flatten([TargetA,TargetB,TargetOP],Target),!.
 
translate(sub(A,B),Target) :-
    translate(A,TargetA),
    translate(B,TargetB),
    TargetOP = [sub],
    flatten([TargetA,TargetB,TargetOP],Target),!.
     
translate(mult(A,B),Target) :-
    translate(A,TargetA),
    translate(B,TargetB),
    TargetOP = [mult],
    flatten([TargetA,TargetB,TargetOP],Target),!.
\end{alltt}
\end{frame}

\begin{frame}[fragile]{Translational Semantics}

\scriptsize
\begin{alltt}
?- translate(add(3,2),C),ppc(C).
    push(3)
    push(2)
    add
C = [push(3), push(2), add].
\end{alltt}

\begin{alltt}
?- translate(add(3,mult(2,x)),C),ppc(C).
    push(3)
    push(2)
    push(x)
    mult
    add
C = [push(3), push(2), push(x), mult, add].
\end{alltt}

\end{frame}

\begin{frame}[fragile]{Translational Semantics}

\small
The boolean expressions can be defined in a similar manner:
\tiny
\begin{alltt}
translate(true,Target) :-
    Target = [push(true)],!.

translate(false,Target) :-
    Target = [push(false)],!.

translate(and(A,B),Target) :-
    translate(A,T1),
    translate(B,T2),
    T3 = [and],
    flatten([T1,T2,T3],Target),!.

translate(or(A,B),Target) :-
    translate(A,T1),
    translate(B,T2),
    T3 = [or],
    flatten([T1,T2,T3],Target),!.

translate(not(A),Target) :-
    translate(A,T1),
    T2 = [neg],
    flatten([T1,T2],Target),!.
\end{alltt}

\small

\end{frame}

\begin{frame}[fragile]{Translational Semantics}

\small
Translational semantics of statements:
\tiny
\begin{alltt}
translate(skip,[]) :- !.

translate(seq(C1,C2),Target) :-
    translate(C1,T1),
    translate(C2,T2),
    flatten([T1,T2],Target),!.

translate(assign(X,A),Target) :-
    translate(A,T1),
    T2 = [pop(X)],
    flatten([T1,T2],Target),!.

translate(if(B,C0,C1),Target) :-
    translate(B,T1),
    T2 = [jmpf(iflabel1)],
    translate(C0,T3),
    T4 = [jmp(iflabel2),label(iflabel1)],
    translate(C1,T5),
    T6 = [label(iflabel2)],
    flatten([T1,T2,T3,T4,T5,T6],Target),!.

translate(whiledo(B,C),Target) :-
    T1 = [label(whilelabel1)],
    translate(B,T2),
    T3 = [jmpf(whilelabel2)],
    translate(C,T4),
    T5 = [jmp(whilelabel1),label(whilelabel2)],
    flatten([T1,T2,T3,T4,T5],Target),!.
\end{alltt}

\end{frame}

\begin{frame}[fragile]{Translational Semantics}

\scriptsize
\begin{alltt}
?- translate(assign(x,1) seq assign(y,mult(2,x)),C),ppc(C).
    push(1)
    pop(x)
    push(2)
    push(x)
    mult
    pop(y)
C = [push(1), pop(x), push(2), push(x), mult, pop(y)].
\end{alltt}

{\bf Note:} the predicate {\tt ppc} simply prints out the list of assembly code instructions in a nice
formatted way.
\end{frame}

\begin{frame}[fragile]{Translational Semantics}

\scriptsize

\begin{alltt}
?- translate(if(le(2,3),assign(i,3),assign(i,4)),C),ppc(C).
    push(2)
    push(3)
    le
    jmpf(iflabel1)
    push(3)
    pop(i)
    jmp(iflabel2)
label(iflabel1)
    push(4)
    pop(i)
label(iflabel2)
C = [push(2), push(3), le, jmpf(iflabel1), ...]
\end{alltt}


\end{frame}

\begin{frame}[fragile]{Translational Semantics}

\scriptsize

\begin{alltt}
?- translate(whiledo(le(x,3),assign(x,add(x,1))),C),ppc(C).
label(whilelabel1)
    push(x)
    push(3)
    le
    jmpf(whilelabel2)
    push(x)
    push(1)
    add
    pop(x)
    jmp(whilelabel1)
label(whilelabel2)
C = [label(whilelabel1), push(x), push(3), le, jmpf(whilelabel2), ...]
\end{alltt}

\end{frame}

\begin{frame}[fragile]{Translational Semantics}

\scriptsize
The reason why we have compilers:

\tiny
\begin{minipage}[t]{1.9in}
\begin{verbatim}
assign(i,1) seq
assign(z,1) seq
whiledo(not(eq(i,n)),
   assign(i,add(i,1)) seq
   assign(z,mult(z,i)))
\end{verbatim}
\end{minipage}
\begin{minipage}[t]{1.9in}
\begin{verbatim}
    push(1)
    pop(i)
    push(1)
    pop(z)
label(whilelabel1)
    push(i)
    push(n)
    eq
    neg
    jmpf(whilelabel2)
    push(i)
    push(1)
    add
    pop(i)
    push(z)
    push(i)
    mult
    pop(z)
    jmp(whilelabel1)
label(whilelabel2)
\end{verbatim}
\end{minipage}
\end{frame}
\end{document}

