\documentclass{beamer}
% definitions for slides for the semantics course - CSC501
% Lutz Hamel, (c) 2006

\usetheme{Warsaw}
\usepackage{bussproofs}
\usepackage{amsmath}
\usepackage{amssymb}
\usepackage{latexsym}
\usepackage{xypic}
\usepackage{alltt}
\usepackage{rotating}
\usepackage{graphicx}
\usepackage{url}

\newcommand{\ul}[1]{\underline{#1}}
\newcommand{\orbar}{\,|\,}
\newcommand{\co}{\,\colon\;}
\newcommand{\syntaxset}[1]{\ensuremath{\mbox{\bf #1}}}
\newcommand{\nonterm}[1]{\ensuremath{\mbox{#1}}}
\newcommand{\term}[1]{\ensuremath{\mbox{\bf #1}}}
\newcommand{\ifstmt}[3]{\ensuremath{{\bf if}\; {#1}\;{\bf then}\;{#2}\;{\bf else}\;{#3}\;\term{end}}}
\newcommand{\whilestmt}[2]{\ensuremath{{\bf while}\; {#1}\;{\bf do}\;{#2}\; \term{end}}}
\newcommand{\funcstmt}[3]{\ensuremath{{\bf fun}\; {#1}\; {\bf is}\; {#2} \; {\bf return}\; {#3}}}
\newcommand{\localstmt}[1]{\ensuremath{{\bf local}\; {#1}}}
\newcommand{\pairmap}[3]{\ensuremath{( {#1}, {#2} ) \mapsto {#3}}}
\newcommand{\single}[1]{\ensuremath{\langle {#1}\rangle}}
\newcommand{\pair}[2]{\ensuremath{({#1}, {#2} )}}
\newcommand{\triple}[3]{\ensuremath{\langle {#1}, {#2}, {#3} \rangle}}
\newcommand{\cond}[3]{\ensuremath{({#1}?\,{#2} : {#3})}}
\newcommand{\condbar}[3]{\ensuremath{({#1}\rightarrow{#2} \mid {#3})}}
\newcommand{\sem}[2]{\ensuremath{{#1}-\!\!-\!\!\!\!\gg{#2}}}
\newcommand{\liftfunc}[1]{\ensuremath{\lfloor{#1}\rfloor}}
\newcommand{\bcode}{\begin{alltt}\scriptsize}
\newcommand{\ecode}{\end{alltt}}
\newcommand{\comment}[1]{}

\newcommand{\fframe}[1]{
\begin{center}
\fbox{
\begin{minipage}{3.6in}
{#1}
\end{minipage}
}
\end{center}
}

\begin{document}

\begin{frame}[fragile]{Models}

\begin{quote}
One person's syntax is another person's semantics.
\end{quote}

One particular view of semantics is that is a regression of languages.  Consider our translational semantics:
\[
\mbox{source language $\rightarrow$ stack machine language $\rightarrow$ Prolog}
\]
If we only consider the first half of this diagram then we are assigning semantics to the source language in terms
of syntactic snippets of the stack machine language.  

\vspace{.1in}

If we only consider the second part of the diagram then
we are interpreting stack machine language syntax in terms of Prolog constructs (the thing we have done all along
in this course).
\end{frame}


\begin{frame}[fragile]{Models}
A fundamental question in semantics is: When is it reasonable to end this regression? 

\vspace{.1in}

The answer: when the final language
in this regression is a formal language that can be understood in mathematical terms.

\vspace{.1in}

There is a reason why we chose Prolog as our semantic modeling language, because Prolog itself has
a semantics -- Model Theory,

\[
\mbox{Prolog $\rightarrow$ Model Theory}
\]

That is, every Prolog program can be interpreted in a ``first order model'' sometimes also called the {\em Herbrand Universe}.\footnote{{en.wikipedia.org/wiki/Term\_algebra}}
\end{frame}

\begin{frame}[fragile]{Models}

Now, if look at what we have done for the major part of the course and then attach the semantics of Prolog we get this diagram,

\[
\mbox{source language $\rightarrow$ Prolog $\rightarrow$ Model Theory}
\]

Considering that interpretation is transitive this means by using Prolog as our defining language for our source languages
we obtain formal (mathematical) models for our own languages!

\end{frame}


\begin{frame}[fragile]{Elements of Model Theory}

\begin{itemize}
\item 
Model theory is the study of semantics for logic and provides the formal justification 
for using Prolog as a defining language and as a theorem prover.

\item
Similar to our models we constructed in this class, the models in Model  Theory
provide an interpretation for the syntactic units appearing in a logic program.
\end{itemize}
\end{frame}


\begin{frame}[fragile]{Elements of Model Theory}

\small
\begin{itemize}
\item 
Let's start with what we know, we know how to write logic programs and deduce 
knowledge from them.

\item
Let's consider the following logic program $P$:
{\scriptsize
\begin{alltt}
odd(s(0)).
odd(s(s(X))) :- odd(X).
\end{alltt}
}
Here the functor {\tt s} represents the successor function that takes an integer value and produces the integer
value that follows that integer.  That is {\tt s(0)} represents 1 and {\tt s(s(0))} represents 2, etc.

\item
Given this program we now can pose queries to extract knowledge:
{\scriptsize
\begin{alltt}
?- odd(s(s(s(s(s(0)))))).
true 

?- odd(s(s(0))).
false.

?- 
\end{alltt}
}
\end{itemize}
\end{frame}


\begin{frame}[fragile]{Elements of Model Theory}

\small
\begin{itemize}
\item 
Given our program $P$ and our query $q$:
{\scriptsize
\begin{alltt}
?- odd(s(s(s(s(s(0)))))).
true 
\end{alltt}
}
We say that $q$ {\em derives} from $P$ and we write
\[
P \vdash q
\]

\item
Here the $\vdash$ operator represents inferencing in logic and in this particular case it represents
inferencing via {\em resolution}.

\end{itemize}
\end{frame}

\begin{frame}[fragile]{Elements of Model Theory}

\small
\begin{itemize}
\item 
Now, when we wrote our program $P$:
{\scriptsize
\begin{alltt}
odd(s(0)).
odd(s(s(X))) :- odd(X).
\end{alltt}
}
we had a particular model in mind.  In particular, we probably thought about the natural numbers $\mathbb{N}$.

\item
Taking the natural numbers together with the $nodd$ operation gives us our model $M$:
\[
(\mathbb{N}, nodd) %%% TODO: the model should be expanded to include the booleans
\]
where $nodd : \mathbb{N} \rightarrow \{true,false\}$ is defined as the function\footnote{Notice that with this definition 0 is an even number.}
\[
nodd(x) = \left\{
\begin{array}{ll}
true & \mbox{ if $x\!\!\!\!\mod 2 = 0,$}\\
false & \mbox{ if $x\!\!\!\!\mod 2 = 1,$}
\end{array}
\right .
\]
for all $x\in\mathbb{N}$.
\end{itemize}
\end{frame}

\begin{frame}[fragile]{Elements of Model Theory}

\small
\begin{itemize}
\item
It is now easy to show that each sentence $p \in P$ is {\em satisfied} by this model, {\it i.e.}, 
each sentence $p\in P$ is true in model $M$, where $P$ is our {\tt odd} program.

% TODO: should talk more about the actual interpretation.
\item
We can formally show this for the first sentence,
\begin{alltt}
odd(s(0)).
\end{alltt}
if we let ${\tt s(0)}\mapsto 1$ and ${\tt odd}\mapsto nodd$ then $nodd(1)$ is true in
$M$.  

\item Similarly for the sentence {\tt odd(s(s(X))) :- odd(X)}, this can be shown by induction over the odd natural numbers.
\end{itemize}
\end{frame}

\begin{frame}[fragile]{Elements of Model Theory}

\begin{itemize}
\item
 If our model satisfies some sentence $p$ then we write:
\[
M \models p
\]
and we say that {\em sentence $p$ is true in model $M$}.

\item
If our model $M$ satisfies every sentence $p$ in $P$ then we write
\[
M \models P
\]
\end{itemize}
\end{frame}


\begin{frame}[fragile]{Elements of Model Theory}

\small
Assume that we have $M \models P$, 
\begin{itemize}
\item 
We say that our logic is {\em sound} if anything that we can derive from our program $P$
is also true in our model $M$. Formally,
\[
P \vdash p \Rightarrow M \models p
\]

\item
We say that our logic is {\em complete} if anything that is true in the model can be
derived from the program $P$. Formally,
\[
M \models p \Rightarrow P \vdash p
\]

\end{itemize}

It has been shown that resolution is sound and complete,\footnote{
{\it Foundations of Logic Programming},Lloyd, J.W.,Springer-Verlag, 1987.}
that means, anything we can query from the program will be true in
the model, and anything that is true in the model can be deduced from the
program using queries.\footnote{Well, with some help - Prolog's implementation 
of resolution turns out not to be complete.}
\end{frame}


\begin{frame}[fragile]{Elements of Model Theory}

\small
\begin{itemize}
\item 
In terms of programming language semantics, let $P$ be a description of
a programming language model, let $M$ be the intended model, then because
of soundness and completeness, any characteristic $c$ about our 
programming language that can be deduced from $P$ will also be true 
in the intended model,
\[
P \vdash c \Rightarrow M \models c
\]
and any characteristic $c$ that is true in $M$ can be proven,
\[
M \models c \Rightarrow P \vdash c
\]

\item
That means, we are justified to use Prolog as a theorem prover to prove
characteristics about our programming language models.
\end{itemize}

\end{frame}

\begin{frame}[fragile]{Elements of Model Theory}

\begin{center}
\makebox{\LARGE\color{red}THE END}
\end{center}
\end{frame}

\end{document}

