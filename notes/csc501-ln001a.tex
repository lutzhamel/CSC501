\documentclass{beamer}
% definitions for slides for the semantics course - CSC501
% Lutz Hamel, (c) 2006

\usetheme{Warsaw}
\usepackage{bussproofs}
\usepackage{amsmath}
\usepackage{amssymb}
\usepackage{latexsym}
\usepackage{xypic}
\usepackage{alltt}
\usepackage{rotating}
\usepackage{graphicx}
\usepackage{url}

\newcommand{\ul}[1]{\underline{#1}}
\newcommand{\orbar}{\,|\,}
\newcommand{\co}{\,\colon\;}
\newcommand{\syntaxset}[1]{\ensuremath{\mbox{\bf #1}}}
\newcommand{\nonterm}[1]{\ensuremath{\mbox{#1}}}
\newcommand{\term}[1]{\ensuremath{\mbox{\bf #1}}}
\newcommand{\ifstmt}[3]{\ensuremath{{\bf if}\; {#1}\;{\bf then}\;{#2}\;{\bf else}\;{#3}\;\term{end}}}
\newcommand{\whilestmt}[2]{\ensuremath{{\bf while}\; {#1}\;{\bf do}\;{#2}\; \term{end}}}
\newcommand{\funcstmt}[3]{\ensuremath{{\bf fun}\; {#1}\; {\bf is}\; {#2} \; {\bf return}\; {#3}}}
\newcommand{\localstmt}[1]{\ensuremath{{\bf local}\; {#1}}}
\newcommand{\pairmap}[3]{\ensuremath{( {#1}, {#2} ) \mapsto {#3}}}
\newcommand{\single}[1]{\ensuremath{\langle {#1}\rangle}}
\newcommand{\pair}[2]{\ensuremath{({#1}, {#2} )}}
\newcommand{\triple}[3]{\ensuremath{\langle {#1}, {#2}, {#3} \rangle}}
\newcommand{\cond}[3]{\ensuremath{({#1}?\,{#2} : {#3})}}
\newcommand{\condbar}[3]{\ensuremath{({#1}\rightarrow{#2} \mid {#3})}}
\newcommand{\sem}[2]{\ensuremath{{#1}-\!\!-\!\!\!\!\gg{#2}}}
\newcommand{\liftfunc}[1]{\ensuremath{\lfloor{#1}\rfloor}}
\newcommand{\bcode}{\begin{alltt}\scriptsize}
\newcommand{\ecode}{\end{alltt}}
\newcommand{\comment}[1]{}

\newcommand{\fframe}[1]{
\begin{center}
\fbox{
\begin{minipage}{3.6in}
{#1}
\end{minipage}
}
\end{center}
}

\begin{document}

\begin{frame}{\large Mathematical  Preliminaries: Logical Notation}

We\footnote{\tiny The material presented here is based on ``Naive Set Theory'' by P. Halmos and ``The Formal Semantics of Programming Languages'' by G. Winskel.}
will use first order logic as the basis for our reasoning.  Without going into the formal details
of first order logic terminology and sentence construction we have the following statements:
\begin{itemize}
\item $A \wedge B$ denotes the conjunction $A$ and $B$,
\item $A \vee B$ denotes the disjunction $A$ or $B$,
\item $\neg A$ denotes the negation not $A$,
\item $A \Rightarrow B$ denotes the implication, if $A$ then $B$,
\item $A \Leftrightarrow B$ denotes the logical equivalence, $A$ if and only if $B$ (often
written as $A$ iff $B$),
\end{itemize}
where $A$ and $B$ are statements or assertions.
\vspace{.1in}
 \end{frame}

\begin{frame}{\large Mathematical  Preliminaries: Logical Notation}
Observe the precedences of the logical operators, ordered from high to low:
\begin{itemize}
\item $\neg$
\item $\wedge,\vee$
\item $\Rightarrow, \Leftrightarrow$
\end{itemize}
\end{frame}


\begin{frame}{A Word or Two about Implication}
The truth table for the implication operator `$\Rightarrow$' can be given as
{\scriptsize
\[
\begin{array}{lcc|c}
&A & B & A \Rightarrow B\\ \hline
(1)& 1 & 0 & 0\\
(2) & 1 & 1 & 1\\
(3) & 0 & 0 & 1\\
(4) & 0 & 1 & 1
\end{array}
\]
}
Entries $(1)$ and $(2)$ are intuitive: When the antecedent $A$ is true but the consequent $B$ is false then
the implication itself is false.  If both the antecedent and the consequent are true then the implication is true.

\vspace{.1in}

However, entries $(3)$ and $(4)$ are somewhat counter intuitive.  They state that if the antecedent $A$ is false
then the implication is true regardless of the value of the consequent.  In other words, we can conclude ``anything''
from an antecedent that is false.  In mathematical jargon we say that $(3)$ and $(4)$  {\bf hold trivially}.
\end{frame}

\begin{frame}{A Word or Two about Implication -- An Example}

{\scriptsize
\[
\begin{array}{l}
\mbox{If Bob is a bachelor, then he is single.}\\
\mbox{Bob is a bachelor.}\\ \hline
\therefore\mbox{Bob is single.}
\end{array}
\]
}
Now consider an antecedent that is not true,
{\scriptsize
\[
\begin{array}{l}
\mbox{If Bob is a bachelor, then he is single.}\\
\mbox{Bob is not a bachelor.}\\ \hline
\therefore\mbox{Bob is not single (by rule $(3)$).}
\end{array}
\]
}
Since the antecedent is not true rule $(3)$ allows us to conclude the
opposite of what the implication dictates.  However, the following is also
valid reasoning,
{\scriptsize
\[
\begin{array}{l}
\mbox{If Bob is a bachelor, then he is single.}\\
\mbox{Bob is not a bachelor.}\\ \hline
\therefore\mbox{Bob is  single (by rule $(4)$).}
\end{array}
\]
}
Not being a bachelor does not necessarily imply that Bob is not single.  For example,
Bob could be a widower or a divorcee.
\end{frame}

\begin{frame}{A Word or Two about Implication}

Given the truth table for implication,
\[
\begin{array}{lcc|c}
&A & B & A \Rightarrow B\\ \hline
(1)& 1 & 0 & 0\\
(2) & 1 & 1 & 1\\
(3) & 0 & 0 & 1\\
(4) & 0 & 1 & 1
\end{array}
\]
this means that in order to show that an implication holds we only have to show that rule $(2)$
holds.  Rule $(1)$ states that the implication is false and rules $(3)$ and $(4)$ are trivially true and
therefore not interesting.
\end{frame}

\begin{frame}{Closely Related: Equivalence}

We write $A \Leftrightarrow B$ if $A$ and $B$ are equivalent.

Given the truth table for the equivalence operator is given as,
\[
\begin{array}{lcc|c}
&A & B & A \Leftrightarrow B\\ \hline
(1)& 1 & 0 & 0\\
(2) & 1 & 1 & 1\\
(3) & 0 & 0 & 1\\
(4) & 0 & 1 & 0
\end{array}
\]
That is, the operator only produces a true value if $A$ and $B$ have the same truth assignment.

Another, and very useful, way to look at the equivalence operator is as follows:
\[
A \Leftrightarrow B \equiv A \Rightarrow B \wedge B \Rightarrow A
\]
{\bf Exercise:} Construct  the above truth table using this definition of the equivalence operator.
\end{frame}

\begin{frame}{\large Mathematical  Preliminaries: Logical Notation}
We also allow predicates (properties) as part of our notation,
\[
P(x)
\]
where the predicate $P$ is true if it holds for $x$ otherwise it is false.  We view our
standard relational operators as binary predicates.  For example, the predicate
$P(x)$ that expresses the fact that $x$ is less or equal to 3 is written as,
\[
P(x) \equiv x \le 3.
\]
{\bf Note:} Predicates can have arities larger than 1, e.g. $P(x,y)$ with $P(x,y)\equiv x \le y$.
\end{frame}

\begin{frame}{\large Mathematical  Preliminaries: Logical Notation}
We also allow for the quantifiers $\exists$ (there exists) and $\forall$ (for all) in our logical
statements,
\begin{itemize}
\item $\exists x.\,P(x)$ -- ``there exists an $x$ such that $P(x)$''
\item $\forall x.\,P(x)$ -- ''for all $x$ such that $P(x)$''
\end{itemize}
Some examples,
\begin{itemize}
\item $\forall x,\exists y.\, y = x^2$
\item $\forall x, \forall y.\, \mbox{female}(x) \wedge \mbox{child}(x,y) \Rightarrow \mbox{mother}(x,y)$
\end{itemize}

\end{frame}


\begin{frame}{\large Mathematical  Preliminaries: Sets}
Sets\footnote{\tiny Read Sections 2.1 and 2.2 in the book by David Schmidt.}
 are unordered collections of objects and are usually denoted by capital letters.  For example,
let $a,b,c$ denote some objects then the set $A$ of these objects is written as,
\[
A = \{a,b,c\}.
\]
There are a number of standard sets which come in handy,
\begin{itemize}
\item $\emptyset$ denotes the empty set, i. e. $\emptyset = \{ \}$,
\item $\mathbb N$ denotes the set of all natural numbers including 0, e. g. ${\mathbb N} = \{0,1,2,3,\cdots\}$,
\item $\mathbb I$ denotes the set of all integers, ${\mathbb I}= \{\cdots,-2,-1,0,1,2,\cdots\}$,
\item $\mathbb R$ denotes the set of all reals,
\item $\mathbb B$ denotes the set of boolean values, ${\mathbb B} = \{\mathit{true}, \mathit{false} \}$.
\end{itemize}
\end{frame}

\begin{frame}{\large Mathematical  Preliminaries: Sets}
The most fundamental property in set theory is the notion of {\em belonging},
\[
a \in A \mbox{ iff  $a$ is an element of the set $A$}.
\]
The notion of belonging allows us
to define {\em subsets},
\[
Z \subseteq A \mbox{ iff } \forall e\in Z.\, e\in A.
\]
We define set {\em equivalence} as,
\[
A = B \mbox{ iff } A \subseteq B \wedge B \subseteq A
\]
\end{frame}


\begin{frame}{\large Mathematical  Preliminaries: Sets}
We can construct new sets from given sets using {\em union},
\[
A \cup B = \{ e \mid e\in A \vee e \in B\},
\]
 and {\em intersection},
\[
A \cap B = \{ e \mid  e\in A \wedge e \in B\}.
\]
There is another important set construction called the {\em cross product},
\[
A\times B = \{ (a,b) \mid  a\in A \wedge b\in B\},
\]
$A\times B$ is the set of all ordered pairs where the first component of the pair is drawn from
the set $A$ and the second component of the pair is drawn from $B$. (

{\bf Exercise:} Let $A=\{a,b\}$ and $B=\{c,d\}$, construct the set $A\times B$.

\end{frame}

\begin{frame}{\large Mathematical  Preliminaries: Sets}
A construction using subsets is the {\em powerset} of some set $X$,
\[
{\mathcal P}(X),
\]
The {\em powerset} of set $X$ is set of all subsets of $X$.  For example,
let $X = \{a,b\}$, then
\[
{\mathcal P}(X) = \{ \emptyset, \{a\}, \{b\}, \{a,b\} \}.
\]
{\bf Note:} $\emptyset \subset X$

\vspace{.1in}

{\bf Exercise:} What would ${\mathcal P}(X\times X)$ look like?

\end{frame}

\begin{frame}{\large Mathematical  Preliminaries: Sets}
\scriptsize
The fact that $\emptyset \subset X$ for any set $X$ is interesting in its own right.  Let's see if we can prove it.

\vspace{.1in}

{\bf Proof:} Proof by contradiction. Assume $X$ is any set. Assume that $\emptyset$ is not a subset of $X$.
Then the definition of subsets,
\[
A \subseteq B \Leftrightarrow \forall e \in A. e \in B,
\]
implies that there exist at least one element in $\emptyset$ that is not also in $X$.  But that is not possible
because $\emptyset$ has no elements -- a contradiction.  Therefore, our assumption the
that $\emptyset$ is not a subset of $X$ must be wrong and we can conclude that $\emptyset \subset X$.

\end{frame}


\begin{frame}{\large Mathematical  Preliminaries: Relations}
A (binary) relation is a set of ordered pairs.  If $R$ is a relation that relates the elements of
set $A$ to the elements $B$, then
\[
R \subseteq A\times B.
\]
This means if $a\in A$ is related to $b\in B$ via the relation $R$, then $(a,b)\in R$.  We often
write
\[
a\, R\, b.
\]
Consider the relational operator $\le$ applied to the set ${\mathbb N}\times {\mathbb N}$.
This induces a relation, call it $\le \subseteq {\mathbb N}\times {\mathbb N}$, with
$(a,b) \in \le$ (or $a \le b$ in our relational notation) if $a\in{\mathbb N}$ is less or equal to $b\in{\mathbb N}$.
\end{frame}

\begin{frame}{\large Mathematical  Preliminaries: Relations}
The first and second components of each pair in some relation $R$ are drawn from different sets called
the {\em projections} of $R$ onto the first and second {\em coordinate}, respectively.  We introduce
the operators {\em domain} and {\em range} to accomplish these projections.  Let $R\subseteq A\times B$,
then,
\[
\mbox{dom}(R) = A,
\]
and
\[
\mbox{ran}(R) = B.
\]
In this case we talk about a relation {\em from} $A$ {\em to} $B$.  The range is often
called the co-domain. If $R\subseteq X\times X$, then
\[
\mbox{dom}(R) = \mbox{ran}(R) = X.
\]
Here we talk about a relation {\em in} $X$.
\end{frame}

\begin{frame}{\large Mathematical  Preliminaries: Relations}
Let $R \subseteq X\times X$ such that $(a,b)\in R$ iff $a=b$.  That is, $R$ is the {\em equality relation}
in  $X$. (What do the elements of the equality relation look like for $\mathbb{N}\times\mathbb{N}$?)

\vspace{.1in}

A relation $R\subseteq X\times X$ is an {\em equivalence relation} if the following conditions hold,
\begin{itemize}
\item $R$ is {\em reflexive}\footnote{Recall that $x\, R\, x \equiv (x,x)\in R$} -- $x\, R\, x$,
\item $R$ is {\em symmetric} -- $x\, R\, y \Rightarrow y\, R\, x$,
\item $R$ is {\em transitive} -- $x\, R\, y \wedge y\, R\, z \Rightarrow x\, R\, z$,
\end{itemize}
where $x,y,z \in X$.

\vspace{.1in}

The {\em smallest} equivalence relation in some set $X$ is the equality relation defined above.
The {\em largest} equivalence relation is some set is the cross product $X\times X$. (Consider the
smallest/largest equiv. relation in $\mathbb I$)
\end{frame}

\begin{frame}{\large Mathematical  Preliminaries: Functions}
A {\em function} $f$ from $X$ to $Y$ is a relation $f \subseteq X\times Y$ such that
\[
\forall x\in X, \exists y,z\in Y.\, (x,y)\in f \wedge (x,z)\in f \Rightarrow y = z.
\]
In other words, each $x\in X$ has a unique value $y\in Y$ with $(x,y)\in f$ or functions are constrained relations.

\vspace{.1in}

We let $X \rightarrow Y$ denote the {\em set of all functions} from $X$ to $Y$
(i. e. $X \rightarrow Y \subset {\mathcal P}(X\times Y)$, why is the subset strict? Hint: it is not a relation), then the customary notation
for specifying functions can be defined as follows,
\[
f\co X \rightarrow Y \mbox{ iff } f\in X\rightarrow Y.
\]
\end{frame}

\begin{frame}{\large Mathematical  Preliminaries: Functions}
For {\em function application} it is customary to write
\[
f(x) = y
\]
for $(x,y)\in f$.  In this case we say that the function is {\em defined} at point $x$.
Otherwise we say that the function is {\em undefined} at point $x$ and we write $f(x)=\perp$.

\vspace{.1in}

Note that $f(\perp) = \perp$  and we say the $f$ is {\em strict}.
\vspace{.1in}

We say that $f\co X \rightarrow Y$ is a {\em total} function if $f$ is defined for all $x\in X$.
Otherwise we say that $f$ is a {\em partial} function.
\end{frame}

\begin{frame}{\large Mathematical  Preliminaries: Functions}
We can now make the notion of a predicate formal -- a predicate is a function whose range (co-domain) is restricted to the
boolean values:
\[
P: X \rightarrow {\mathbb B}
\]
where $P$ is a predicate that returns true or false for the objects in set $X$.

\vspace{.1in}

{\bf Example:} Let $U$ be the set of all possible objects -- a universe if you like, and let,
\[
\mathit{human}: U \rightarrow {\mathbb B}
\]
be the predicate that returns true if the object is a human and will return false otherwise, then
\begin{eqnarray*}
\mathit{human}(\mbox{socrates}) = \mathit{true}\\
\mathit{human}(\mbox{car}) = \mathit{false}
\end{eqnarray*}
\end{frame}


\begin{frame}{\large Mathematical  Preliminaries: Exercises}
\begin{enumerate}
\item In your own words explain what the function $m\co X\times Y \rightarrow Z$ does.
\item How would you describe the function $c \co X \rightarrow (Y \rightarrow Z)$?
\item In your own words explain what the relation $R \subseteq (X\times Y)\times (Z\times W)$ does.
\end{enumerate}
\end{frame}





\end{document}
%%%%%%%%%%%%%%%%%%%%%%%%%%% end of template1.tex %%%%%%%%%%%%%%%%%%%%%%%%%%%%%%%%
