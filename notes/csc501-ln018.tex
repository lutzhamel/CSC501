\documentclass{beamer}
% definitions for slides for the semantics course - CSC501
% Lutz Hamel, (c) 2006

\usetheme{Warsaw}
\usepackage{bussproofs}
\usepackage{amsmath}
\usepackage{amssymb}
\usepackage{latexsym}
\usepackage{xypic}
\usepackage{alltt}
\usepackage{rotating}
\usepackage{graphicx}
\usepackage{url}

\newcommand{\ul}[1]{\underline{#1}}
\newcommand{\orbar}{\,|\,}
\newcommand{\co}{\,\colon\;}
\newcommand{\syntaxset}[1]{\ensuremath{\mbox{\bf #1}}}
\newcommand{\nonterm}[1]{\ensuremath{\mbox{#1}}}
\newcommand{\term}[1]{\ensuremath{\mbox{\bf #1}}}
\newcommand{\ifstmt}[3]{\ensuremath{{\bf if}\; {#1}\;{\bf then}\;{#2}\;{\bf else}\;{#3}\;\term{end}}}
\newcommand{\whilestmt}[2]{\ensuremath{{\bf while}\; {#1}\;{\bf do}\;{#2}\; \term{end}}}
\newcommand{\funcstmt}[3]{\ensuremath{{\bf fun}\; {#1}\; {\bf is}\; {#2} \; {\bf return}\; {#3}}}
\newcommand{\localstmt}[1]{\ensuremath{{\bf local}\; {#1}}}
\newcommand{\pairmap}[3]{\ensuremath{( {#1}, {#2} ) \mapsto {#3}}}
\newcommand{\single}[1]{\ensuremath{\langle {#1}\rangle}}
\newcommand{\pair}[2]{\ensuremath{({#1}, {#2} )}}
\newcommand{\triple}[3]{\ensuremath{\langle {#1}, {#2}, {#3} \rangle}}
\newcommand{\cond}[3]{\ensuremath{({#1}?\,{#2} : {#3})}}
\newcommand{\condbar}[3]{\ensuremath{({#1}\rightarrow{#2} \mid {#3})}}
\newcommand{\sem}[2]{\ensuremath{{#1}-\!\!-\!\!\!\!\gg{#2}}}
\newcommand{\liftfunc}[1]{\ensuremath{\lfloor{#1}\rfloor}}
\newcommand{\bcode}{\begin{alltt}\scriptsize}
\newcommand{\ecode}{\end{alltt}}
\newcommand{\comment}[1]{}

\newcommand{\fframe}[1]{
\begin{center}
\fbox{
\begin{minipage}{3.6in}
{#1}
\end{minipage}
}
\end{center}
}

\begin{document}
\begin{frame}[fragile]{Program Correctness \& Iteration}

Review of proving programs correct that have loops and in-determinants.

\vspace{.1in}

Pre- and post-conditions for a program with a loop
\begin{quote}
\begin{minipage}{1.5in}
\{$\rm pre$\}\\
{\rm init} {\bf seq}\\
\{$\rm inv$\}  \\                                  
{\bf while} $b$ {\bf do}  \\                       
{\tt\verb"   "}\{$\rm inv$ $\wedge$ $b$\}  \\
{\tt\verb"   "}$c$    \\
{\tt\verb"   "}\{$\rm inv$\}  \\                            
\{$\rm inv$ $\wedge$ $\neg b$\}\\
\{$\rm post$\}  
\end{minipage}
\begin{minipage}{2in}
{\bf NOTE: } We now have pre- and postconditions for each statement in this iterative
program. {\em These conditions will hold in all iterations of the loop}.
\end{minipage}
\end{quote}                          
\end{frame}


\begin{frame}[fragile]{Program Correctness \& Iteration}
This gives us a new proof rule for {\em partial correctness} of loops:
\begin{quote}
if\\
{\verb"    "}$\{\rm pre\}$ {\rm init} $\{\rm inv\}$ $\wedge$\\
{\verb"    "}$\{{\rm inv} \wedge b\}$ c $\{\rm inv\}$ $\wedge$\\
{\verb"    "}${\rm inv} \wedge \neg b \Rightarrow {\rm post}$\\
then\\
{\verb"    "}`${\rm init}\; {\bf seq}\; \whilestmt{b}{c}$' is correct\\
\end{quote}

{\bf NOTE: } We call this partial correctness because we make no assertions about termination.
All we assert is that, if the computation terminates, then it will be correct.

\end{frame}


\begin{frame}[fragile]{Program Correctness \& Iteration}

\small

Or written in our notation, partial correctness of loops:
\begin{quote}
if\\
{\verb"    "}$\sem{({\rm init},S)}{Q} \wedge [{\rm pre}(S)\Rightarrow {\rm inv}(Q)]\; \wedge$\\ 
{\verb"    "}$\sem{(c,S)}{Q} \wedge \sem{(b,S)}{B} \wedge [({\rm inv}(S) \wedge B)\Rightarrow {\rm inv}(Q)]\; \wedge$\\
{\verb"    "}$\sem{(b,T)}{B} \wedge [(inv(T) \wedge \neg\,B) \Rightarrow post(T)]$\\
{\verb"    "}\\
then 
\begin{quote}
{\verb"    "}`${\rm init}\; {\bf seq}\; \whilestmt{b}{c}$'  is correct\\
\end{quote}
\end{quote}


\end{frame}

\begin{frame}[fragile]{Program Correctness \& Iteration}
\tiny
\begin{alltt}
% pow-n-loop.pl                                                                                                                              

:- ['sem-func.pl'].

:- >>> 'prove that the program p:'.
:- >>> '       var(i) seq'.
:- >>> '       var(z) seq'.
:- >>> '       assign(i,1) seq'.
:- >>> '       assign(z,m) seq'.
:- >>> '       whiledo(not(eq(i,n)),'.
:- >>> '               assign(i,add(i,1)) seq'.
:- >>> '               assign(z,mult(z,m))'.
:- >>> '       )'.
:- >>> 'is correct for any value of m and n with n>0'.
:- >>> 'pre(R) = initialstate(env([bind(vm,m),bind(vn,n)],s))'.
:- >>> 'post(T) = lookup(z,T,vm^vn)'.
:- >>> 'inv(Q) = lookup(i,Q,vi) ^ lookup(z,Q,vm^vi)'.

:- >>> 'define the parts of our program'.
init((var(i) seq var(z) seq assign(i,1) seq assign(z,m))).
guard(not(eq(i,n))).
body((assign(i,add(i,1)) seq assign(z,mult(z,m)))).

:- >>> 'define a model for our power operation'.
:- dynamic pow/3.
pow(B,1,B).

pow(B,P,R) :-
    T1 is P-1,
    pow(B,T1,T2),
    R is B*T2.
\end{alltt}
\end{frame}

\begin{frame}[fragile]{Program Correctness \& Iteration}
\tiny
\begin{alltt}
:- >>> 'first proof obligation'.
:- >>> 'assume precondition'.
:- asserta(initialstate(env([bind(vm,m),bind(vn,n)],s))).
:- >>> 'prove the invariant'.
:- init(P),initialstate(IS),(P,IS) -->> Q,lookup(i,Q,VI),lookup(z,Q,VZ),pow(vm,VI,VZ).
:- retract(initialstate(env([bind(vm,m),bind(vn,n)],s))).

:- >>> 'second  proof obligation'.
:- >>> 'assume invariant on s'.
:- asserta(initialstate(env([bind(vi,i),bind(vz,z),bind(vm,m),bind(vn,n)],s))).
:- asserta(pow(vm,vi,vz)).
% implies                                                                                                                                    
:- asserta(pow(vm,vi+1,vz*vm)).
:- >>> 'assume guard on s'.
:- asserta((not(eq(i,n)),s) -->> true).
:- >>> 'prove the invariant on Q'.
:- body(Bd),initialstate(IS),(Bd,IS) -->> Q,lookup(i,Q,VI),lookup(z,Q,VZ),pow(vm,VI,VZ).
:- retract(initialstate(env([bind(vi,i),bind(vz,z),bind(vm,m),bind(vn,n)],s))).
:- retract(pow(vm,vi,vz)).
:- retract(pow(vm,vi+1,vz*vm)).
:- retract((not(eq(i,n)),s) -->> true).
\end{alltt}
\end{frame}

\begin{frame}[fragile]{Program Correctness \& Iteration}
\tiny
\begin{alltt}
:- >>> 'third  proof obligation'.
:- >>> 'assume the invariant on s'.
:- asserta(initialstate(env([bind(vi,i),bind(vz,z),bind(vm,m),bind(vn,n)],s))).
:- asserta(pow(vm,vi,vz)).
:- >>> 'assume NOT guard on any s'.
:- asserta((not(eq(i,n)),s) -->> not(true)).
% implies                                                                                                                                    
:- asserta((eq(i,n),s) -->> true).
% implies                                                                                                                                    
:- asserta(pow(vm,vn,vz)).
:- >>> 'prove postcondition on s'.
:- initialstate(IS),lookup(z,IS,VZ),pow(vm,vn,VZ).
:- retract(initialstate(env([bind(vi,i),bind(vz,z),bind(vm,m),bind(vn,n)],s))).
:- retract(pow(vm,vi,vz)).
:- retract((not(eq(i,n)),s) -->> not(true)).
:- retract((eq(i,n),s) -->> true).
:- retract(pow(vm,vn,vz)).
\end{alltt}
\end{frame}



\begin{frame}[fragile]{Program Correctness \& Recursive Function Calls}
\small
The key insights to proving programs with recursive function correct is that 
\begin{itemize}
\item
There exist at least one
variable in argument list of the recursive function 
we call the {\em recursion variable} which measures the progress of
the recursion.

\item
The body of the recursive function consists of at least two separate parts:
\begin{itemize}
\item the part of the body that gets executed when recursion terminates (the base case)
\item the part of the body that gets executed during normal recursion (the recursive step)
\end{itemize}

\item
We can always identify an ``accumulator variable'' that contains the partially computed result at
any particular step in the recursion.
\end{itemize}
\end{frame}

\begin{frame}[fragile]{Program Correctness \& Recursive Function Calls}

We introduce two new predicates for the correctness proof:
\begin{description}
\item[callcond] - describes the current condition within a called function
\item[finv] - this is the {\em function invariant} and is similar to the loop invariant,
it describes the condition at the ``accumulator variable''
\end{description}
\end{frame}


\begin{frame}[fragile]{Program Correctness \& Recursive Function Calls}
\scriptsize
This allows us to state the following proof schema:
\vspace{.1in}

\begin{minipage}{2in}
\tiny
\begin{alltt}
function f (i)  
   if (<term cond>)
      {\rm\{callcond ^ term cond\}}
      return = <base case value>
      {\rm\{finv\}}
   else
      {\rm\{callcond ^ not term cond\}}
      return = <local comp> + f(i-1)
      {\rm\{finv\}}

{\rm\{pre ^ finv\}}
z = f(k)  
{\rm\{post\}}
\end{alltt}
\end{minipage}
$\Rightarrow$
\begin{minipage}{1.5in}
\tiny
\begin{alltt}
function f (i)  
   if (<term cond>)
      {\rm\{callcond ^ term cond\}}
      body1
      {\rm\{finv\}}
   else
      {\rm\{callcond ^ not term cond\}}
      body2
      {\rm\{finv\}}

{\rm\{pre ^ finv\}}
final  
{\rm\{post\}}
\end{alltt}
\end{minipage}

\vspace{.2in}

Proof Obligations:
\begin{enumerate}
\item \{callcond $\wedge$ termination condition\} body1 \{finv\}
\item \{callcond $\wedge$ $\neg$ termination condition\} body2 \{finv\}
\item \{pre $\wedge$ finv\} final \{post\}
\end{enumerate}
\end{frame}


\begin{frame}[fragile]{Program Correctness \& Recursive Function Calls}
\tiny
\begin{alltt}
% pow-n-func.pl                                                                                                                              
:- ['sem-func.pl'].
:- >>> 'prove that the program p:'.
:- >>> '     (fun(pow,'.
:- >>> '            [b,p],'.
:- >>> '            var(result) seq'.
:- >>> '            if(eq(p,1),'.
:- >>> '               assign(result,b),'.
:- >>> '               assign(result,mult(b,call(pow,[assign(b,b),assign(p,sub(p,1))])))),'.
:- >>> '            result) seq'.
:- >>> '        var(z) seq'.
:- >>> '        assign(z,call(pow,[assign(b,m),assign(p,n)]))'.
:- >>> 'is correct for any value of n and m and n>0'.
:- >>> 'pre(R) = initialstate(env([bind(vm,m),bind(vn,n)],s))'.
:- >>> 'post(T) = lookup(z,T,vm^vn)'.
:- >>> 'inv(Q) = lookup(b,Q,vb) ^ lookup(p,Q,vp) ^ lookup(result,Q,vb^vp)'.

:- >>> 'define the parts of our program'.
guard(eq(p,1)).
body1(var(result) seq assign(result,b)).
body2(var(result) seq assign(result,mult(b,call(pow,[assign(b,b),assign(p,sub(p,1))])))).
final(var(z) seq assign(z,result)).

:- >>> 'define a model for our power operation'.
:- dynamic pow/3.
pow(B,1,B).

pow(B,P,R) :-
    T1 is P-1,
    pow(B,T1,T2),
    R is B*T2.
\end{alltt}
\end{frame}

\begin{frame}[fragile]{Program Correctness \& Recursive Function Calls}
\tiny
\begin{alltt}
:- >>> 'Proof by case analysis on recursion variable p'.
:- >>> 'first proof obligation'.
:- >>> 'assume call condition -- guard is true'.
:- asserta(initialstate(env([bind(vb,b),bind(1,p)],s))).
:- >>> 'prove the invariant'.
:- body1(P),initialstate(IS),(P,IS)-->>Q,lookup(b,Q,VB),lookup(p,Q,VP),lookup(result,Q,VR),pow(VB,VP,VR).
:- retract(initialstate(env([bind(vb,b),bind(1,p)],s))).

:- >>> 'second proof obligation'.
:- >>> 'assume call condition -- guard is false: vp =/= 1'.
:- asserta(initialstate(env([bind(vb,b),bind(vp,p)],s))).
% assume that the recursive call returns a value k                                                                                           
:- asserta((call(pow,[assign(b,b),assign(p,sub(p,1))]),S) -->> (k,S)).
% where the value k is defined as                                                                                                            
:- asserta(pow(vb,vp-1,k)).
% which implies                                                                                                                              
:- asserta(pow(vb,vp,vb*k)).
:- >>> 'prove the invariant'.
:- body2(P),initialstate(IS),(P,IS)-->>Q,lookup(b,Q,VB),lookup(p,Q,VP),lookup(result,Q,VR),pow(VB,VP,VR).
:- retract(initialstate(env([bind(vb,b),bind(vp,p)],s))).
:- retract((call(pow,[assign(b,b),assign(p,sub(p,1))]),S) -->> (k,S)).
:- retract(pow(vb,vp-1,k)).
:- retract(pow(vb,vp,vb*k)).
\end{alltt}
\end{frame}

\begin{frame}[fragile]{Program Correctness \& Recursive Function Calls}
\tiny
\begin{alltt}
:- >>> 'third proof obligation'.
:- >>> 'assume precondition -- result = q = vm^vn'.
:- >>> '   this follows from the second proof obligation with vp = vn'.
:- asserta(initialstate(env([bind(vm,m),bind(vn,n),bind(q,result)],s))).
:- asserta(pow(vm,vn,q)).
:- >>> 'prove the post condition'.
:- final(P),initialstate(IS),(P,IS)-->>Q,lookup(z,Q,VZ),pow(vm,vn,VZ).
:- retract(initialstate(env([bind(vm,m),bind(vn,n),bind(q,result)],s))).
:- retract(pow(vm,vn,q)).
\end{alltt}
\end{frame}

\end{document}

