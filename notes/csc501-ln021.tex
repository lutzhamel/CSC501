\documentclass{beamer}
% definitions for slides for the semantics course - CSC501
% Lutz Hamel, (c) 2006

\usetheme{Warsaw}
\usepackage{bussproofs}
\usepackage{amsmath}
\usepackage{amssymb}
\usepackage{latexsym}
\usepackage{xypic}
\usepackage{alltt}
\usepackage{rotating}
\usepackage{graphicx}
\usepackage{url}

\newcommand{\ul}[1]{\underline{#1}}
\newcommand{\orbar}{\,|\,}
\newcommand{\co}{\,\colon\;}
\newcommand{\syntaxset}[1]{\ensuremath{\mbox{\bf #1}}}
\newcommand{\nonterm}[1]{\ensuremath{\mbox{#1}}}
\newcommand{\term}[1]{\ensuremath{\mbox{\bf #1}}}
\newcommand{\ifstmt}[3]{\ensuremath{{\bf if}\; {#1}\;{\bf then}\;{#2}\;{\bf else}\;{#3}\;\term{end}}}
\newcommand{\whilestmt}[2]{\ensuremath{{\bf while}\; {#1}\;{\bf do}\;{#2}\; \term{end}}}
\newcommand{\funcstmt}[3]{\ensuremath{{\bf fun}\; {#1}\; {\bf is}\; {#2} \; {\bf return}\; {#3}}}
\newcommand{\localstmt}[1]{\ensuremath{{\bf local}\; {#1}}}
\newcommand{\pairmap}[3]{\ensuremath{( {#1}, {#2} ) \mapsto {#3}}}
\newcommand{\single}[1]{\ensuremath{\langle {#1}\rangle}}
\newcommand{\pair}[2]{\ensuremath{({#1}, {#2} )}}
\newcommand{\triple}[3]{\ensuremath{\langle {#1}, {#2}, {#3} \rangle}}
\newcommand{\cond}[3]{\ensuremath{({#1}?\,{#2} : {#3})}}
\newcommand{\condbar}[3]{\ensuremath{({#1}\rightarrow{#2} \mid {#3})}}
\newcommand{\sem}[2]{\ensuremath{{#1}-\!\!-\!\!\!\!\gg{#2}}}
\newcommand{\liftfunc}[1]{\ensuremath{\lfloor{#1}\rfloor}}
\newcommand{\bcode}{\begin{alltt}\scriptsize}
\newcommand{\ecode}{\end{alltt}}
\newcommand{\comment}[1]{}

\newcommand{\fframe}[1]{
\begin{center}
\fbox{
\begin{minipage}{3.6in}
{#1}
\end{minipage}
}
\end{center}
}

\begin{document}

\begin{frame}[fragile]{Compiler Correctness}

The idea of a correct compiler is that the intended behavior
of the source program is preserved by the translated program.

\end{frame}

\begin{frame}[fragile]{Compiler Correctness}

\small
Recall that {\tt translate(c,c')} is our predicate that takes a source program {\tt c} and
produces the target code {\tt c'}.

\tiny
\begin{alltt}
:- >>> 'executing: assign(x,2) seq assign(y,mult(2,x))'.

program1(assign(x,2) seq assign(y,mult(2,x))).

:- >>> 'Source Semantics'.
:- program1(P),(P,e) -->> Env, Env = env([bind(4,y)|_],e).

:- >>> 'Target Semantics'.
:- program1(P),translate(P,C),(C,C,([],e)) -->> ([],Env), Env = env([bind(4,y)|_],e).

\end{alltt}
\end{frame}

\begin{frame}[fragile]{Compiler Correctness}

\tiny
\begin{alltt}
:- >>> 'executing: if(le(2,3),assign(i,3),assign(i,4))'.

program2(if(le(2,3),assign(i,3),assign(i,4))).

:- >>> 'Source Semantics'.
:- program2(P),(P,e) -->> Env, Env = env([bind(3,i)|_],e).

:- >>> 'Target Semantics'.
:- program2(P),translate(P,C),(C,C,([],e)) -->> ([],Env), Env = env([bind(3,i)|_],e).
\end{alltt}
\end{frame}

\begin{frame}[fragile]{Compiler Correctness}

\tiny
\begin{alltt}
:- >>> 'executing: '.
:- >>> 'assign(n,3) seq'.
:- >>> 'assign(i,1) seq'.
:- >>> 'assign(z,1) seq'.
:- >>> 'whiledo(not(eq(i,n)),'.
:- >>> '   assign(i,add(i,1)) seq'.
:- >>> '   assign(z,mult(z,i)))'.

program3(assign(n,3) seq
         assign(i,1) seq
         assign(z,1) seq
         whiledo(not(eq(i,n)),
           assign(i,add(i,1)) seq
           assign(z,mult(z,i)))).

:- >>> 'Source Semantics'.
:- program3(P),(P,e) -->> Env, Env = env([bind(6,z)|_],e).

:- >>> 'Target Semantics'.
:- program3(P),translate(P,C),(C,C,([],e)) -->> ([],Env), Env = env([bind(6,z)|_],e).\end{alltt}
\end{frame}


\begin{frame}[fragile]{Compiler Correctness}

\small
The compiler correctness problem is a special case of our 
{\em semantic equivalence} problem.

\vspace{.1in}
Consider the case of translating statements, then for any $c\in\syntaxset{Com}$
and its corresponding translated code $c'$ with ${\rm translate}(c,c')$,

\vspace{.1in}
{\scriptsize
\[
c \sim c' \mbox{ iff }\forall e,\exists E[\sem{(c,e)}{E} \wedge 
	\sem{(c',([\, ],e))}{([\, ],E)}]
\]
}
\end{frame}


\begin{frame}[fragile]{Compiler Correctness}

\small
Similarly,  the case of translating arithmetic expressions, then for any $a\in\syntaxset{Aexp}$
and its corresponding translated code $a'$ with ${\rm translate}(a,a')$,

\vspace{.1in}
{\scriptsize
\[
a \sim a' \mbox{ iff } \forall e,\exists V[\sem{(a,e)}{V} \wedge 
	\sem{(a',([\, ],e))}{([ V ],e)}]
\]
}
\end{frame}


\begin{frame}[fragile]{Compiler Correctness}

\small
And finally,  the case of translating boolean expressions, then for any $b\in\syntaxset{Bexp}$
and its corresponding translated code $b'$ with ${\rm translate}(b,b')$,

\vspace{.1in}
{\scriptsize
\[
b \sim b' \mbox{ iff } \forall e,\exists B, B' [\sem{(b,e)}{B} \wedge 
	\sem{(b',([\, ],e))}{([ B' ],e)} \wedge
	 B'  = B]
\]
}
\end{frame}

\begin{frame}[fragile]{Compiler Correctness}

\small
The correctness proofs themselves are proofs by induction.  
There is no problem here because translation is a syntactic property -- we are NOT trying to 
prove an algorithm written in the source language correct -- we are trying to prove that whatever
algorithm was written in the source language (correct or not) is preserved by the translation.

\end{frame}

\begin{frame}[fragile]{Compiler Correctness: Booleans}

\tiny
\begin{alltt}
% it is sufficient to show that for every b in Bexp we have
%
%  (forall b,e)[(b,e) -->> V1 ^ 
%               translate(b,C) ^ 
%               (C,C,([],e)) --> ([V2],e) ^
%               V1 = V2]
%
% proof by induction on Bexp.

%%%%%%%%%%%%%%%%%%%%%%%%%%%%%%%%%%%%%%%%%%%%%%%%%%%%%%%%%%%%%%%%%%%%%
:- >>> 'case true'.
:- (true,e) -->> V1,
   translate(true,C),
   (C,C,([],e)) -->> ([V2],e),
   V1 = V2.

%%%%%%%%%%%%%%%%%%%%%%%%%%%%%%%%%%%%%%%%%%%%%%%%%%%%%%%%%%%%%%%%%%%%%
:- >>> 'case false'.
:- (false,e) -->> V1,
   translate(false,C),
   (C,C,([],e)) -->>([V2],e),
   V1 = V2.
\end{alltt}
\end{frame}

\begin{frame}[fragile]{Compiler Correctness: Booleans}

\tiny
\begin{alltt}
% the following assumptions hold for the relational operators
:- asserta((a,e) -->> va).  % arithmetic exp a -> va
:- asserta((b,e) -->> vb).  % arithmetic exp b -> vb
:- asserta(int(va)).
:- asserta(int(vb)).
:- asserta(translate(a,ca)).
:- asserta(translate(b,cb)).
:- asserta((ca,_,(S,E)) -->> ([va|S],E)).
:- asserta((cb,_,(S,E)) -->> ([vb|S],E)).

:- >>> 'case eq(a,b)'.
:- (eq(a,b),e) -->> V1,
   translate(eq(a,b),C),
   (C,C,([],e))  -->> ([V2],e),
   V1 = V2.

:- >>> 'case le(a,b)'.
:- (le(a,b),e) -->> V1,
   translate(le(a,b),C),
   (C,C,([],e)) -->> ([V2],e),
   V1 = V2.

:- retract((a,e) -->> va).
:- retract((b,e) -->> vb).
:- retract(int(va)).
:- retract(int(vb)).
:- retract(translate(a,ca)).
:- retract(translate(b,cb)).
:- retract((ca,_,(S,E)) -->> ([va|S],E)).
:- retract((cb,_,(S,E)) -->> ([vb|S],E)).
\end{alltt}
\end{frame}

\begin{frame}[fragile]{Compiler Correctness: Booleans}

\tiny
\begin{verbatim}
:- asserta((a,e) -->> va).
:- asserta((b,e) -->> vb).
:- asserta(bool(va)).
:- asserta(bool(vb)).
:- asserta(translate(a,ca)).
:- asserta(translate(b,cb)).
:- asserta((ca,_,(S,E)) -->> ([va|S],E)).
:- asserta((cb,_,(S,E)) -->> ([vb|S],E)).

:- >>> 'case not(a)'.
:- (not(a),e) -->> V1,
    translate(not(a),C),
    (C,C,([],e)) -->> ([V2],e),
    V1 = V2.

:- >>> 'case and(a,b)'.
:- (and(a,b),e) -->> V1,
    translate(and(a,b),C),
    (C,C,([],e)) -->> ([V2],e),
    V1 = V2.

:- >>> 'case or(a,b)'.
. . .

:- retract((a,e) -->> va).
:- retract((b,e) -->> vb).
:- retract(bool(va)).
:- retract(bool(vb)).
:- retract(translate(a,ca)).
:- retract(translate(b,cb)).
:- retract((ca,_,(S,E)) -->> ([va|S],E)).
:- retract((cb,_,(S,E)) -->> ([vb|S],E)).
\end{verbatim}
\end{frame}

\begin{frame}[fragile]{Compiler Correctness: Commands}

\tiny
\begin{verbatim}
% it is sufficient to show that for every c in Com we have
%
%  (forall cc,e)[(cc,e) -->> E1) ^
%               translate(cc,C) ^
%               (C,C,([],e)) -->> ([],E1))]
%
% proof by induction on Com.


%%%%%%%%%%%%%%%%%%%%%%%%%%%%%%%%%%%%%%%%%%%%%%%%%%%%%%%%%%%%%%%%%%%%%
:- >>> 'case skip'.
:- (skip,e) -->> E1,
    translate(skip,C),
    (C,C,([],e)) -->> ([],E1).

%%%%%%%%%%%%%%%%%%%%%%%%%%%%%%%%%%%%%%%%%%%%%%%%%%%%%%%%%%%%%%%%%%%%%
:- >>> 'case assign'.
:- asserta((a,e) -->> va).
:- asserta(translate(a,ca)).
:- asserta((ca,_,(L,S)) -->> ([va|L],S)).

:- (assign(x,a),e) -->> E1,
   translate(assign(x,a),C),
   (C,C,([],e)) -->> ([],E1).

:- retract((a,e) -->> va).
:- retract(translate(a,ca)).
:- retract((ca,_,(L,S)) -->> ([va|L],S)).
\end{verbatim}
\end{frame}

\begin{frame}[fragile]{Compiler Correctness: Commands}

\tiny
\begin{verbatim}
%%%%%%%%%%%%%%%%%%%%%%%%%%%%%%%%%%%%%%%%%%%%%%%%%%%%%%%%%%%%%%%%%%%%%
:- >>> 'case seq'.
:- asserta((c1,_) -->> e1).
:- asserta((c2,e1) -->> e2).
:- asserta(translate(c1,cc1)).
:- asserta(translate(c2,cc2)).
:- asserta((cc1,_,([],_)) -->> ([],e1)).
:- asserta((cc2,_,([],e1)) -->> ([],e2)).

:- (seq(c1,c2),e) -->> E1,
    translate(seq(c1,c2),C),
    (C,C,([],e)) -->> ([],E1).

:- retract((c1,_) -->> e1).
:- retract((c2,e1) -->> e2).
:- retract(translate(c1,cc1)).
:- retract(translate(c2,cc2)).
:- retract((cc1,_,([],_)) -->> ([],e1)).
:- retract((cc2,_,([],e1)) -->> ([],e2)).
\end{verbatim}
\end{frame}



\begin{frame}[fragile]{Compiler Correctness: Commands}

\tiny
\begin{verbatim}
:- >>> 'case if'.
:- asserta((c1,e) -->> e1).
:- asserta((c2,e) -->> e2).
:- asserta(translate(b,cb)).
:- asserta(translate(c1,cc1)).
:- asserta(translate(c2,cc2)).
:- asserta((cc1,_,([],e)) -->> ([],e1)).
:- asserta((cc2,_,([],e)) -->> ([],e2)).

:- >>> '    case analysis, b=true'.
:- asserta((b,e) -->> true).
:- asserta((cb,_,([],e)) -->> ([true],e)).

:- (if(b,c1,c2),e) -->> E1,
    translate(if(b,c1,c2),C),
    (C,C,([],e)) -->> ([],E1).

:- retract((b,e) -->> true).
:- retract((cb,_,([],e)) -->> ([true],e)).

:- >>> '    case analysis, b=false'.
:- asserta((b,e) -->> false).
:- asserta((cb,_,([],e)) -->> ([false],e)).

:- (if(b,c1,c2),e) -->> E1,
    translate(if(b,c1,c2),C),
    (C,C,([],e)) -->> ([],E1).

:- retract((b,e) -->> false).
:- retract((cb,_,([],e)) -->> ([false],e)).
...
\end{verbatim}
\end{frame}

\begin{frame}[fragile]{Compiler Correctness: Commands}

\tiny
\begin{verbatim}
%%%%%%%%%%%%%%%%%%%%%%%%%%%%%%%%%%%%%%%%%%%%%%%%%%%%%%%%%%%%%%%%%%%%%
:- >>> 'case while'.
:- asserta((c,e) -->> vc).
:- asserta(translate(b,cb)).
:- asserta(translate(c,cc)).
:- asserta((cc,_,([],e)) -->> ([],vc)).

:- >>> '    case analysis, b=false'.
:- asserta((b,e) -->> false).
:- asserta((cb,_,([],e)) -->> ([false],e)).

:- (whiledo(b,c),e) -->> e,
    translate(whiledo(b,c),C),
    (C,C,([],e)) -->> ([],e).

:- retract((b,e) -->> false).
:- retract((cb,_,([],e)) -->> ([false],e)).
\end{verbatim}
\end{frame}


\begin{frame}[fragile]{Compiler Correctness: Commands}

\scriptsize
Here we prove that one iteration of the loop executes correctly, subsequent iterations of
the loop are assumed to terminate in some state {\tt vt}.

\tiny
\begin{alltt}
:- >>> '    case analysis, b=true'.
:- asserta((b,e) -->> true).
:- asserta((cb,_,([],e)) -->> ([true],e)).

% lemma: prove that the translation of whiledo(b,c) is the piece of stack machine code shown
:- translate(
    whiledo(b,c),
    [label(whilelabel1),cb,jmpf(whilelabel2),cc,jmp(whilelabel1),label(whilelabel2)]).

% it is sufficient to prove the correctness for the loop for only one iteration
% assume that the remaining itereations after the first iteration give us some 
% terminal state vt - we assume this both in the source and target language
:- asserta((whiledo(b,c),vc) -->> vt). 
:- asserta(([label(whilelabel1),cb,jmpf(whilelabel2),cc,jmp(whilelabel1),label(whilelabel2)],
                _,([],vc)) -->> ([],vt)).

:- (whiledo(b,c),e) -->> vt,
    translate(whiledo(b,c),C),
    (C,C,([],e)) -->> ([],vt).

:- retract((b,e) -->> true).
:- retract((cb,_,([],e)) -->> ([true],e)).
:- retract((whiledo(b,c),vc) -->> vt). 
:- retract(([label(whilelabel1),cb,jmpf(whilelabel2),cc,jmp(whilelabel1),label(whilelabel2)],
                _,([],vc)) -->> ([],vt)).
\end{alltt}
\end{frame}

\begin{frame}[fragile]{Assignment}

Assignment \#6 -- see website
\end{frame}

\end{document}

