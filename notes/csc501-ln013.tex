%template1.tex
%The following LaTeX source file represents the simplest kind of slide presentation; no overlays, no included graphics. Substitute your favorite style for ``pascal''. To create the PDF file template1.pdf, (1) be sure to use the prosper class, then (2) execute the command latex template1.tex, and (3) the command dvipdf template1.dvi.

%%%%%%%%%%%%%%%%%%%%%%%%%%%%%%% template1.tex %%%%%%%%%%%%%%%%%%%%%%%%%%%%%%%%%%%
\documentclass{beamer}
% definitions for slides for the semantics course - CSC501
% Lutz Hamel, (c) 2006

\usetheme{Warsaw}
\usepackage{bussproofs}
\usepackage{amsmath}
\usepackage{amssymb}
\usepackage{latexsym}
\usepackage{xypic}
\usepackage{alltt}
\usepackage{rotating}
\usepackage{graphicx}
\usepackage{url}

\newcommand{\ul}[1]{\underline{#1}}
\newcommand{\orbar}{\,|\,}
\newcommand{\co}{\,\colon\;}
\newcommand{\syntaxset}[1]{\ensuremath{\mbox{\bf #1}}}
\newcommand{\nonterm}[1]{\ensuremath{\mbox{#1}}}
\newcommand{\term}[1]{\ensuremath{\mbox{\bf #1}}}
\newcommand{\ifstmt}[3]{\ensuremath{{\bf if}\; {#1}\;{\bf then}\;{#2}\;{\bf else}\;{#3}\;\term{end}}}
\newcommand{\whilestmt}[2]{\ensuremath{{\bf while}\; {#1}\;{\bf do}\;{#2}\; \term{end}}}
\newcommand{\funcstmt}[3]{\ensuremath{{\bf fun}\; {#1}\; {\bf is}\; {#2} \; {\bf return}\; {#3}}}
\newcommand{\localstmt}[1]{\ensuremath{{\bf local}\; {#1}}}
\newcommand{\pairmap}[3]{\ensuremath{( {#1}, {#2} ) \mapsto {#3}}}
\newcommand{\single}[1]{\ensuremath{\langle {#1}\rangle}}
\newcommand{\pair}[2]{\ensuremath{({#1}, {#2} )}}
\newcommand{\triple}[3]{\ensuremath{\langle {#1}, {#2}, {#3} \rangle}}
\newcommand{\cond}[3]{\ensuremath{({#1}?\,{#2} : {#3})}}
\newcommand{\condbar}[3]{\ensuremath{({#1}\rightarrow{#2} \mid {#3})}}
\newcommand{\sem}[2]{\ensuremath{{#1}-\!\!-\!\!\!\!\gg{#2}}}
\newcommand{\liftfunc}[1]{\ensuremath{\lfloor{#1}\rfloor}}
\newcommand{\bcode}{\begin{alltt}\scriptsize}
\newcommand{\ecode}{\end{alltt}}
\newcommand{\comment}[1]{}

\newcommand{\fframe}[1]{
\begin{center}
\fbox{
\begin{minipage}{3.6in}
{#1}
\end{minipage}
}
\end{center}
}

\begin{document}


\begin{frame}[fragile]{Function Calls}
A function is a parameterized portion of code within a larger program which performs a specific task. 
\vspace{.1in}

A function in a programming language is  in many ways similar to a mathematical function, but can have side-effects outside of the simple "function return value" .
\end{frame}

\begin{frame}[fragile]{Function Calls}

\small
What exactly happens at function calls?  Consider this simple example in a C like language:

\vspace{.1in}

\begin{minipage}{1.3in}
\begin{alltt}\scriptsize
fun f (x) \{
    var t = x + 1;
    return t;
\}

...

var i = f(3);
print(i);

...

\end{alltt}
\end{minipage}
\begin{minipage}{2.9in}
%\scriptsize
{\bf Observations:} 
\begin{itemize}
\item In many programming languages communication to a function is via {\em formal} and {\em actual} parameters {\em by value}.  
\item Communication from a function is via {\em return values}.
\item Functions have local scope -- locally defined variables.
\item The formal parameter are declared as local variables.
\end{itemize}
\end{minipage}
\end{frame}

\begin{frame}[fragile]{Function Calls}

\scriptsize
How do we interpret function definitions and calls?

\vspace{.1in}

\begin{minipage}{2.2in}
\begin{alltt}\scriptsize
fun f (x) \{
    var t = x + 1;
    return t;
\}

...

var i = f(3);
print(i);

...

\end{alltt}
\end{minipage}
\begin{minipage}{1.4in}
\begin{alltt}\scriptsize
var f = fun (x) \{
    var t = x + 1;
    return t;
\}

...

var i = f (x = 3);
print(i);

...

\end{alltt}
\end{minipage}
{\bf Observations:}
\begin{itemize}
\item Function names act like variables that have been initialized with a {\em function value}!
\item During a function call we simply look up the function stored in the variable, activate it,
and pass it the actual parameters.
\item The actual parameters act like initial values for the formal parameters during a
function call.
\end{itemize}
\end{frame}

\begin{frame}[fragile]{Function Parameters}

\small

There are two different ways to associate actual with formal parameters:
\begin{description}
\item[Positional Correspondence] -- Here the actual parameters are associated with the formal parameters  via their position in the actual and formal parameter lists.\\
\begin{alltt}
fun foo(a,b) ...
... call foo(1,2)...
\end{alltt}
Here actual parameter 1 will be associated with the formal parameter {\tt a} and actual parameter 2 will be associated with formal parameter {\tt b}.
\item[Keyword Correspondence] -- Here the actual arguments are associated with the formal parameters by an explicit assignment.
\begin{alltt}
fun foo(a,b) ...
... call foo(a=1,b=2)...
\end{alltt}
\end{description}
\end{frame}

\begin{frame}[fragile]{Function Parameters}
Our implementation:
\begin{itemize}
\item
We implement function parameters by call-by-value using keyword correspondence.
\item
The biggest problem we will be facing are {\em side effects}; function calls can modify 
global variables and functions can appear in both arithmetic and boolean expressions!
We need to be able to model both of these aspects of functions in imperative languages.
\end{itemize}
\end{frame}

\begin{frame}[fragile]{Function Calls}
\scriptsize
In our language we introduce  new syntactic structures to deal with  functions.
The first one is the {\em function call}:
\begin{quote}
\begin{alltt}
A ::= call(f,[ PL ])  |  call(f, [ ])
\end{alltt}
\end{quote}
here {\tt f} represents function names and {\tt PL} is the non-terminal for actual parameter lists,
\begin{quote}
\begin{alltt}
PL ::= P , PL  |  P
P  ::= assign(x,A) 
\end{alltt}
\end{quote}
The second structure is the {\em function definition}:
\begin{quote}
\begin{alltt}
C  ::= fun(f,[ FL ],C,A)  |  fun(f,[ ],C,A)
FL ::= x , FL  |  x
\end{alltt}
\end{quote}
Here {\tt FL} is the formal argument list. 
This is a highly stylized version of a function definition, think of it as
\begin{quote}
\begin{alltt}
fun f(x) is C returns A
\end{alltt}
\end{quote}
\end{frame}

\begin{frame}[fragile]{Function Calls}
\scriptsize
This will enable us to write programs such as
\begin{quote}
\begin{alltt}
fun inc(x) is skip returns add(x,1) seq
var(q) seq 
assign(q,call(inc,[assign(x,q)]))
\end{alltt}
\end{quote}
Or consider the factorial program
\begin{quote}
\begin{alltt}
fun fact(i) is 
        var(result) seq 
        if(eq(i,1), 
            assign(result,1),
            assign(result,mult(i,call(fact,[assign(i,sub(i,1))]))
    returns result seq
var(x) seq
assign(x,call(fact,[assign(i,3)]))
\end{alltt}
\end{quote}

\end{frame}


\begin{frame}[fragile]{Function Calls}
\scriptsize

The semantics of a function declaration is binding a {\em funval} to a {\em function variable},
{\tiny
\begin{quote}
\begin{alltt}
(fun(F,L,C,A),State) -->> OState :-        % function declaration
    declarevar(F,{\color{red}funval(L,C,A)},State,OState).
\end{alltt}
\end{quote}
}
The semantics of a function call is returning a value from the execution of the function,
{\tiny
\begin{quote}
\begin{alltt}
(call(F,[ ]),State) -->> Val :-        % function call
    lookup(F,State,{\color{red}funval([ ],C,A))},
    pushenv(State,LocalState),
    ({\color{red}C},LocalState) -->> S1,
    ({\color{red}A},S1) -->> Val,
    popenv(S1,_),!.      % TROUBLE!!!
\end{alltt}
\end{quote}
\begin{quote}
\begin{alltt}
(call(F,PList),State) -->> Val :-        % function call
    lookup(F,State,{\color{red}funval(FList,C,A))},
    pushenv(State,LocalState),
    declareparams(FList,LocalState,S1),
    initparams(PList,S1,S2),    
    ({\color{red}C},S2) -->> S3,
    ({\color{red}A},S3) -->> Val,
    popenv(S3,_),!.      % TROUBLE!!!
\end{alltt}
\end{quote}
}
{\bf Note:} We are only dealing with side effect free functions. Why?

\end{frame}


\begin{frame}[fragile]{Function Calls}
\scriptsize
In order to include side effects we need to create a new semantic function
that interprets arithmetic expressions, including function calls, and carries side effect information
with it.

{\tiny
\begin{quote}
\begin{alltt}
(call(F,[ ]),State) -->> {\color{red}(Val,OState)} :-        % function call
    lookup(F,State,funval([ ],C,A)),
    pushenv(State,LocalState),
    (C,LocalState) -->> S1,
   (A,S1) -->> {\color{red}(Val,S2)},
    popenv(S2,{\color{red}OState}),!.     
\end{alltt}
\end{quote}
\begin{quote}
\begin{alltt}
(call(F,PList),State) -->> {\color{red}(Val,OState)} :-        % function call
    lookup(F,State,funval(FList,C,A)),
    pushenv(State,LocalState),
    declareparams(FList,LocalState,S1),
    initparams(PList,S1,S2),    
    (C,S2) -->> S3,
    (A,S3) -->> {\color{red}(Val,S4)},
    popenv(S4,{\color{red}OState}),!.
\end{alltt}
\end{quote}
}
\end{frame}

\begin{frame}[fragile]{Function Calls}
\scriptsize

Side effects impose an ordering on the evaluation of operands in an expression:
\begin{quote}
\begin{alltt}\tiny
(add(A,B),State) -->> (Val,OState) :-      % addition
     (A,State) -->> (ValA,S),
     (B,S) -->> (ValB,OState),
     Val xis ValA + ValB,!.
\end{alltt}
\end{quote}
compare this to our side effect free semantics:
\begin{quote}
\begin{alltt}\tiny
(add(A,B),State) -->> Val :-       % addition
     (A,State) -->> ValA,
     (B,State) -->> ValB,
     Val xis ValA + ValB,!.
\end{alltt}
\end{quote}
\end{frame}

\begin{frame}[fragile]{Function Calls}
\scriptsize
In optimizing programming languages function calls with side effects together with boolean expressions are a notorious
source of bugs.  Consider this C program:
\begin{quote}\tiny
\begin{verbatim}
#include <stdio.h>

int cnt = 0;
int x = 0;

int f(void) {
  cnt++;
  return x;
}

void main(void) {

  if (f() && f())
    printf("condition is true\n");
  else
    printf("condition is false\n");

  printf("function f was called %d time(s)\n",cnt);
}
\end{verbatim}
\end{quote}
What kind of output would you expect with {\tt x == 0}?  What kind of output would you expect
with {\tt x == 1}?

\end{frame}



\begin{frame}[fragile]{Function Calls}
\scriptsize

Side effects ripple through the whole language definition. Consider the boolean expressions:
\begin{quote}
\begin{alltt}\tiny
(not(A),State) -->> (Val,OState) :-      % not
    (A,State) -->> (ValA,OState),
    Val xis (not ValA),!.

(and(A,B),State) -->> (Val,OState) :-     % and
    (A,State) -->> (ValA,S),
    (B,S) -->> (ValB,OState),
    Val xis (ValA and ValB),!.
\end{alltt}
\end{quote}
Consider the statements:
\begin{quote}
\begin{alltt}\tiny
(assign(X,A),State) -->> OState :-     % assignment
    lookup(X,State,_),
    (A,State) -->> (ValA,S),               
    bindval(X,ValA,S,OState),!.    

(put(A),State) -->> OState :-           % writing
    (A,State) -->> (ValA,OState),
    write(A),
    write(' is '),
    writeln(ValA),!.  
\end{alltt}
\end{quote}

\end{frame}




\end{document}
%%%%%%%%%%%%%%%%%%%%%%%%%%% end of template1.tex %%%%%%%%%%%%%%%%%%%%%%%%%%%%%%%%

