% Notes:
% introduce semantics as follows:
% try to explain what 1+1 does to somebody who has no knowledge of the natural numbers or addition�
% one way to proceed is this:
% first show that 1+1 is an instance of a larger collection of terms I can write -- that is, specify a grammar for these terms
% next we build a *computational* model of the natural numbers:
%  \emptyset
% \{ \emptyset \}
% \{ \emptyset, \{ \emptyset \}\}
% \{ \emptyset, \{ \emptyset \}, \{ \emptyset, \{ \emptyset \}\}\}
% �.
% notice that this is a definition of the natural numbers without any reference to the "standard" symbols that we usually employ,
% also the numbers are ordered via set inclusion as you would expect from the natural numbers.
% We refer to this set as Nat.
% Also observe that the arity of each set represents exactly what we expect from a natural number -- for example if we count to
% three we would expect three "items".
% Now, this notation is cumbersome, we can introduce a function that takes advantage of the fact that the natural numbers
% are ordered and given any natural number in set notation will compute its successor:
%   s: Nat -> Nat with the definition s(A) = A \cup \{ A \}
% with this function we can represent the set Nat much easier
% \emptyset
% s(\emptyset)
% s(s(\emptyset))
% s(s(s(\emptyset)))
% �..
% We now have a nice computational model of counting: we simply apply the successor function s as many times to the set representing zero
% as there are items to be counted.
% In addition to counting our original problem of 1+1 also dealt with addition, so we need to construct a computational model of
% addition call the function add: Nat x Nat -> Nat with
%   add(X,0) = X
%   add(X,s(Y)) = s(add(X,Y))
%
% 
\documentclass{beamer}
% definitions for slides for the semantics course - CSC501
% Lutz Hamel, (c) 2006

\usetheme{Warsaw}
\usepackage{bussproofs}
\usepackage{amsmath}
\usepackage{amssymb}
\usepackage{latexsym}
\usepackage{xypic}
\usepackage{alltt}
\usepackage{rotating}
\usepackage{graphicx}

\newcommand{\ul}[1]{\underline{#1}}
\newcommand{\orbar}{\,|\,}
\newcommand{\co}{\,\colon\;}
\newcommand{\syntaxset}[1]{\ensuremath{\mbox{\bf #1}}}
\newcommand{\nonterm}[1]{\ensuremath{\mbox{#1}}}
\newcommand{\term}[1]{\ensuremath{\mbox{\bf #1}}}
\newcommand{\ifstmt}[3]{\ensuremath{{\bf if}\; {#1}\;{\bf then}\;{#2}\;{\bf else}\;{#3}\;\term{end}}}
\newcommand{\whilestmt}[2]{\ensuremath{{\bf while}\; {#1}\;{\bf do}\;{#2}\; \term{end}}}
\newcommand{\funcstmt}[3]{\ensuremath{{\bf fun}\; {#1}\; {\bf is}\; {#2} \; {\bf return}\; {#3}}}
\newcommand{\localstmt}[1]{\ensuremath{{\bf local}\; {#1}}}
\newcommand{\pairmap}[3]{\ensuremath{( {#1}, {#2} ) \mapsto {#3}}}
\newcommand{\single}[1]{\ensuremath{\langle {#1}\rangle}}
\newcommand{\pair}[2]{\ensuremath{({#1}, {#2} )}}
\newcommand{\triple}[3]{\ensuremath{\langle {#1}, {#2}, {#3} \rangle}}
\newcommand{\cond}[3]{\ensuremath{({#1}?\,{#2} : {#3})}}
\newcommand{\condbar}[3]{\ensuremath{({#1}\rightarrow{#2} \mid {#3})}}
\newcommand{\sem}[2]{\ensuremath{{#1}-\!\!-\!\!\!\!\gg{#2}}}
\newcommand{\liftfunc}[1]{\ensuremath{\lfloor{#1}\rfloor}}
\newcommand{\bcode}{\begin{alltt}\scriptsize}
\newcommand{\ecode}{\end{alltt}}
\newcommand{\comment}[1]{}

\newcommand{\fframe}[1]{
\begin{center}
\fbox{
\begin{minipage}{3.6in}
{#1}
\end{minipage}
}
\end{center}
}







\begin{document}

\begin{frame}{Natural Semantics}

Goals:
\begin{itemize}
\item Define the syntax of a simple imperative language
\item Define a semantics using {\em natural deduction}\footnote{Natural deduction is an 
instance of first-order logic; that is, it is the formal language of first-order logic
coupled with a ``natural" deductive system based on proof trees.}
\end{itemize}
\end{frame}

\begin{frame}{A Simple Imperative Language}
\scriptsize
The following is the grammar
 $G=(\Gamma=T\cup N,\rightarrow,\gamma)$ for our simple imperative language with $T$ and $N$ as the obvious sets\footnote{What are they?}, $\gamma=C$ the start symbol (denoted as $C^*$ in the rule set), and $\rightarrow$ defined as,
{
\[
\begin{array}{rcl}
\nonterm{A} &\rightarrow& D \orbar V \orbar \nonterm{A} +\nonterm{A} \orbar \nonterm{A} - \nonterm{A} \orbar 
	\nonterm{A} * \nonterm{A}\orbar ( \nonterm{A} )\\

\nonterm{B} &\rightarrow& \nonterm{T}  \orbar \nonterm{A} = \nonterm{A} \orbar
	\nonterm{A} \le \nonterm{A} \orbar ! \nonterm{B} \orbar \nonterm{B} \&\& \nonterm{B} \orbar 
	\nonterm{B} || \nonterm{B} \orbar (\nonterm{B})\\

\nonterm{C}^* &\rightarrow& \mbox{\bf skip} \orbar \nonterm{V} := \nonterm{A} \orbar\nonterm{C}\; ; \nonterm{C} \orbar
	\mbox{\bf if} \; \nonterm{B}\; \mbox{\bf then}\;\nonterm{C}\; \mbox{\bf else}\; \nonterm{C} \; \term{end}\orbar
	\mbox{\bf while}\; \nonterm{B}\; \mbox{\bf  do}\; \nonterm{C}\; \term{end}\\

\nonterm{T} &\rightarrow& \mbox{\bf true} \orbar \mbox{\bf false}\\

\nonterm{D} &\rightarrow& \nonterm{Q} \orbar - \nonterm{Q}\\

\nonterm{Q} &\rightarrow& \term{0}\, \nonterm{Q} \orbar \ldots \orbar  \term{9} \,\nonterm{Q} \orbar \term{0} \orbar \ldots\orbar\term{9}\\

\nonterm{V} &\rightarrow& \term{a}\,\nonterm{V} \orbar \ldots\orbar \term{z}\,\nonterm{V} \orbar \term{a} \orbar \ldots \term{z}
\end{array}
\]
}
\end{frame}




\begin{frame}{Semantics}
%\scriptsize
Our goal is to define a semantics
for each syntactic component of our language so that composing the semantics of each component
will then give us a semantics of programs written in that language.

\vspace{.1in}
Furthermore, 
we want to construct our models or semantics using predicates, relations, and functions; that means
we need to describe our syntax using {\em sets}.\footnote{\tiny Read Sections 1.1 and 2.1 and 2.2 in the book by David Schmidt.}

\vspace{.1in}

We
\begin{itemize}
\item need a more mathematical view of syntax -- {\em abstract syntax}
\item ignore pragmatics like operator precedence and actual  parsing
\item introduce  {\em syntactic sets} sometimes also called {\em syntactic domains}  (not to be confused with semantic domains!)
\end{itemize}

\end{frame}

\begin{frame}{Semantics}
As stated above, we want to construct our models using predicates, relations, and functions but when we look at our grammar $G$
we only have a single set: $L(G)$,
\[
L(G) = \{ q \mid \gamma \Rightarrow^* q \wedge q \in T^*\}
\]
There is not much we can do with this set because we don't see the individual syntactic structures in the strings on this set.

\vspace{.1in}

{\bf Idea:} What if we introduce a set for each non-terminal - non-terminals tend to capture the structure of syntactic units.
\end{frame}

\begin{frame}{Syntactic Sets}

\scriptsize

Grammar:
\[
\begin{array}{rcl}
\nonterm{A} &\rightarrow& D \orbar V \orbar \nonterm{A} +\nonterm{A} \orbar \nonterm{A} - \nonterm{A} \orbar 
	\nonterm{A} * \nonterm{A}\orbar ( \nonterm{A} )\\

\nonterm{B} &\rightarrow& \nonterm{T}  \orbar \nonterm{A} = \nonterm{A} \orbar
	\nonterm{A} \le \nonterm{A} \orbar ! \nonterm{B} \orbar \nonterm{B} \&\& \nonterm{B} \orbar 
	\nonterm{B} || \nonterm{B} \orbar (\nonterm{B})\\

\nonterm{C}^* &\rightarrow& \mbox{\bf skip} \orbar \nonterm{V} := \nonterm{A} \orbar\nonterm{C}\; ; \nonterm{C} \orbar
	\mbox{\bf if} \; \nonterm{B}\; \mbox{\bf then}\;\nonterm{C}\; \mbox{\bf else}\; \nonterm{C} \; \term{end}\orbar
	\mbox{\bf while}\; \nonterm{B}\; \mbox{\bf  do}\; \nonterm{C}\; \term{end}\\

\nonterm{T} &\rightarrow& \mbox{\bf true} \orbar \mbox{\bf false}\\

\nonterm{D} &\rightarrow& \nonterm{Q} \orbar - \nonterm{Q}\\

\nonterm{Q} &\rightarrow& \term{0}\, \nonterm{Q} \orbar \ldots \orbar  \term{9} \,\nonterm{Q} \orbar \term{0} \orbar \ldots\orbar\term{9}\\

\nonterm{V} &\rightarrow& \term{a}\,\nonterm{V} \orbar \ldots\orbar \term{z}\,\nonterm{V} \orbar \term{a} \orbar \ldots \term{z}
\end{array}
\]

Syntactic Sets:
\begin{itemize}
\item $\syntaxset{T} = \{  \mbox{\bf true} ,\mbox{\bf false} \}$.
\item $\syntaxset{I} = \{ q \mid \nonterm{D} \Rightarrow^* q \wedge q \in T^*\}$. 
\item $\syntaxset{Loc} = \{ q \mid \nonterm{V} \Rightarrow^* q \wedge q \in T^*\}$ .
\item $\syntaxset{Aexp} = \{ q \mid \nonterm{A} \Rightarrow^* q \wedge q \in T^*\}$ .
\item $\syntaxset{Bexp} = \{ q \mid \nonterm{B} \Rightarrow^* q \wedge q \in T^*\}$ .
\item $\syntaxset{Com} = L(G) = \{ q \mid \nonterm{C} \Rightarrow^* q \wedge q \in T^*\}$. 
\end{itemize}

\end{frame}


\begin{frame}{IMP - A Simple Imperative Language}

{\bf Syntactic Sets} (syntax only!!! not to be confused with the mathematical notions of integers, booleans, or variables, {\it etc.}):\footnote{This material is based on the book by Glynn Winskel, ``The Formal Semantics 
of Programming Languages.''}
\begin{description}
\item[\syntaxset{I}] This set consists of all positive and negative integer {\bf digits} including zero 
\item[\syntaxset{T}] Truth constants \term{true} and \term{false}.
\item[\syntaxset{Loc}] Locations (variable names as strings).
\item[\syntaxset{Aexp}] Arithmetic expressions
\item[\syntaxset{Bexp}] Boolean expressions
\item[\syntaxset{Com}] Commands
\end{description}

{\bf NOTE:} $\syntaxset{I} \ne \mathbb{I}$, where $\mathbb{I}$ is the set of all integer {\em values}.

{\bf NOTE:} $\syntaxset{T} \ne \mathbb{B}$, where $\mathbb{B}$ is the set of all boolean {\em values}.

\end{frame}

\begin{frame}{Syntax meets Semantics}
\scriptsize

The two notes on the previous slide put us squarely in the middle of {\em syntax meets semantics}.  The separation 
of these two is especially critical for numbers because our use of standard notation has blurred the fact that the number
symbols we write are indeed only symbols that need some sort of interpretation.  The fact that the notion of a number is
independent of the symbols we write is highlighted by the fact that we can define the natural numbers without number symbols:
\[
\begin{array}{l}
\emptyset \\
\{\emptyset \} \\  
\{ \emptyset , \{\emptyset \} \}\\
\{ \emptyset , \{\emptyset \}, \{ \emptyset , \{\emptyset \} \} \}\\
\{ \emptyset , \{\emptyset \}, \{ \emptyset , \{\emptyset \} \} , \{ \emptyset , \{\emptyset \}, \{ \emptyset , \{\emptyset \} \} \}\}\\
\vdots
\end{array}
\]
Note that each set has precisely as many elements as the standard notation denotes.\footnote{This can easily be extended to the integers $\mathbb{I}$, see {\tiny http://wiki.math.toronto.edu/TorontoMathWiki/index.php/Definition:\_Set\_of\_Integers}}  
The set of all these sets make up the natural numbers $\mathbb{N}$.
Furthermore, the sets are ordered according to the subset relation,
\[
\emptyset \subset \{\emptyset \} \subset \{ \emptyset , \{\emptyset \} \} \subset \{ \emptyset , \{\emptyset \}, \{ \emptyset , \{\emptyset \} \} \} \subset \cdots
\]

\end{frame}

\begin{frame}{Syntax meets Semantics}
\scriptsize

Since our standard notation is nothing but a syntactic representation we can write a grammar and then an interpretation for the symbols, 
\[
\begin{array}{rcl}
\nonterm{N} &\rightarrow& \nonterm{D} \orbar \nonterm{D}\nonterm{N}\\
\nonterm{D} &\rightarrow& 1 \orbar 2 \orbar 3 \orbar 4 \orbar 5 \orbar 6 \orbar 7 \orbar 8 \orbar 9 \orbar 0 
\end{array}
\]
with the {\em interpretation} over the set of all derived strings,
\[
\begin{array}{rcl}
0 & \mapsto & \emptyset \\
1 & \mapsto & \{\emptyset \} \\  
2 & \mapsto & \{ \emptyset , \{\emptyset \} \}\\
3 & \mapsto &\{ \emptyset , \{\emptyset \}, \{ \emptyset , \{\emptyset \} \} \}\\
4 & \mapsto & \{ \emptyset , \{\emptyset \}, \{ \emptyset , \{\emptyset \} \} , \{ \emptyset , \{\emptyset \}, \{ \emptyset , \{\emptyset \} \} \}\}\\
\vdots
\end{array}
\]
Given this interpretation, what {\em value} does the symbol $22$ represent.
\end{frame}

\begin{frame}{Syntax meets Semantics}
\scriptsize

Why is this interesting?  It shows that the mathematical notion of $\mathbb{N}$ is not tied to our syntactic representation.
In fact, we could come up with a brand new way of writing the natural numbers by shifting the numeric keys on my computer keyboard:
\[
\begin{array}{rcl}
\nonterm{N} &\rightarrow& \nonterm{D} \orbar \nonterm{D}\nonterm{N}\\
\nonterm{D} &\rightarrow& ! \orbar @ \orbar \# \orbar \$ \orbar \% \orbar \wedge \orbar \& \orbar * \orbar ( \orbar ) 
\end{array}
\]
with the {\em interpretation}, call it $h$, over all derived strings,
\[
\begin{array}{rcl}
) & \mapsto & \emptyset \\
! & \mapsto & \{\emptyset \} \\  
@ & \mapsto & \{ \emptyset , \{\emptyset \} \}\\
\# & \mapsto &\{ \emptyset , \{\emptyset \}, \{ \emptyset , \{\emptyset \} \} \}\\
\$ & \mapsto & \{ \emptyset , \{\emptyset \}, \{ \emptyset , \{\emptyset \} \} , \{ \emptyset , \{\emptyset \}, \{ \emptyset , \{\emptyset \} \} \}\}\\
\vdots
\end{array}
\]
Given this interpretation, what {\em value} does the symbol $@@$ represent.
\end{frame}

\begin{frame}{Syntax meets Semantics}
\small

This idea can also be represented as the following diagram.  Let $G$ be a grammar for the syntax of some natural number symbols,
then $L(G)$ is the set of all the symbols we can write, consider,
{\scriptsize
\[
\begin{array}{rcl}
\nonterm{N}^* &\rightarrow& \nonterm{D} \orbar \nonterm{D}\nonterm{N}\\
\nonterm{D} &\rightarrow& ! \orbar @ \orbar \# \orbar \$ \orbar \% \orbar \wedge \orbar \& \orbar * \orbar ( \orbar ) 
\end{array}
\]
}
Then an interpretation $h$ as defined in the previous slide is
\[
\xymatrix{
\mathbb{N}\\
L(G)\ar[u]^h
}
\]
That is, the interpretation $h$ takes each symbol in $L(G)$ and assigns it an element in $\mathbb{N}$ -- assigns it {\em meaning}!
We will see this as a recurring theme in this class.
\end{frame}


\begin{frame}{Formation Rules for Syntactic Sets}
\small

Back to our small imperative language, we have the {\em deductively} defined syntactic sets:
Syntactic Sets:
\begin{itemize}
\item $\syntaxset{T} = \{  \mbox{\bf true} ,\mbox{\bf false} \}$.
\item $\syntaxset{I} = \{ q \mid \nonterm{D} \Rightarrow^* q \wedge q \in T^*\}$. 
\item $\syntaxset{Loc} = \{ q \mid \nonterm{V} \Rightarrow^* q \wedge q \in T^*\}$ .
\item $\syntaxset{Aexp} = \{ q \mid \nonterm{A} \Rightarrow^* q \wedge q \in T^*\}$ .
\item $\syntaxset{Bexp} = \{ q \mid \nonterm{B} \Rightarrow^* q \wedge q \in T^*\}$ .
\item $\syntaxset{Com} = L(G) = \{ q \mid \nonterm{C} \Rightarrow^* q \wedge q \in T^*\}$. 
\end{itemize}

There is another way to form the syntactic sets -- {\em inductively} defined syntactic sets.

\end{frame}

\begin{frame}{Formation Rules for Syntactic Sets}
\scriptsize

Let us consider the syntactic set \syntaxset{Aexp}.  Our grammar production for arithmetic expressions is
\[
\nonterm{A} \rightarrow D \orbar V \orbar \nonterm{A} +\nonterm{A} \orbar \nonterm{A} - \nonterm{A} \orbar 
	\nonterm{A} * \nonterm{A}\orbar ( \nonterm{A} )
\]
We already know: $\syntaxset{Aexp} = \{ q \mid A \Rightarrow^* q \wedge q \in T^*\}$.

\vspace{.1in}
Given the above production we can define membership in the syntactic set as follows,
\begin{minipage}[t]{2in}
\begin{itemize}
\item $n \in  \syntaxset{Aexp}$ if $n \in \syntaxset{I}$,
\item $x \in \syntaxset{Aexp}$ if $x\in \syntaxset{Loc}$,
\item $a_0 + a_1 \in \syntaxset{Aexp}$ if $a_0, a_1 \in \syntaxset{Aexp}$,
\end{itemize}
\end{minipage}
\begin{minipage}[t]{2in}
\begin{itemize}
\item $a_0 - a_1 \in \syntaxset{Aexp}$ if $a_0, a_1 \in \syntaxset{Aexp}$,
\item $a_0 * a_1 \in \syntaxset{Aexp}$ if $a_0, a_1 \in \syntaxset{Aexp}$,
\item $( a )\in \syntaxset{Aexp}$ if $a \in \syntaxset{Aexp}$.
\end{itemize}
\end{minipage}

\vspace{.1in}

This is an {\em inductive definition} of the set \syntaxset{Aexp}: we start with the ``primitives" $n$ and
$x$ and then build more and more complex expressions as part of \syntaxset{Aexp}.

\vspace{.1in}

It is easy to see that the inductive definition and the production rule describe the same structures, that is,
given a production we can easily generate an inductive definition of the underlying syntactic set and, conversely, given an inductive definition of the syntactic set we can construct a production.
{\em This means we can use productions and inductive set definitions interchangeably.}\footnote{\tiny Later on we will learn techniques that allow us to prove that these are equivalent definitions.}

\end{frame}


\begin{frame}{Formation Rules for Syntactic Sets}
Observations for \syntaxset{Aexp}:
\begin{itemize}
\item $\syntaxset{I} \subset \syntaxset{Aexp}$,
\item $\syntaxset{Loc} \subset \syntaxset{Aexp}$,
\item contains structured {\em syntactic} objects that {\bf look like} arithmetic expressions,
\[
\syntaxset{Aexp} = \{1,5,x,\mbox{price},5-3*y,x+1,\ldots\}.
\]
\end{itemize}
\end{frame}


\begin{frame}{Formation Rules for Syntactic Sets}

If the deductively defined set and the inductively defined set are equivalent, why bother introducing the inductively defined set? 

\vspace{.2in}

Turns out many of our proofs have to establish properties over entire syntactic sets.  The inductive nature of the syntactic sets allows us to use a 
proof technique called {\em structural induction} to show that a property holds for the entire syntactic set.

\end{frame}



\begin{frame}{\large Evaluation of Arithmetic Expressions}
%\scriptsize
Why are we so keen on representing syntax as sets?  

\vspace{.1in}

We want to model computation denoted by the syntax in a mathematically meaningful
way, that is, we want to use {\em functions} and {\em relations} to {\em interpret} the syntax and model a computation.

\vspace{.1in}

Functions and relations need sets as their domains and co-domains.

\end{frame}

\begin{frame}{\large Evaluation of Arithmetic Expressions}

In order to accomplish our goal we need define one additional entity.  Let $\Sigma = \syntaxset{Loc} \rightarrow \mathbb{I}$  be the {\em set of all states}, where each element $\sigma \in \Sigma$ is a {\em state} and is considered
a function of the form,
\[
\sigma\co \syntaxset{Loc} \rightarrow \mathbb{I}
\]
where $\mathbb{I}$ represents the integer values.
A state is a {\em function} such that $\sigma(x)$ will give the value of the {\em contents} 
for any $x \in \syntaxset{Loc}$.

\vspace{.1in}

The set $\Sigma$ contains a distinguished state, $\sigma_0$, called the {\em initial state}, where $\sigma_0(x) = 0$, for all $x \in \syntaxset{Loc}$.

\end{frame}



\begin{frame}{\large Evaluation of Arithmetic Expressions}
We define the {\em evaluation}  or {\em interpretation} for arithmetic expressions as a function
from $\syntaxset{Aexp} \times \Sigma$ into $\mathbb{I}$, that is, we define an evaluation
call it $\mapsto$ as the function
\[
\mapsto \co \syntaxset{Aexp} \times \Sigma \rightarrow \mathbb{I}
\]
such that, given an arithmetic expression $a \in \syntaxset{Aexp}$, some state $\sigma \in \Sigma$, and 
an appropriate value $k \in \mathbb{I}$, we write
\[
\pair{a}{\sigma} \mapsto k 
\]
We say that the function evaluates an arithmetic expression  $a$ in the context of a state $\sigma$ to 
the outcome $k$.

\vspace{.1in}

The pair $\pair{a}{\sigma}$ is called a {\em configuration}.

\end{frame}

\begin{frame}{\large Evaluation of Arithmetic Expressions}

\small
We need one more auxiliary function before we can define the actual evaluation function,
\[
\mathit{eval}\co \syntaxset{I} \rightarrow \mathbb{I}
\]
Given an element of the syntactic representation of numbers \syntaxset{I}, this
function will return the mathematical equivalent of this representation in $\mathbb{I}$.

\vspace{.1in}

Let $6 \in \syntaxset{I}$, then $\mathit{eval}(6) = 6$, where the latter $6$ is in $\mathbb{I}$.
Another, perhaps more telling example is that our grammar allows us to generated numbers of
the form `$007$' for example.  But $\mathit{eval}(007) = 7$.  That is, $007\in\syntaxset{I}$ and
$7\in\mathbb{I}$.

\vspace{.1in}

We are not very formal about this and will
drop this operation in later definitions, where it is clear from context that some $n \in \syntaxset{I}$
represents a value in $\mathbb I$.
\end{frame}

\begin{frame}{Natural Semantics}
With all this machinery we are finally in the position to write down rules that define the
evaluation function `$\mapsto$' for arithmetic expressions.

\vspace{.1in}

Observe that the rules mirror precisely the inductive definition of the set \syntaxset{Aexp},
in this way we are assured that the evaluation function is defined for all possible values/elements
of the set \syntaxset{Aexp}.

\vspace{.1in}

The rules will be rules in the style of natural deduction.
\end{frame}

\begin{frame}{Natural Deduction}

\small


Natural deduction is a formal system that has terms, functions, and relations as alphabet
and that has two kinds of rules:
\begin{description}

\item[Inference Step]\hspace{1in}\\
\begin{prooftree}
\AxiomC{premise$_1$}
\AxiomC{$\cdots$}
\AxiomC{premise$_n$}
\RightLabel{\hspace{.1in}\small (condition)}
\TrinaryInfC{conclusion}
 \end{prooftree}

\item[Assumption]\hspace{1in}\\  
\begin{prooftree}
\AxiomC{}
\UnaryInfC{conclusion}
\end{prooftree}

\end{description}

{\bf NOTE:} An inference step is valid if all its premises are true.

{\bf NOTE:} An assumption is an inference step without premises (or with a premise that is always true).

\vspace{.1in}

%Read the paper ``Natural Semantics'' by G. Kahn, sections 1 through 4.
\end{frame}

\begin{frame}{\large Evaluation of Arithmetic Expressions}
\begin{description}

\item[Evaluation of numbers]\hspace{1in}\\  
\begin{prooftree}
\AxiomC{}
\UnaryInfC{$( n, \sigma ) \mapsto \mathit{eval}(n)$}
\end{prooftree}

\item[Evaluation of locations]\hspace{1in}\\
\begin{prooftree}
\AxiomC{}
\UnaryInfC{$( x, \sigma ) \mapsto \sigma(x)$}
\end{prooftree}

\item[Evaluation of sums]\hspace{1in}\\
\begin{prooftree}
\AxiomC{$( a_0,\sigma) \mapsto k_0$}
\AxiomC{$( a_1,\sigma) \mapsto k_1$}
\RightLabel{\hspace{.1in}\small where $k= k_0 + k_1$}
\BinaryInfC{$( a_0 + a_1, \sigma) \mapsto k$}
 \end{prooftree}
\end{description}

with $k$, $k_0$, $k_1  \in \mathbb{I}$, $a_0$, $a_1 \in \syntaxset{Aexp}$, and 
$\sigma \in \Sigma$.

\end{frame}

\begin{frame}{\large Evaluation of Arithmetic Expressions}
\small
{\bf Observation:} We are abusing notation slightly.  Technically, both the premises and the conclusions should evaluate to a boolean
value which in our case is not true, the function $\mapsto$ is declared as,
\[
\mapsto \co \syntaxset{Aexp} \times \Sigma \rightarrow \mathbb{I}
\]
That is, it evaluates to an integer value.  We can easily remedy this by turning the function into a {\em predicate}:
\[
\mapsto \co \syntaxset{Aexp} \times \Sigma \times \mathbb{I}  \rightarrow \mathbb{B}
\]
so the rule for the evaluation of variable names would become something like,
\begin{prooftree}
\AxiomC{}
\UnaryInfC{$( x, \sigma,\sigma(x)) \mapsto \mbox{\bf true}$}
\end{prooftree}
Which is a cumbersome notation and we will stick with our original notation remembering that we can always make this rigorous by turning
the $\mapsto$ function into a predicate.
\end{frame}


\begin{frame}{\large Evaluation of Arithmetic Expressions}
\begin{description}
\item[Evaluation of subtractions]\hspace{1in}\\
\begin{prooftree}
\AxiomC{$( a_0,\sigma) \mapsto k_0$} 
 \AxiomC{$( a_1,\sigma) \mapsto k_1$}
 \RightLabel{\hspace{.1in}\small where $k = k_0 - k_1$}
 \BinaryInfC{$( a_0 - a_1, \sigma) \mapsto k$}
 \end{prooftree}
 
 \item[Evaluation of products]\hspace{1in}\\
\begin{prooftree}
\AxiomC{$( a_0,\sigma) \mapsto k_0$} 
\AxiomC{$( a_1,\sigma) \mapsto k_1$}
\RightLabel{\hspace{.1in}\small where $k = k_0 \times k_1$}
\BinaryInfC{$( a_0 * a_1, \sigma) \mapsto k$}
\end{prooftree}

 \item[Evaluation of parenthesized terms]\hspace{1in}\\
\begin{prooftree}
\AxiomC{$( a,\sigma) \mapsto k$} 
\UnaryInfC{$( (a), \sigma) \mapsto k$}
\end{prooftree}
\end{description}

with $k$, $k_0$, $k_1  \in \mathbb{I}$, $a$,$a_0$, $a_1 \in \syntaxset{Aexp}$, and 
$\sigma \in \Sigma$.
\end{frame}

\begin{frame}{Arithmetic Expression Summary}
\scriptsize
%\begin{prooftree}
\AxiomC{}
\RightLabel{\hspace{.1in}for $n\in \syntaxset{I}$}
\UnaryInfC{$( n, \sigma ) \mapsto \mathit{eval}(n)$}
\DisplayProof\\
\vspace{.2in}
%\end{prooftree}
%\begin{prooftree}
\AxiomC{}
\RightLabel{\hspace{.1in}for $x\in \syntaxset{Loc}$}
\UnaryInfC{$( x, \sigma ) \mapsto \sigma(x)$}
\DisplayProof\\
\vspace{.2in}
%\end{prooftree}
%\begin{prooftree}
\AxiomC{$( a_0,\sigma) \mapsto k_0$}
\AxiomC{$( a_1,\sigma) \mapsto k_1$}
\RightLabel{\hspace{.1in}\small where $k= k_0 + k_1$}
\BinaryInfC{$( a_0 + a_1, \sigma) \mapsto k$}
\DisplayProof\\
\vspace{.2in}
% \end{prooftree}
%\begin{prooftree}
\AxiomC{$( a_0,\sigma) \mapsto k_0$} 
 \AxiomC{$( a_1,\sigma) \mapsto k_1$}
 \RightLabel{\hspace{.1in}\small where $k = k_0 - k_1$}
 \BinaryInfC{$( a_0 - a_1, \sigma) \mapsto k$}
\DisplayProof\\
\vspace{.2in}
% \end{prooftree}
%\begin{prooftree}
\AxiomC{$( a_0,\sigma) \mapsto k_0$} 
\AxiomC{$( a_1,\sigma) \mapsto k_1$}
\RightLabel{\hspace{.1in}\small where $k = k_0 \times k_1$}
\BinaryInfC{$( a_0 * a_1, \sigma) \mapsto k$}
\DisplayProof\\
\vspace{.2in}
%\end{prooftree}
\AxiomC{$( a,\sigma) \mapsto k$} 
\UnaryInfC{$( (a), \sigma) \mapsto k$}
\DisplayProof\\
\vspace{.2in}
with $k$, $k_0$, $k_1  \in \mathbb{I}$, $a$,$a_0$, $a_1 \in \syntaxset{Aexp}$, and 
$\sigma \in \Sigma$.
\end{frame}

\begin{frame}{\large Evaluation of Arithmetic Expressions}

{\bf Observation:} the line
\begin{quote}
with $k$, $k_0$, $k_1  \in \mathbb{I}$, $a$,$a_0$, $a_1 \in \syntaxset{Aexp}$, and $\sigma \in \Sigma$.
\end{quote}
on the previous slide and all our semantic definitions is absolutely necessary.  We are dealing with first order logic 
and therefore all variables need to be quantified.  Stating
\[
k, k_0, k_1  \in \mathbb{I}
\]
is the same thing as saying
\[
\forall k, k_0, k_1  \in \mathbb{I}
\]
That is, in all our semantic rules we have universal quantification.  If we did not state this then the variables in the rules would be
considered free variables and it possible to show that inferencing with free variables can get us into trouble.
\end{frame}

\begin{frame}{\large Evaluation of Arithmetic Expressions}

{\bf Observation:} The inference rules are a structural definition of a function over the syntactic set \syntaxset{Aexp} which together with
an appropriate state maps each
syntactic element in that set into a value in the set of all integer values, $\mathbb{I}$, i.e., the meaning of an syntactic expression given some state is an
integer value,
\[
\xymatrix{
\mathbb{I}\\
\syntaxset{Aexp}\times\Sigma\ar[u]
}
\]

\end{frame}


\begin{frame}{\large Evaluation of Arithmetic Expressions}

{\bf Example:} Let $ae \equiv (2 * 3) + 5$, prove that the semantic value of this expression in some state $\sigma$ is equal  to $11$ using the rules  above
{\scriptsize
\begin{prooftree}
\AxiomC{}
\UnaryInfC{$( 2,\sigma) \mapsto \mathit{eval}(2)=2$}
\AxiomC{}
\UnaryInfC{$( 3,\sigma) \mapsto \mathit{eval}(3)=3$}
\BinaryInfC{$( 2 * 3,\sigma) \mapsto 6$}
\UnaryInfC{$( (2 * 3),\sigma) \mapsto 6$}
\AxiomC{}
\UnaryInfC{$( 5,\sigma) \mapsto \mathit{eval}(5)=5$}
\BinaryInfC{$( (2*3)+5, \sigma ) \mapsto 11$}
\end{prooftree}
}
$\Rightarrow$ The final result does not depend on the state -- all constant expressions.
\end{frame}

\begin{frame}{\large Evaluation of Arithmetic Expressions}

{\bf Example:}Now, let $ae \equiv v + 1$, where $v \in \syntaxset{Loc}$, prove that the semantic value of this expression in some
state $\sigma \in \Sigma$ is equal to $\sigma(v)+1$.
\begin{prooftree}
\AxiomC{}
\UnaryInfC{$( v,\sigma )  \mapsto \sigma(v)$}
\AxiomC{}
\UnaryInfC{$( 1,\sigma ) \mapsto 1$}
\BinaryInfC{$( v + 1, \sigma) \mapsto \sigma(v) + 1$}
\end{prooftree}
$\Rightarrow$ We cannot fully evaluate this expression because we don't know enough
about the state $\sigma$.
\end{frame}

\begin{frame}{\large Evaluation of Arithmetic Expressions}

{\bf Example:} Now, let $ae \equiv v + 1$, where $v \in \syntaxset{Loc}$, prove that the semantic value of this expression in the initial state
$\sigma_0 \in \Sigma$ is equal to $1$.
\begin{prooftree}
\AxiomC{}
\UnaryInfC{$( v,\sigma_0 )  \mapsto \sigma_0(v)=0$}
\AxiomC{}
\UnaryInfC{$( 1,\sigma_0 ) \mapsto 1$}
\BinaryInfC{$( v + 1, \sigma_0) \mapsto 1$}
\end{prooftree}
$\Rightarrow$ We can fully evaluate this expression because we know the definition of
the  initial state $\sigma_0$.
\end{frame}

\end{document}
%%%%%%%%%%%%%%%%%%%%%%%%%%% end of template1.tex %%%%%%%%%%%%%%%%%%%%%%%%%%%%%%%%

