\documentclass{beamer}
% definitions for slides for the semantics course - CSC501
% Lutz Hamel, (c) 2006

\usetheme{Warsaw}
\usepackage{bussproofs}
\usepackage{amsmath}
\usepackage{amssymb}
\usepackage{latexsym}
\usepackage{xypic}
\usepackage{alltt}
\usepackage{rotating}
\usepackage{graphicx}

\newcommand{\ul}[1]{\underline{#1}}
\newcommand{\orbar}{\,|\,}
\newcommand{\co}{\,\colon\;}
\newcommand{\syntaxset}[1]{\ensuremath{\mbox{\bf #1}}}
\newcommand{\nonterm}[1]{\ensuremath{\mbox{#1}}}
\newcommand{\term}[1]{\ensuremath{\mbox{\bf #1}}}
\newcommand{\ifstmt}[3]{\ensuremath{{\bf if}\; {#1}\;{\bf then}\;{#2}\;{\bf else}\;{#3}\;\term{end}}}
\newcommand{\whilestmt}[2]{\ensuremath{{\bf while}\; {#1}\;{\bf do}\;{#2}\; \term{end}}}
\newcommand{\funcstmt}[3]{\ensuremath{{\bf fun}\; {#1}\; {\bf is}\; {#2} \; {\bf return}\; {#3}}}
\newcommand{\localstmt}[1]{\ensuremath{{\bf local}\; {#1}}}
\newcommand{\pairmap}[3]{\ensuremath{( {#1}, {#2} ) \mapsto {#3}}}
\newcommand{\single}[1]{\ensuremath{\langle {#1}\rangle}}
\newcommand{\pair}[2]{\ensuremath{({#1}, {#2} )}}
\newcommand{\triple}[3]{\ensuremath{\langle {#1}, {#2}, {#3} \rangle}}
\newcommand{\cond}[3]{\ensuremath{({#1}?\,{#2} : {#3})}}
\newcommand{\condbar}[3]{\ensuremath{({#1}\rightarrow{#2} \mid {#3})}}
\newcommand{\sem}[2]{\ensuremath{{#1}-\!\!-\!\!\!\!\gg{#2}}}
\newcommand{\liftfunc}[1]{\ensuremath{\lfloor{#1}\rfloor}}
\newcommand{\bcode}{\begin{alltt}\scriptsize}
\newcommand{\ecode}{\end{alltt}}
\newcommand{\comment}[1]{}

\newcommand{\fframe}[1]{
\begin{center}
\fbox{
\begin{minipage}{3.6in}
{#1}
\end{minipage}
}
\end{center}
}







\begin{document}


\begin{frame}[fragile]{Iteration Program Correctness}

\small
{\bf Correctness of Programs without Iteration:}

\vspace{.1in}

Let $\rm pre$ and $\rm post$ be predicates over states, and let $p$
be a program, then 
\[
[\sem{(p,s)}{Q} \wedge {\rm pre}(s) \Rightarrow {\rm pos}t(Q)]\mbox{ implies $p$ is correct}
\]
for all $s \in {\tt state}$.  We often write
\[
\{{\rm pre}\} p \{{\rm post}\} \mbox{ implies $p$ is correct}
\]
where the state $s$ is implicit.



\end{frame}

\begin{frame}[fragile]{Program Correctness}

\small

Unfortunately this notion of correctness does not extend to the general case of programs
with iteration.\footnote{Pre- and post conditions cannot be used to show program correctness
for all values of the loop index - you would need some sort of inductive argument as we will
see a little later on.}
However, we can still use this proof rule to prove {\em specific} programs correct.  This
program computes the factorial of $3$:
\[
p \equiv {\rm while}(le(1,i), {\rm assign}(z,{\rm mult}(z,i)) \; {\rm seq}\; {\rm assign}(i,{\rm sub}(i,1)))
\]
with the precondition
\[
{\rm pre}(S) \equiv {\rm lookup}(i,S,{\color{red}3}) \wedge {\rm lookup}(z,S,1)
\]
and the postcondition
\[
{\rm post}(Q) \equiv {\rm lookup}(i,Q,0) \wedge {\rm lookup}(z,Q,3!)
\]
That is, the program is correct iff
\[
\{{\rm pre}\}\; p \; \{{\rm post}\}
\]
\end{frame}

\begin{frame}[fragile]{Program Correctness}

\small

This works because for specific values for $i$ we can {\em unfold} the iteration and view
$p$ essentially as a sequence of statements without iteration:
\[
\begin{array}{rl}
p'\equiv & {\rm assign}(z,{\rm mult}(z,i)) \; {\rm seq}\; {\rm assign}(i,{\rm sub}(i,1)) \;{\rm seq}\\
	& {\rm assign}(z,{\rm mult}(z,i)) \; {\rm seq}\; {\rm assign}(i,{\rm sub}(i,1)) \;{\rm seq}\\
          & {\rm assign}(z,{\rm mult}(z,i)) \; {\rm seq}\; {\rm assign}(i,{\rm sub}(i,1)) 
\end{array}
\]
And is clear that our proof rule
\[
\{{\rm pre}\}\; p' \; \{{\rm post}\}
\]
applies.\footnote{One way to show this is to develop pre and post conditions for each statement and
then compose them to get the overall correctness proof.}
\end{frame}

\begin{frame}[fragile]{Program Correctness}

\small

Now consider our program for any value of i:
\[
p \equiv {\rm while}(le(1,i), {\rm assign}(z,{\rm mult}(z,i)) \; {\rm seq}\; {\rm assign}(i,{\rm sub}(i,1)))
\]
with the precondition
\[
{\rm pre}(S) \equiv {\rm lookup}(i,S,{\color{red}vi}) \wedge {\rm lookup}(z,S,1)
\]
and the postcondition
\[
{\rm post}(Q) \equiv {\rm lookup}(i,S,0) \wedge {\rm lookup}(z,S,{\color{red}vi!})
\]
Now, in order to prove $p$ correct we would have to show that our proof rule
\[
\{{\rm pre}\}\; p \; \{{\rm post}\}
\]
holds for every value of $vi$, that is, every possible unfolding of the while loop.
Clearly that is not possible.
\end{frame}

\begin{frame}[fragile]{Program Correctness}

To get around this we could probably come up  with some sort of inductive argument
on the loop iteration (recall our induction on states a while back).

\vspace{.1in}

Turns out that there is a very clever way to do this without induction:  {\em loop invariants}

\vspace{.1in}

Loop invariants allow us to perform "inductionless induction".
\end{frame}

\begin{frame}[fragile]{Program Correctness}

Definition of {\em loop invariant},\footnote{\url{https://en.wikipedia.org/wiki/Loop_invariant}}

\begin{quote}
In computer science, a loop invariant is a property of a program loop that is true before (and after) each iteration. It is a logical assertion, sometimes checked within the code by an assertion call. Knowing its invariant(s) is essential in understanding the effect of a loop.

\vspace{.2in}

In formal program verification loop invariants are expressed by formal predicate logic and used to prove properties of loops and by extension algorithms that employ loops (usually correctness properties). The loop invariants will be true on entry into a loop and following each iteration, so that on exit from the loop both the loop invariants and the loop termination condition can be guaranteed.
\end{quote}

\end{frame}




\begin{frame}[fragile]{Program Correctness}
In this new approach we divide programs with loops into parts:
\[
\{{\rm pre}\} \;{\rm init}\; {\bf seq}\; \whilestmt{b}{c}\;\{post\}
\]

where {\rm init} is code that sets up the loop.

\vspace{.1in}

We now introduce a new predicate called {\rm inv} that captures the loop invariant
in such a way that it relates one iteration to the next in a similar sense as does the inductive step
during an inductive argument.
\end{frame}

\begin{frame}[fragile]{Program Correctness \& Iteration}

We augment our proof procedure with an additional predicate on states: $\rm inv$.

\vspace{.1in}

With this we can set up the predicates for the loop as follows:
\begin{quote}
\begin{minipage}{1.5in}
\{$\rm pre$\}\\
{\rm init} {\bf seq}\\
\{$\rm inv$\}  \\                                  
{\bf while} $b$ {\bf do}  \\                       
{\tt\verb"   "}\{$\rm inv$ $\wedge$ $b$\}  \\
{\tt\verb"   "}$c$    \\
{\tt\verb"   "}\{$\rm inv$\}  \\                            
\{$\rm inv$ $\wedge$ $\neg b$\}\\
\{$\rm post$\}  
\end{minipage}
\begin{minipage}{2in}
{\bf NOTE: } We now have pre- and postconditions for each statement in this iterative
program. {\em These conditions will hold in all iterations of the loop}.
\end{minipage}
\end{quote}                          
\end{frame}


\begin{frame}[fragile]{Program Correctness}
This gives us a new proof rule for {\em partial correctness} of loops:
\begin{quote}
if\\
{\verb"    "}$\{\rm pre\}$ {\rm init} $\{\rm inv\}$ $\wedge$\\
{\verb"    "}$\{{\rm inv} \wedge b\}$ c $\{\rm inv\}$ $\wedge$\\
{\verb"    "}$({\rm inv} \wedge \neg b) \Rightarrow {\rm post}$\\
then\\
{\verb"    "}`${\rm init}\; {\bf seq}\; \whilestmt{b}{c}$' is correct\\
\end{quote}

{\bf NOTE: } We call this partial correctness because we make no assertions about termination.
All we assert is that, if the computation terminates, then it will be correct.

\end{frame}


\begin{frame}[fragile]{Program Correctness}

\small

Or written in our notation, partial correctness of loops:
\begin{quote}
if\\
{\verb"    "}$\sem{({\rm init},S)}{Q} \wedge [{\rm pre}(S)\Rightarrow {\rm inv}(Q)]\; \wedge$\\ 
{\verb"    "}$\sem{(c,S)}{Q} \wedge \sem{(b,S)}{B} \wedge [({\rm inv}(s) \wedge B)\Rightarrow {\rm inv}(Q)]\; \wedge$\\
{\verb"    "}$\sem{(b,T)}{B} \wedge [(inv(T) \wedge \neg\,B) \Rightarrow post(T)]$\\
{\verb"    "}\\
then 
\begin{quote}
{\verb"    "}`${\rm init}\; {\bf seq}\; \whilestmt{b}{c}$'  is correct\\
\end{quote}
\end{quote}


\vspace{.2in}
Finding loop invariants automatically  is an active research, good candidates for loop invariants are usually expression
that involve the loop counter.
\end{frame}
\end{document}

