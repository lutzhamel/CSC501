\documentclass{beamer}
% definitions for slides for the semantics course - CSC501
% Lutz Hamel, (c) 2006

\usetheme{Warsaw}
\usepackage{bussproofs}
\usepackage{amsmath}
\usepackage{amssymb}
\usepackage{latexsym}
\usepackage{xypic}
\usepackage{alltt}
\usepackage{rotating}
\usepackage{graphicx}

\newcommand{\ul}[1]{\underline{#1}}
\newcommand{\orbar}{\,|\,}
\newcommand{\co}{\,\colon\;}
\newcommand{\syntaxset}[1]{\ensuremath{\mbox{\bf #1}}}
\newcommand{\nonterm}[1]{\ensuremath{\mbox{#1}}}
\newcommand{\term}[1]{\ensuremath{\mbox{\bf #1}}}
\newcommand{\ifstmt}[3]{\ensuremath{{\bf if}\; {#1}\;{\bf then}\;{#2}\;{\bf else}\;{#3}\;\term{end}}}
\newcommand{\whilestmt}[2]{\ensuremath{{\bf while}\; {#1}\;{\bf do}\;{#2}\; \term{end}}}
\newcommand{\funcstmt}[3]{\ensuremath{{\bf fun}\; {#1}\; {\bf is}\; {#2} \; {\bf return}\; {#3}}}
\newcommand{\localstmt}[1]{\ensuremath{{\bf local}\; {#1}}}
\newcommand{\pairmap}[3]{\ensuremath{( {#1}, {#2} ) \mapsto {#3}}}
\newcommand{\single}[1]{\ensuremath{\langle {#1}\rangle}}
\newcommand{\pair}[2]{\ensuremath{({#1}, {#2} )}}
\newcommand{\triple}[3]{\ensuremath{\langle {#1}, {#2}, {#3} \rangle}}
\newcommand{\cond}[3]{\ensuremath{({#1}?\,{#2} : {#3})}}
\newcommand{\condbar}[3]{\ensuremath{({#1}\rightarrow{#2} \mid {#3})}}
\newcommand{\sem}[2]{\ensuremath{{#1}-\!\!-\!\!\!\!\gg{#2}}}
\newcommand{\liftfunc}[1]{\ensuremath{\lfloor{#1}\rfloor}}
\newcommand{\bcode}{\begin{alltt}\scriptsize}
\newcommand{\ecode}{\end{alltt}}
\newcommand{\comment}[1]{}

\newcommand{\fframe}[1]{
\begin{center}
\fbox{
\begin{minipage}{3.6in}
{#1}
\end{minipage}
}
\end{center}
}







\begin{document}

\begin{frame}[fragile]{Semantics of the Source Language}

\scriptsize
Source Language:
\tiny
\begin{verbatim}
  A ::= n
     |  x
     |  add(A,A)
     |  sub(A,A)
     |  mult(A,A)

  B ::= true
     |  false
     |  eq(A,A)
     |  le(A,A)
     |  not(B)
     |  and(B,B)
     |  or(B,B)

  C ::= skip
     |  assign(x,A)
     |  seq(C,C)
     |  if(B,C,C)
     |  whiledo(B,C)
\end{verbatim}
\end{frame}

\begin{frame}[fragile]{Semantics of the Source Language}

\scriptsize
Semantics are the same compared to our initial simple imperative language; arithmetic
expressions:

\tiny
\begin{alltt}
(C,_) -->> C :-                    % constants
    int(C),!.

(X,State) -->> Val :-              % variables
    atom(X),
    lookup(X,State,Val),!.

(add(A,B),State) -->> Val :-       % addition
    (A,State) -->> ValA,
    (B,State) -->> ValB,
    Val xis ValA + ValB,!.

(sub(A,B),State) -->> Val :-       % subtraction
    (A,State) -->> ValA,
    (B,State) -->> ValB,
    Val xis ValA - ValB,!.

(mult(A,B),State) -->> Val :-     % multiplication
    (A,State) -->> ValA,
    (B,State) -->> ValB,
    Val xis ValA * ValB,!.
\end{alltt}
\end{frame}

\begin{frame}[fragile]{Semantics of the Source Language}

\scriptsize
Boolean
expressions:

\tiny
\begin{alltt}
(true,_) -->> true :- !.               % constants

(false,_) -->> false :- !.             % constants

(eq(A,B),State) -->> Val :-            % equality
    (A,State) -->> ValA,
    (B,State) -->> ValB,
    Val xis (ValA =:= ValB),!.

(le(A,B),State) -->> Val :-            % le
    (A,State) -->> ValA,
    (B,State) -->> ValB,
    Val xis (ValA =< ValB),!.

(not(A),State) -->> Val :-             % not
    (A,State) -->> ValA,
    Val xis (not ValA),!.

(and(A,B),State) -->> Val :-           % and
    (A,State) -->> ValA,
    (B,State) -->> ValB,
    Val xis (ValA and ValB),!.

(or(A,B),State) -->> Val :-            % or
    (A,State) -->> ValA,
    (B,State) -->> ValB,
    Val xis (ValA or ValB),!.
\end{alltt}
\end{frame}


\begin{frame}[fragile]{Semantics of the Source Language}

\scriptsize
Statements:

\tiny
\begin{alltt}
(skip,State) -->> State :- !.          % skip                                                                                                

(assign(X,A),State) -->> OState :-     % assignment                                                                                          
    (A,State) -->> ValA,
    put(X,ValA,State,OState),!.

(seq(C0,C1),State) -->> OState :-      % composition, seq                                                                                    
    (C0,State) -->> S0,
    (C1,S0) -->> OState,!.

(if(B,C0,_),State) -->> OState :-     % if                                                                                                   
    (B,State) -->> true,
    (C0,State) -->> OState,!.

(if(B,_,C1),State) -->> OState :-     % if                                                                                                   
    (B,State) -->> false,
    (C1,State) -->> OState,!.

(whiledo(B,_),State) -->> OState :-    % while                                                                                               
    (B,State) -->> false,
    State=OState,!.

(whiledo(B,C),State) -->> OState :-    % while                                                                                               
    (B,State) -->> true,
    (C,State) -->> SC,
    (whiledo(B,C),SC) -->> OState,!.
\end{alltt}
\end{frame}

\begin{frame}[fragile]{Semantics of the Target Language}

\scriptsize
Target Language:
\tiny
\begin{verbatim}
 prog ::= [ cmseq ]   |  [ ]

 cmseq ::=  cm  |  cm , cmseq

 cm ::= push(V)
     |  add
     |  sub
     |  mult
     |  and
     |  or
     |  neg
     |  eq
     |  le
     |  pop(x)
     |  label(L)
     |  jmp(L)
     |  jmpt(L)
     |  jmpf(L)
     |  stop

 V ::=  x  |  n  | true | false

 L ::=  <alpha string>
\end{verbatim}

 A state in our machine is a term of arity two where the first component
 is an integer stack used for expression evaluation and the second component
 is a binding environment for variables:
\begin{quote}
    '(Stack,Environment)'
\end{quote}

\end{frame}

\begin{frame}[fragile]{Semantics of the Target Language}

\scriptsize
Flow of control instructions:  perhaps the most surprising part of the semantics for our target language
is the notion of a {\em continuation}.  A continuation is a way to model the address space of the target machine
so that we can perform {\em jumps} to labels.

\tiny
\begin{alltt}
% the predicate '(+Syntax,+Continuation,+State) -->> -State' computes
% the semantic value for each syntactic structure

([],_,State) -->> State :- !.         % an empty instruction sequence is a noop

([stop|_],_,State) -->> State :- !.   % the 'stop' instruction ignores the rest of the program

([jmp(L)|_],Cont,State) -->> OState :-
    afindlabel(L,Cont,JT),
    (JT,Cont,State) -->> OState,!.

([jmpt(_)|P],Cont,([false|Stk],Env)) -->> OState :-
        (P,Cont,(Stk,Env)) -->> OState,!.
    
([jmpt(L)|_],Cont,([true|Stk],Env)) -->> OState :-
         afindlabel(L,Cont,JT),(JT,Cont,(Stk,Env)) -->> OState,!.
    
([jmpf(L)|_],Cont,([false|Stk],Env)) -->> OState :-
         afindlabel(L,Cont,JT),(JT,Cont,(Stk,Env)) -->> OState,!.

([jmpf(_)|P],Cont,([true|Stk],Env)) -->> OState :-
        (P,Cont,(Stk,Env)) -->> OState,!.
\end{alltt}

\end{frame}

\begin{frame}[fragile]{Semantics of the Target Language}

\scriptsize
A continuation is a copy of the original program and we use it to look up jump targets:

{\tiny
\begin{alltt}
% the predicate 'afindlabel(+Label,+Continuation,-JumpTarget)'
% looks up a label definition in the continutation and returns its associate code.
:- dynamic afindlabel/3.

afindlabel(L,[label(L)|P],[label(L)|P]).

afindlabel(L,[_|P],JT) :-
    afindlabel(L,P,JT).

afindlabel(_,[],_) :-
   writeln('ERROR: label not found.'),!,fail.
\end{alltt}
}
\end{frame}

\begin{frame}[fragile]{Semantics of the Target Language}

\scriptsize
Computational instructions:
\tiny
\begin{verbatim}
%%% computational instructions
([Instr|P],Cont,State) -->> OState :-  % interpret an instruction sequence.
    (Instr,Cont,State) -->> IState,
    (P,Cont,IState) -->> OState,!.

(push(C),_,(Stk,Env)) -->> ([C|Stk],Env) :-        % constants
    int(C),!.

(push(X),_,(Stk,Env)) -->> ([ValX|Stk],Env) :-     % variables
    atom(X),
    alookup(X,Env,ValX),!.

(pop(X),_,([ValA|Stk],Env)) -->> (Stk,OEnv) :-  % store
    aput(X,ValA,Env,OEnv),!.

(add,_,([ValB,ValA|Stk],Env)) -->> ([Val|Stk],Env) :-  % addition
    Val xis ValA + ValB,!.

...

(and,_,([ValB,ValA|Stk],Env)) -->> ([Val|Stk],Env) :-  % and
    Val xis (ValA and ValB),!.

...

(neg,_,([ValA|Stk],Env)) -->> ([Val|Stk],Env) :-  % not
    Val xis (not ValA),!.

(label(_),_,State) -->> State :- !.
\end{verbatim}
\end{frame}

\begin{frame}[fragile]{Semantics of the Target Language}

\scriptsize
Interpreting the target language in its model,
\begin{alltt}
?- ['target.pl'].
%  xis.pl compiled 0.00 sec, 6,824 bytes
% target.pl compiled 0.00 sec, 14,600 bytes
true.

?- assert(program([push(1),push(2),add])).
true.

?- program(P), (P,{\color{red}P},([],e)) -->> S.
P = [push(1), push(2), add],
S = ([3], e).

?- 
\end{alltt}
The {\tt P} in red is the continuation.
\end{frame}

\end{document}

