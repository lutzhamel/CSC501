\documentclass{beamer}
% definitions for slides for the semantics course - CSC501
% Lutz Hamel, (c) 2006

\usetheme{Warsaw}
\usepackage{bussproofs}
\usepackage{amsmath}
\usepackage{amssymb}
\usepackage{latexsym}
\usepackage{xypic}
\usepackage{alltt}
\usepackage{rotating}
\usepackage{graphicx}

\newcommand{\ul}[1]{\underline{#1}}
\newcommand{\orbar}{\,|\,}
\newcommand{\co}{\,\colon\;}
\newcommand{\syntaxset}[1]{\ensuremath{\mbox{\bf #1}}}
\newcommand{\nonterm}[1]{\ensuremath{\mbox{#1}}}
\newcommand{\term}[1]{\ensuremath{\mbox{\bf #1}}}
\newcommand{\ifstmt}[3]{\ensuremath{{\bf if}\; {#1}\;{\bf then}\;{#2}\;{\bf else}\;{#3}\;\term{end}}}
\newcommand{\whilestmt}[2]{\ensuremath{{\bf while}\; {#1}\;{\bf do}\;{#2}\; \term{end}}}
\newcommand{\funcstmt}[3]{\ensuremath{{\bf fun}\; {#1}\; {\bf is}\; {#2} \; {\bf return}\; {#3}}}
\newcommand{\localstmt}[1]{\ensuremath{{\bf local}\; {#1}}}
\newcommand{\pairmap}[3]{\ensuremath{( {#1}, {#2} ) \mapsto {#3}}}
\newcommand{\single}[1]{\ensuremath{\langle {#1}\rangle}}
\newcommand{\pair}[2]{\ensuremath{({#1}, {#2} )}}
\newcommand{\triple}[3]{\ensuremath{\langle {#1}, {#2}, {#3} \rangle}}
\newcommand{\cond}[3]{\ensuremath{({#1}?\,{#2} : {#3})}}
\newcommand{\condbar}[3]{\ensuremath{({#1}\rightarrow{#2} \mid {#3})}}
\newcommand{\sem}[2]{\ensuremath{{#1}-\!\!-\!\!\!\!\gg{#2}}}
\newcommand{\liftfunc}[1]{\ensuremath{\lfloor{#1}\rfloor}}
\newcommand{\bcode}{\begin{alltt}\scriptsize}
\newcommand{\ecode}{\end{alltt}}
\newcommand{\comment}[1]{}

\newcommand{\fframe}[1]{
\begin{center}
\fbox{
\begin{minipage}{3.6in}
{#1}
\end{minipage}
}
\end{center}
}







\begin{document}

\begin{frame}[fragile]{Pre- and Postconditions}

\scriptsize

Up to now our program specifications only considered the effect of the program on
the state, we simply assumed that the state before the program executes is suitable.

\vspace{.1in}

Let us refine our notion of program specification a little bit to include a specification
of what we expect the state to look like before our program executes.

\vspace{.1in}

We do this with a pair of predicates: the precondition $\rm pre$ and the postcondition $\rm post$,\footnote{\tiny Definitions due to Wikipedia.}
\begin{description}
\item[Precondition] --
A precondition is a predicate
that must always be true just prior to the execution of some section of code. 
If a precondition is violated,
the effect of the section of code becomes undefined and thus may or may not
carry out its intended work. 

\item[Postcondition] --
A postcondition is a predicate that must always be true just after the execution of some section of code.  If the postcondition is violated then the section of code did not execute correctly.
\end{description}

\end{frame}

\begin{frame}[fragile]{Pre- and Postconditions}

\small

Formally we state this as follows:
\[
\forall s,\exists P,Q\; [\sem{(P,s}{Q} \wedge {\rm pre}(s) \Rightarrow{\rm post}(Q)],
\]
for some program $P$, where $s$ and $Q$ are states.

\vspace{.1in}

\begin{quote}
\it The program is correct if when the precondition holds on the initial state then this implies that the
postcondition holds on the final state.
\end{quote}
\end{frame}



\begin{frame}[fragile]{Pre- and Postconditions}

\small

Consider again our swap program.  Being more explicit about the contents
of the initial state we have the precondition
\[
{\rm pre}(R) \equiv  {\rm lookup}(x,R,vx) \wedge {\rm lookup}(y,R,vy)
\]
and the postcondition
\[
{\rm post}(T) \equiv  {\rm lookup}(x,T,vy) \wedge {\rm lookup}(y,T,vx)
\]
\end{frame}


\begin{frame}[fragile]{Pre- and Postconditions}

{\scriptsize
\begin{verbatim}
% swap-prepost.pl
:-['sem.pl'].

:- >>> 'show that program P="assign(t,x) seq assign(x,y) seq assign(y,t))"'.
:- >>> 'satisfies the program specification:'.
:- >>> ' pre(R) = lookup(x,R,vx) ^ lookup(y,R,vy)'.
:- >>> ' post(T) = lookup(x,T,vy) ^ lookup(y,T,vx)'.

program(assign(t,x) seq assign(x,y) seq assign(y,t)).                                              
                                                                                                   
:- >>> 'assert precondition'.                                                                      
:- asserta(lookup(x,s,vx)).
:- asserta(lookup(y,s,vy)).                                                                        
                                                                                                   
:- >>> 'show that postcondition holds'.                                                            
:- program(P),
     (P,s) -->> Q,
     lookup(x,Q,vy),
     lookup(y,Q,vx).                                                                               
\end{verbatim}
}
\end{frame}

\begin{frame}[fragile]{Pre- and Postconditions}

\small

Consider the program that computes the maximum value of two variables, $n$ and $m$,
and deposits the result into $z$.

\vspace{.1in}

The precondition is
\[
{\rm pre}(S) \equiv {\rm lookup}(m,S,vm) \wedge {\rm lookup}(n,S,vn)
\]
and the postcondition is
\[
{\rm post}(Q) \equiv {\rm lookup}(z,Q,VZ) \wedge V \mbox{ xis }({\rm max}(vm,vn)) \wedge VZ = V
\]
\end{frame}

\begin{frame}[fragile]{Pre- and Post-Conditions}

\tiny
\begin{verbatim}
% max-prepost.pl
:-['sem.pl'].

:- >>> 'show that program P="if(le(n,m),assign(z,m),assign(z,n))"'.
:- >>> 'satisfies the program specification:'.
:- >>> ' pre(S) = lookup(m,S,vm) ^ lookup(n,S,vm)'.
:- >>> ' post(Q) = lookup(z,Q,VZ)) ^ V xis max(vm,vn) ^ VZ = V'.

program(if(le(n,m),assign(z,m),assign(z,n))).

:- >>> 'assert precondition'.
:- asserta(lookup(m,s,vm)).
:- asserta(lookup(n,s,vn)).                                                                        
                                                                                                   
:- >>> 'show that postcondition holds; case analysis on values vm and vn'.                                                       
:- >>> 'case max(vm,vn)=vm'.                                                                       
:- asserta(vm xis max(vm,vn)).
% this implies that
:- asserta(true xis (vn =< vm)).
:- program(P),  (P,s) -->> Q, lookup(z,Q,VZ), V xis max(vm,vn), VZ = V.
:- retract(vm xis max(vm,vn)).
:- retract(true xis (vn =< vm)).

:- >>> 'case max(vm,vn)=vn'.
:- asserta(vn xis max(vm,vn)).
% this implies that
:- asserta(false xis (vn =< vm)).
:- program(P), (P,s) -->> Q, lookup(z,Q,VZ), V xis max(vm,vn), VZ = V.
:- retract(vn xis max(vm,vn)).
:- retract(false xis (vn =< vm)).
\end{verbatim}

\end{frame}

\end{document}

