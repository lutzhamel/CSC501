\documentclass{beamer}
% definitions for slides for the semantics course - CSC501
% Lutz Hamel, (c) 2006

\usetheme{Warsaw}
\usepackage{bussproofs}
\usepackage{amsmath}
\usepackage{amssymb}
\usepackage{latexsym}
\usepackage{xypic}
\usepackage{alltt}
\usepackage{rotating}
\usepackage{graphicx}

\newcommand{\ul}[1]{\underline{#1}}
\newcommand{\orbar}{\,|\,}
\newcommand{\co}{\,\colon\;}
\newcommand{\syntaxset}[1]{\ensuremath{\mbox{\bf #1}}}
\newcommand{\nonterm}[1]{\ensuremath{\mbox{#1}}}
\newcommand{\term}[1]{\ensuremath{\mbox{\bf #1}}}
\newcommand{\ifstmt}[3]{\ensuremath{{\bf if}\; {#1}\;{\bf then}\;{#2}\;{\bf else}\;{#3}\;\term{end}}}
\newcommand{\whilestmt}[2]{\ensuremath{{\bf while}\; {#1}\;{\bf do}\;{#2}\; \term{end}}}
\newcommand{\funcstmt}[3]{\ensuremath{{\bf fun}\; {#1}\; {\bf is}\; {#2} \; {\bf return}\; {#3}}}
\newcommand{\localstmt}[1]{\ensuremath{{\bf local}\; {#1}}}
\newcommand{\pairmap}[3]{\ensuremath{( {#1}, {#2} ) \mapsto {#3}}}
\newcommand{\single}[1]{\ensuremath{\langle {#1}\rangle}}
\newcommand{\pair}[2]{\ensuremath{({#1}, {#2} )}}
\newcommand{\triple}[3]{\ensuremath{\langle {#1}, {#2}, {#3} \rangle}}
\newcommand{\cond}[3]{\ensuremath{({#1}?\,{#2} : {#3})}}
\newcommand{\condbar}[3]{\ensuremath{({#1}\rightarrow{#2} \mid {#3})}}
\newcommand{\sem}[2]{\ensuremath{{#1}-\!\!-\!\!\!\!\gg{#2}}}
\newcommand{\liftfunc}[1]{\ensuremath{\lfloor{#1}\rfloor}}
\newcommand{\bcode}{\begin{alltt}\scriptsize}
\newcommand{\ecode}{\end{alltt}}
\newcommand{\comment}[1]{}

\newcommand{\fframe}[1]{
\begin{center}
\fbox{
\begin{minipage}{3.6in}
{#1}
\end{minipage}
}
\end{center}
}







\begin{document}

\begin{frame}[fragile]{Proofs: Program Correctness}
One of the great advantages of formal semantics is that we can actually prove that
a program will behave correctly for {\em all} expected input values.

\vspace{.1in}

In order for this to work we need the notion of a {\em program specification}.


\vspace{.1in}

The program specification act as the {\em yard stick} for the expected program behavior
for any set of input values.

\end{frame}

\begin{frame}[fragile]{Program Specifications}
$\Rightarrow$ A program specification is a universally quantified sentence over states in first order logic.

\vspace{.1in}

Consider the following program specification for some program $p$ and variables $x$ and $y$:
\[
\begin{array}{rl}
\forall s,\exists Q,VX,VY& [\sem{(p,s)}{Q} \wedge \\
   		& {\rm lookup}(y,s,VY) \wedge {\rm lookup}(x,Q,VY)\wedge \\
	           & {\rm lookup}(x,s,VX) \wedge{\rm lookup}(y,Q,VX) ]
\end{array}
\]
This specification states that running the program $p$ in state $s$ will give rise to some state $Q$.
Furthermore, looking up the variable $y$ in state $s$ is the same as looking up the variable $x$ in
state $Q$ and {\em vice versa}.

\vspace{.1in}

This is a program specification for a {\em swap} program that swaps the values of $x$ and $y$.
\end{frame}

\begin{frame}[fragile]{Program Specifications}

Now, consider the  program $p$ written in our simple language IMP defined in `sem.pl':
\[
p \equiv {\bf assign}(t, x)\; {\bf  seq}\; {\bf assign}(x,y) \; {\bf seq}\; {\bf assign}(y, t)
\]
Without formal semantics and a program specification we would  simply try ``a bunch'' of values, and if the results
look good we would infer that the program works.  But there will always be a doubt that it will work for all states
since trying a bunch of values does not constitute a proof.

\vspace{.1in}

However, given our formal semantics we can prove that this program {\em satisfies} the specification
and therefore we can prove that the program works for all possible states.
\end{frame}

\begin{frame}[fragile]{Program Specifications}

{\tiny
\begin{verbatim}
% swap.pl
:-['sem.pl'].

:- >>> 'show that program P="assign(t,x) seq assign(x,y) seq assign(y,t))"'.
:- >>> 'satisfies the program specification:'.
:- >>> ' (P,s)-->>Q,lookup(y,s,VY),lookup(x,Q,VY),lookup(x,s,VX),lookup(y,Q,VX)'.

program(assign(t,x) seq assign(x,y) seq assign(y,t)).                                              
                                                                                                   
:- asserta(lookup(x,s,vx)).
:- asserta(lookup(y,s,vy)).                                                                        
                                                                                                   
:- program(P),
     (P,s) -->> Q,
     lookup(y,s,VY),
     lookup(x,Q,VY),
     lookup(x,s,VX),                                                                               
     lookup(y,Q,VX).
\end{verbatim}
}
\end{frame}




\begin{frame}[fragile]{Program Specifications}
Now consider the program specification
{\scriptsize
\[
\begin{array}{rl}
\forall s,\exists Q,V1,V2& [\sem{(p,s)}{Q} \wedge \\
   		& {\rm lookup}(z,s,V1) \wedge {\rm lookup}(z,Q,V2) \wedge\\
		& V2 =  2 * V1 ]
\end{array}
\]

}
It is easy to see that the program $p \equiv {\bf assign}(z,{\bf mult}(2,z))$ satisfies the specification.

\vspace{.1in}

But so does this program $p \equiv {\bf assign}(z,{\bf add}(z,z))$.

\vspace{.3in}

$\Rightarrow$ Program specifications are {\em implementation independent}!
\end{frame}

\begin{frame}[fragile]{Program Specifications}

{\scriptsize
\begin{verbatim}
% double.pl
:-['sem.pl'].

:- >>> 'show that program P="assign(z,add(z,z)))"'.
:- >>> 'satisfies the program specification:'.
:- >>> ' (p,s) -->> Q,lookup(z,s,V1),lookup(z,Q,V2),V2 = 2*V1'.

program(assign(z,add(z,z))).

:- asserta(lookup(z,s,vz)).                                                                        
:- asserta(2*I xis I+I). % property of integers                                               
                                                                                                   
:- program(P),
      (P,s) -->> Q,
      lookup(z,s,V1),                                                                              
      lookup(z,Q,V2),
      V2 = 2 * V1.
\end{verbatim}
}
\end{frame}


\end{document}

