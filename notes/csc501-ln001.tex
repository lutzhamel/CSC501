\documentclass{beamer}
% definitions for slides for the semantics course - CSC501
% Lutz Hamel, (c) 2006

\usetheme{Warsaw}
\usepackage{bussproofs}
\usepackage{amsmath}
\usepackage{amssymb}
\usepackage{latexsym}
\usepackage{xypic}
\usepackage{alltt}
\usepackage{rotating}
\usepackage{graphicx}

\newcommand{\ul}[1]{\underline{#1}}
\newcommand{\orbar}{\,|\,}
\newcommand{\co}{\,\colon\;}
\newcommand{\syntaxset}[1]{\ensuremath{\mbox{\bf #1}}}
\newcommand{\nonterm}[1]{\ensuremath{\mbox{#1}}}
\newcommand{\term}[1]{\ensuremath{\mbox{\bf #1}}}
\newcommand{\ifstmt}[3]{\ensuremath{{\bf if}\; {#1}\;{\bf then}\;{#2}\;{\bf else}\;{#3}\;\term{end}}}
\newcommand{\whilestmt}[2]{\ensuremath{{\bf while}\; {#1}\;{\bf do}\;{#2}\; \term{end}}}
\newcommand{\funcstmt}[3]{\ensuremath{{\bf fun}\; {#1}\; {\bf is}\; {#2} \; {\bf return}\; {#3}}}
\newcommand{\localstmt}[1]{\ensuremath{{\bf local}\; {#1}}}
\newcommand{\pairmap}[3]{\ensuremath{( {#1}, {#2} ) \mapsto {#3}}}
\newcommand{\single}[1]{\ensuremath{\langle {#1}\rangle}}
\newcommand{\pair}[2]{\ensuremath{({#1}, {#2} )}}
\newcommand{\triple}[3]{\ensuremath{\langle {#1}, {#2}, {#3} \rangle}}
\newcommand{\cond}[3]{\ensuremath{({#1}?\,{#2} : {#3})}}
\newcommand{\condbar}[3]{\ensuremath{({#1}\rightarrow{#2} \mid {#3})}}
\newcommand{\sem}[2]{\ensuremath{{#1}-\!\!-\!\!\!\!\gg{#2}}}
\newcommand{\liftfunc}[1]{\ensuremath{\lfloor{#1}\rfloor}}
\newcommand{\bcode}{\begin{alltt}\scriptsize}
\newcommand{\ecode}{\end{alltt}}
\newcommand{\comment}[1]{}

\newcommand{\fframe}[1]{
\begin{center}
\fbox{
\begin{minipage}{3.6in}
{#1}
\end{minipage}
}
\end{center}
}







\begin{document}

\begin{frame}{CSC 501 -- Semantics of Programming Languages}
\begin{itemize}
\item Subtitle: An Introduction to Formal Methods.
\item	Instructor: Dr. Lutz Hamel
\item	Email: lutzhamel@uri.edu
\item	Office: Tyler, Rm 251
\end{itemize}

\vspace{.1in}

\begin{center}
    \includegraphics[height=35mm]{images/hokusai}
\end{center}

\end{frame}

\begin{frame}{Books}
There are no required books in this course; however, occasionally I will assign readings based on material
available on the web.
\end{frame}

\begin{frame}{Course Objectives}

The aim of this course is to
\begin{itemize}
\item Familiarize you with the basic techniques of applying formal methods to programming languages.

\item This includes constructing models for programming languages and using these models to prove properties such as correctness and equivalence of programs.

\item Look at all major programming language constructs including assignments, loops, type systems, and procedure calls together with their models.

\item Introduce mechanical theorem provers so that we can test and prove properties of non-trivial programs.
\end{itemize}
\end{frame}

\begin{frame}{Some Definitions}
\vspace{.1in}
{\bf Definition}: In {\bf programming language semantics} we are concerned with the {\em rigorous mathematical study} of the {\em meaning} of programming languages. The meaning of a language is given by a {\em formal system} that describes the possible computations expressible within that language.
\end{frame}

\begin{frame}{Some Definitions}
\vspace{.1in}
{\bf Definition:} In computer science and software engineering, {\bf formal methods} are  techniques for the specification, development and verification of software and hardware systems based on {\em formal systems}.

\end{frame}


\begin{frame}{Formal Systems}
\vspace{.1in}
{\bf Definition:}
A {\bf formal system} consists of a {\em formal language} and a set of  {\em inference rules}.
The formal language is composed of
primitive symbols that make up well formed formulas and the inference rules are used to derive expressions from other expressions within the formal system. A formal system may be formulated and studied for its intrinsic properties, or it may be intended as a description (i.e. a model) of external phenomena.\footnote{Wikipedia}

\vspace{.1in}
In order to be truly useful in computer science, we require our formal systems to be {\em machine executable}.

\end{frame}

\begin{frame}{Uses of Formal Methods}

\begin{description}
\item[\em Implementation Issues] Formally specified models can be considered machine-independent specifications of
program behavior. They can act as ``yard sticks'' for the correctness of program implementations, transformations, and
optimizations.
\item[\em Verification] Basis of methods for reasoning about program
properties (e.g. equivalence) and program specifications (program correctness).
\item[\em Language Design] Can bring to light ambiguities and unforeseen
subtleties in programming language constructs.
\end{description}
\end{frame}





\begin{frame}[fragile]{Observations}
When programming we can observe two mental activities:
\begin{itemize}
\item We construct {\em correct looking} programs - {\em syntactically} correct programs.
\item We construct {\em models} of the intended computation in our minds. Consider,
\bcode
x := 1
while (x <= 10) do
     writeln(x)
     x := x + 1
end whiledo
\ecode
Any person with some familiarity of programming immediately has a mental picture that this program
will generate a list of integers from 1 through 10.
\end{itemize}

\end{frame}

\begin{frame}{Programming Language Definitions}

Mirroring our intuition, language definitions consist of two parts:

\begin{description}
\item[\em Syntax] The formal description of the
{\bf structure} of  well-formed expressions, phrases, programs, etc.
\item[\em Semantics] The formal description of the {\bf  meaning} of the syntactic features of a programming language
usually understood in terms of the runtime {\bf behavior} each syntactic construct evokes.  The formal description of the behavior of all
the syntactic features of a language is considered a {\bf model} of the language.
\end{description}
\end{frame}

\begin{frame}{Evaluation/Interpretation}
Syntax and semantics of a programming language are usually related via an {\em evaluation relation}
or {\em interpretation}, say $h$.  Then we say that the interpretation $h$ takes each syntactic element and
maps it into the appropriate semantic construct.

\vspace{.1in}

We often represent this with the diagram
\[
\xymatrix{
\mathit{Semantics}\\
\mathit{Syntax}\ar[u]^h
}
\]
{\bf Note:} In order for the interpretation $h$ to make any sense we will have to define the syntax and
semantics in terms of sets.

\end{frame}


\begin{frame}{Formal Systems and Programs}

The formal systems we will be using in this course are:
\begin{itemize}
\item First-order logic extended with natural deduction -- natural semantics.
\item The {\em first order predicate calculus} (often also called first order logic) to construct semantics of programming languages.
\end{itemize}
\end{frame}

\begin{frame}{Readings}
\begin{itemize}
\item Read Chapter 0 in "Denotational Semantics" by David Schmidt (available from the course website).
\item Read Sections 2.1 and 2.2 in "Denotational Semantics" by David Schmidt.
\end{itemize}
\end{frame}


\end{document}
%%%%%%%%%%%%%%%%%%%%%%%%%%% end of template1.tex %%%%%%%%%%%%%%%%%%%%%%%%%%%%%%%%
